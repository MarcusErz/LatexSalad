\newpage
\chapter{Vergleich der graphen-basierten Algorithmen}
\label{chap:vgl}

Die empirische Anwendung der graphen-basierten Algorithmen auf Zeitreihendaten erfolgt, sowohl mit statischen als auch mit dynamischen Algorithmen. Im Rahmen des Forschungsprojekts wurde der Fokus auf vier verschiedene Algorithmen gelegt, die das Potential zur erfolgreichen Erkennung von Ausrei�ern in Zeitreihen hatten. Diese werden nachfolgend verglichen und die Erkenntnisse des Forschungsprojekts aus \autoref{chap:static}, sowie \autoref{chap:dynamic} festgehalten. Durch den Vergleich soll ermittelt werden, wie erfolgreich ein Algorithmus bei der Erkennung von Ausrei�ern in Daten ist. 


\begin{table}[H]
	\centering
	\begin{tabular}{p{0.19\linewidth}|P{0.1\linewidth}|P{0.13\linewidth}|P{0.14\linewidth}|P{0.14\linewidth}|P{0.13\linewidth}}
		\toprule
		& \textbf{Statisch} & \textbf{Dynamisch} & \textbf{Qualit�t Ausrei�er-erkennung Netzwerke} & \textbf{Qualit�t Ausrei�er-erkennung Zeitreihen} & \textbf{Performanz} \\
		\midrule
		IsoMap-based & \textbf{\textcolor{green}{+}} & \textbf{\textcolor{red}{-}} & \textbf{\textcolor{red}{-}} & \textbf{o}  & \textbf{\textcolor{green}{+}} \\
		\midrule
		Perculation-based & \textbf{\textcolor{green}{+}} & \textbf{\textcolor{red}{-}} & \textbf{\textcolor{red}{-}} & \textbf{\textcolor{green}{+}} & \textbf{\textcolor{green}{+}} \\
		\midrule
		NetSimile & \textbf{\textcolor{red}{-}} & \textbf{\textcolor{green}{+}} & \textbf{\textcolor{green}{+}} & \textbf{\textcolor{green}{++}} & \textbf{\textcolor{green}{+}} \\
		\midrule
		MIDAS & \textbf{\textcolor{red}{-}} & \textbf{\textcolor{green}{+}} & \textbf{\textcolor{green}{++}} & \textbf{\textcolor{red}{-}} & \textbf{o}  \\
		\bottomrule
	\end{tabular}
	\caption{Vergleich der Algorithmen}
	\label{tab:bewertung}
\end{table}


In der \autoref{tab:bewertung} sind die Kriterien zu entnehmen, die zum Vergleich der Algorithmen herangezogen wurden. So werden alle Algorithmen zun�chst einmal in statisch und dynamisch, entsprechend der Taxonomie des Forschungsprojekts, unterteilt. Im n�chsten Schritt werden die Ergebnisse aller Algorithmen hinsichtlich ihrer F�higkeit gegen�bergestellt, Ausrei�er in Graphen und Zeitreihen qualitativ zu erkennen. Zuletzt liegt der Fokus auf ihrer Performance im Vergleich zu den anderen Algorithmen. Nachfolgend werden die Ergebnisse ausgef�hrt.
\\ \\
\workTodo{die Vergleiche autoref einf�gen.}
Der IsoMap-based und der Perculation-based Algorithmus sind statische Algorithmen, die jeweils nur auf einem Graphen angewendet werden k�nnen. Aus diesem Grund eignen sich diese nicht f�r die Ausrei�ererkennung in Netzwerken, da hierbei mehrere Graphen �bergeben werden. Beide k�nnen jedoch erfolgreich auf Zeitreihen angewendet werden, wobei beim IsoMap-based Algorithmus die Ausrei�er nur teilweise erkennbar sind, da sich die Ausrei�er nur geringf�gig von den anderen Werten unterscheiden. Der Perculation-based Algorithmus hingegen zeigt eindeutige Ausrei�er, weshalb dieser Algorithmus geeignet f�r eine qualitative Ausrei�ererkennung ist. Bei den Ausrei�er-Typen \dq Signal-Aussetzer\dq\space und \dq Frequenz�nderung\dq\space k�nnen beide Algorithmen keine Ausrei�er identifizieren.  Die Performance beider Algorithmen ist gut, da sie die Berechnungen innerhalb weniger Sekunden durchf�hren.

Der NetSimile- und der MIDAS-Algorithmus k�nnen beide auf Netzwerkdatens�tzen angewandt werden, da sie dynamisch sind. Der Unterschied hierbei liegt bei dem Ausrei�er-Score. Beim NetSimile sind die erkannten Ausrei�er nahezu identisch, wodurch eine Klassifizierung unzureichend ist. Beim MIDAS hingegen sind die Ausschl�ge weitaus gr��er, wodurch eine Klassifizierung erzielt wird. Bei der Anwendung auf Zeitreihen liefert der NetSimile sehr gute Ergebnisse auf den Ausrei�ertypen, die verwendet wurden (vgl. \autoref{sec:ns-ts-1}). Da sich bei der Transformation von Zeitreihen zu Graphen die Gewichtung nur geringf�gig unterscheidet und der MIDAS-Algorithmus Ausrei�er nur bei einer hohen Kantengewichtung erkennt, fallen die Ergebnisse hierbei schlechter aus (vgl. \autoref{sec:resultsOTs}). Dementsprechend ist MIDAS f�r die Erkennung von Ausrei�ern in Zeitreihen eher ungeeignet. Bei der Performanz ben�tigte der NetSimile-Algorithmus urspr�nglich bis zu einer Stunde, da dieser rechenintensive Bibliotheken verwendete. Durch die Verwendung anderer Bibliotheken wurde eine Optimierung erreicht. Die Visualisierung des Graphens ist hierdurch zwar nicht mehr m�glich, jedoch wird die Rechenzeit auf wenige Sekunden reduziert. Der MIDAS ben�tigt hingegen mehrere Minuten, wodurch der NetSimile nach der Optimierung performanter geworden ist.