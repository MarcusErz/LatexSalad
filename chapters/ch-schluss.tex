\chapter{Fazit und Ausblick}
\label{chap:fua}

\section{Fazit}
\label{sec:fua-f}

Die Erkenntnisse und Ergebnisse des Forschungsprojekts zeigen deutlich, dass graphen-basierte Algorithmen erfolgreich auf Zeitreihendaten angewandt werden k�nnen. Im Speziellen erkennen dynamische Algorithmen verschiedenste Ausrei�ertypen mit einer hohen Qualit�t. Zudem ber�cksichtigen diese graphen-basierten Algorithmen den Faktor Zeit, der essentiell f�r diesen Datentyp ist.
Inbesondere der NetSimile-Algorithmus erf�llte, nach Optimierung, die Anforderungen des Forschungsprojekts. Dieser Algorithmus ist dynamisch, multidimensional, performant und die Erkennung von Ausrei�ern in Zeitreihen erfolgt mit einer hohen Genauigkeit. Der Schwerpunkt hierbei liegt bei der Wahl der richtigen Features. Strukturelle Merkmale sind bei vollst�ndig verkn�pften Graphen eher ungeeignet und sollten ersetzt werden durch Features wie bspw. die Gewichtung der Kanten. 

%\workTodo{Tabelle in den Anhang}
%\begin{table}[h!]
%	\centering
%	\begin{tabular}{P{0.3\linewidth}|P{0.31\linewidth}|P{0.31\linewidth}}
%		\toprule
%		& \textbf{Netsimile} & \textbf{MIDAS} \\
%		\midrule
%		AmpRise & 54 min 32 sec & 05 min 26 sec \\
%		\midrule
%		Drift & 57 min 29 sec & 05 min 22 sec \\
%		\midrule
%		Flatmiddle & 62 min 50 sec & 06 min 01 sec \\
%		\midrule
%		Increase Noise & 51 min 56 sec & 05 min 40 sec \\
%		\midrule
%		Jumps Down & 51 min 30 sec & 05 min 53 sec \\
%		\midrule
%		Jumps Up & 50 min 40 sec & 05 min 12 sec \\
%		\midrule
%		No Jump & 65 min 45 sec & 06 min 01 sec \\
%		\midrule
%		Peaks & 61 min 31 sec & 05 min 55 sec \\
%		\midrule
%		Sequence Change & 49 min 53 sec & 06 min 41 sec \\
%		\midrule
%		Darpa & 185 min 49 sec & 05 min 29 sec \\
%		\midrule
%		Enron & 01 min 47 sec & 05 min 14 sec \\
%		\bottomrule
%	\end{tabular}
%	\caption{�berblick der Performanz in [min]}
%	\label{tab:performance}
%\end{table}



\section{Ausblick}
\label{sec:fua-a}

Im Rahmen des Forschungsprojekts wurden drei Thematiken behandelt. Zun�chst einmal wurde eine M�glichkeit ermittelt Zeitreihendaten in einen Graphen umzuwandeln. In einem weiteren Schritt wurden graphen-basierte Algorithmen auf die transformierten Zeitreihendaten angewandt. Im Anschluss wurden diese Algorithmen hinsichtlich ihrer Eignung zur Ausrei�ererkennung verglichen.

Durch die erfolgreiche Transformation von Zeitreihendaten in Graphen kann der Vorgang ebenso f�r andere Datenkategorien herangezogen werden, unter der Pr�misse, dass Distanzen zwischen den einzelnen Elementen des Datensatzes gebildet werden k�nnen. So ist es im n�chsten Schritt m�glich bspw. Ausrei�er in Finanzdaten oder Bilddaten zu erkennen. 

Der NetSimile-Algorithmus kann durch neue Merkmale erg�nzt werden. So ist es zuk�nftig m�glich, neben den mathematischen Merkmalen, ebenso statistische Algorithmen als Feature einzusetzen. Dadurch ergibt sich eine erweiterte und optimierte M�glichkeit um Aussagen hinsichtlich Ausrei�ern treffen zu k�nnen. Eine heutige Problematik des NetSimile ist zum einen die, dass er nur den Ausrei�er-Graphen zur�ckgibt und zum anderen wird der Graph erst dann berechnet, wenn dieser alle Kanten eines Zeitintervalls beinhaltet. Hierbei k�nnen weitere Optimierungen folgen, die bspw. den Ausrei�er-Graphen auf Knoten oder Kanten untersucht, die f�r den Ausrei�er-Score am relevantesten sind. Zudem kann eine Methode ermittelt werden, die einen Graphen iterativ vergr��ert, damit eventuell schon vor dem vollst�ndigen Berechnen des Graphs bestimmt werden kann, ob es sich um einen Ausrei�er handelt. Zuletzt kann der NetSimile so erweitert werden, dass er in Echtzeit Ausrei�er, bspw. im IoT- oder Industrie 4.0-Umfeld, entdeckt. 
