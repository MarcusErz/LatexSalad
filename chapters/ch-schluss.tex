\chapter{Fazit und Ausblick}
\label{chap:fua}

\section{Fazit}
\label{sec:fua-f}

\workTodo{Fazit schreiben}
Der Perculation-based Algorithmus liefert bessere Ergebnisse f�r die meisten Ausrei�er-Typen als der Iso Map. Bei den Ausrei�er Typen ?Signal-Aussetzer? und ?Frequenz�nderung? k�nnen beide Algorithmen keine Ausrei�er identifizieren. 
Zudem sind beide Algorithmen f�r die statische Analyse ausgerichtet und k�nnten deswegen bspw. nur als Feature f�r den Netsimile verwendet werden. 
Der Netsimile hingegen erkennt auch diese beiden Ausrei�er Typen mit einer hohen Genauigkeit, aber ist weniger geeignet f�r die Ausrei�ererkennung in Netzwerken, da sich die Ausrei�er zu wenig voneinander unterscheiden und deswegen schwer zu interpretieren sind. Der MIDAS Algorithmus eignet sich hervorragend f�r die Ausrei�ererkennung in Netzwerken und liefert �hnliche Ausrei�er im Enron Datensatz wie der ver�ffentlichte Algorithmus SEDANSPOT. F�r die Ausrei�ererkennung in Zeitreihen ist der MIDAS aber eher ungeeignet. 
Zwecks Performanz konnte der Netsimile soweit optimiert werden, dass dieser in wenigen Sekunden durchgef�hrt werden kann, sowie die beiden statischen Algorithmen. MIDAS hingegen ben�tigt mehrere Minuten zur Analyse. 
Der Netsimile erf�llt demnach die Anforderungen am besten. Der Schwerpunkt hierbei liegt bei der Wahl der richtigen Features. Strukturelle Merkmale sind eher ungeeignet bei vollst�ndig verkn�pften Graphen und sollten ersetzt werden durch Features, wie bspw. die Gewichtung der Kanten. 


\begin{table}[h]
	\centering
	\begin{tabular}{p{0.19\linewidth}|P{0.1\linewidth}|P{0.13\linewidth}|P{0.13\linewidth}|P{0.13\linewidth}|P{0.14\linewidth}}
		\toprule
		& \textbf{Statisch} & \textbf{Dynamisch} & \textbf{Qualit�t Ausrei�er-Erkennung ENRON} & \textbf{Qualit�t Ausrei�er-Erkennung Zeitreihen} & \textbf{Performance} \\
		\midrule
		ISOMAP-based & \textbf{\textcolor{green}{+}} & \textbf{\textcolor{red}{-}} & \textbf{\textcolor{red}{-}} & \textbf{\textcolor{green}{+}} & \textbf{\textcolor{green}{+}} \\
		\midrule
		Perculation-based & \textbf{\textcolor{green}{+}} & \textbf{\textcolor{red}{-}} & \textbf{\textcolor{red}{-}} & \textbf{\textcolor{green}{++}} & \textbf{o} \\
		\midrule
		Netsimile & \textbf{\textcolor{red}{-}} & \textbf{\textcolor{green}{+}} & \textbf{\textcolor{green}{+}} & \textbf{\textcolor{green}{++}} & \textbf{\textcolor{green}{+}} \\
		\midrule
		MIDAS & \textbf{\textcolor{red}{-}} & \textbf{\textcolor{green}{+}} & \textbf{\textcolor{green}{++}} & \textbf{\textcolor{red}{-}} & \textbf{\textcolor{green}{+}} \\
		\bottomrule
	\end{tabular}
	\caption{Vergleich der Algorithmen}
	\label{tab:bewertung}
\end{table}



\begin{table}[h!]
	\centering
	\begin{tabular}{P{0.3\linewidth}|P{0.3\linewidth}|P{0.3\linewidth}}
		\toprule
		& \textbf{Netsimile} & \textbf{MIDAS} \\
		\midrule
		AmpRise & 54 min 32 sec & 05 min 26 sec \\
		\midrule
		Drift & 57 min 29 sec & 05 min 22 sec \\
		\midrule
		Flatmiddle & 62 min 50 sec & 06 min 01 sec \\
		\midrule
		Increase Noise & 51 min 56 sec & 05 min 40 sec \\
		\midrule
		Jumps Down & 51 min 30 sec & 05 min 53 sec \\
		\midrule
		Jumps Up & 50 min 40 sec & 05 min 12 sec \\
		\midrule
		No Jump & 65 min 45 sec & 06 min 01 sec \\
		\midrule
		Peaks & 61 min 31 sec & 05 min 55 sec \\
		\midrule
		Sequence Change & 49 min 53 sec & 06 min 41 sec \\
		\midrule
		Darpa & 185 min 49 sec & 05 min 29 sec \\
		\midrule
		Enron & 01 min 47 sec & 05 min 14 sec \\
		\bottomrule
	\end{tabular}
	\caption{�berblick der Performanz in [min]}
	\label{tab:performance}
\end{table}



\section{Ausblick}
\label{sec:fua-a}
Der Netsimile Algorithmus wurde bisher nur auf Zeitreihendaten und Graphen angewendet. Interessant w�ren jedoch noch Bilder, Biologische Daten und andere Netzwerke. Die einzige Voraussetzung ist die Berechnung einer Distanz zwischen einzelnen Elementen in diesen Daten. Zudem gibt es viele statische Algorithmen, die als Feature dem Netsimile hinzugef�gt werden k�nnen. Erkennt der Perculation Algorithmus bspw. Ausrei�er-Typen wie Increase-noise, Jumpsdown, Jumpsup und Flatmiddle, so kann nach weiteren Algorithmen gesucht werden, die die anderen Ausrei�er Typen erkennt. Zudem wird beim Netsimile Algorithmus lediglich der Graph zur�ckgegeben der als Ausrei�er identifiziert wurde, nicht aber der Grund daf�r. Hierf�r k�nnten weitere Untersuchungen durchgef�hrt werden, wie bspw. die Analyse der Features, welches sich am meisten von den anderen unterschieden hat oder einer anschlie�enden Analyse durch den Oddball Algorithmus, welche die Anomalien innerhalb des Graphens untersucht. 

Des weiteren k�nnte eine allgemeine Forschung gestartet werden, welche untersucht, welche Features f�r spezifische Anwendungsgebiete geeignet sein k�nnten. 


\workTodo{Ausblick schreiben}
