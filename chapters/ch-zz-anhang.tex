\chapter{OddBall}
\label{chap:a-ob}
\section{Threshold-Ergebnisse}
\label{chap:a-ob-th}
\begin{table}[H]
	\centering
	\includegraphics[width=\textwidth, height=0.72\textheight]{fig/tabelle-threshold-1}
	\caption{Threshold-Ergebnisse}
\end{table}

\section{Local-Outlier-Factor-Ergebnisse}
\label{chap:a-ob-lof}

\begin{table}[H]
\centering
\includegraphics[width=\textwidth]{fig/tabelle-lof-1}
\end{table}

\begin{table}[H]
\centering
\includegraphics[width=\textwidth]{fig/tabelle-lof-2}
\end{table}

\begin{table}[H]
\centering
\includegraphics[width=\textwidth]{fig/tabelle-lof-3}
\end{table}

\begin{table}[H]
\centering
\includegraphics[width=\textwidth]{fig/tabelle-lof-4}
\caption{Local-Outlier-Factor-Ergebnisse}
\end{table}


\section{Local-Outlier-Probability-Ergebnisse}
\label{chap:a-ob-loop}
\begin{table}[H]
	\centering
	\includegraphics[width=\textwidth]{fig/tabelle-loop-1}
\end{table}

\begin{table}[H]
	\centering
	\includegraphics[width=\textwidth]{fig/tabelle-loop-2}
\end{table}

\begin{table}[H]
\centering
\includegraphics[width=\textwidth]{fig/tabelle-loop-3}
\end{table}

\begin{table}[H]
\centering
\includegraphics[width=\textwidth]{fig/tabelle-loop-4}
\end{table}

\begin{table}[H]
\centering
\includegraphics[width=\textwidth]{fig/tabelle-loop-5}
\end{table}

\begin{table}[H]
\centering
\includegraphics[width=\textwidth]{fig/tabelle-loop-6}
\caption{Local-Outlier-Probability-Ergebnisse}
\end{table}

\section{Vergleich LOF und LoOP}
\label{chap:a-ob-lof-loop}

\begin{table}[H]
	\centering
	\includegraphics[width=\textwidth]{fig/tabelle-lof-loop}
	\caption{Vergleich LOF und LoOP}
\end{table}

\section{DBSCAN-Ergebnisse}
\label{chap:a-ob-dbscan}

\begin{table}[H]
	\centering
	\includegraphics[width=\textwidth]{fig/tabelle-dbscan-1}
\end{table}

\begin{table}[H]
\centering
\includegraphics[width=\textwidth]{fig/tabelle-dbscan-2}
\end{table}

\begin{table}[H]
\centering
\includegraphics[width=\textwidth]{fig/tabelle-dbscan-3}
\caption{DBSCAN-Ergebnisse}
\end{table}

\section{DBSCAN+LoOP-Ergebnisse}
\label{chap:a-ob-dbscan-loop}

\begin{table}[H]
	\centering
	\includegraphics[width=\textwidth]{fig/tabelle-dbscan-loop-1}
\end{table}

\begin{table}[H]
\centering
\includegraphics[width=\textwidth]{fig/tabelle-dbscan-loop-4}
\caption{DBSCAN+LoOP-Ergebnisse}
\end{table}

\chapter{SCAN-Ergebnisse}
\label{chap:a-ob-scan}

\begin{table}[H]
	\centering
	\includegraphics[width=\textwidth]{fig/tabelle-scan-1}
\end{table}

\begin{table}[H]
\centering
\includegraphics[width=\textwidth]{fig/tabelle-scan-2}
\end{table}

\begin{table}[H]
\centering
\includegraphics[width=\textwidth]{fig/tabelle-scan-3}
\end{table}

\begin{table}[H]
\centering
\includegraphics[width=\textwidth]{fig/tabelle-scan-4}
\end{table}

\begin{table}[H]
\centering
\includegraphics[width=\textwidth]{fig/tabelle-scan-5}
\end{table}

\begin{table}[H]
\centering
\includegraphics[width=\textwidth]{fig/tabelle-scan-6}
\end{table}

\begin{table}[H]
\centering
\includegraphics[width=\textwidth]{fig/tabelle-scan-7}
\caption{SCAN-Ergebnisse}
\end{table}


\chapter{Random-Walk}
\label{chap:a-rw}
Die nachfolgend aufgelisteten Graphen zeigen die Ergebnisse der Ausrei�er Erkennung mithilfe des Random-Walk Algorithmus. Die Graphen bestehen dabei aus unterschiedlichen Zeitreihen, den identifizierten Ausrei�ern sowie dem Ausrei�er-Score. In den Kapiteln \autoref{chap:a-rw-3D} und \autoref{chap:a-rw-5D} wurden die Dimensionen 2-3 bzw. 2-5 nicht dargestellt, um die �bersichtlichkeit zu gew�hrleisten.

\section{Zweidimensionales Signal}
\label{chap:a-rw-2D}
\begin{figure}[!htbp]
	\centering
	\begin{minipage}[t]{0.45\textwidth}
		\includegraphics[width=\textwidth]{fig/rw-2d-1}
		\caption{Anomalie: Einzelne Peaks}
	\end{minipage}
	\begin{minipage}[t]{0.45\textwidth}
		\includegraphics[width=\textwidth]{fig/rw-2d-2}
		\caption{Anomalie: Zunahme an Rauschen}
	\end{minipage}
\end{figure}

\begin{figure}[!htbp]
	\centering
	\begin{minipage}[t]{0.45\textwidth}
		\includegraphics[width=\textwidth]{fig/rw-2d-3}
		\caption{Anomalie: Signal Drift}
	\end{minipage}
	\begin{minipage}[t]{0.45\textwidth}
		\includegraphics[width=\textwidth]{fig/rw-2d-4}
		\caption{Anomalie: Zunahme der Amplitude}
	\end{minipage}
\end{figure}

\begin{figure}[!htbp]
	\centering
	\begin{minipage}[t]{0.49\textwidth}
		\includegraphics[width=\textwidth]{fig/rw-2d-5}
		\caption{Anomalie: Zyklus mit h�herer Amplitude}
	\end{minipage}
	\begin{minipage}[t]{0.49\textwidth}
		\includegraphics[width=\textwidth]{fig/rw-2d-6}
		\caption{Anomalie: Zyklus mit geringerer Amplitude}
	\end{minipage}
\end{figure}

\begin{figure}[!htbp]
	\centering
	\begin{minipage}[t]{0.49\textwidth}
		\includegraphics[width=\textwidth]{fig/rw-2d-7}
		\caption{Anomalie: Zyklus Aussetzer}
	\end{minipage}
	\begin{minipage}[t]{0.49\textwidth}
		\includegraphics[width=\textwidth]{fig/rw-2d-8}
		\caption{Anomalie: Signal Aussetzer}
	\end{minipage}
\end{figure}

\begin{figure}
	\centering
	\begin{minipage}[t]{0.49\textwidth}
		\includegraphics[width=\textwidth]{fig/rw-2d-9}
		\caption{Anomalie: Frequenz�nderung der Zyklen}
	\end{minipage}
\end{figure}

\section{Dreidimensionales Signal}
\label{chap:a-rw-3D}

\begin{figure}[!htbp]
	\centering
	\begin{minipage}[t]{0.45\textwidth}
		\includegraphics[width=\textwidth]{fig/rw-3d-1}
		\caption{Anomalie: Einzelne Peaks}
	\end{minipage}
	\begin{minipage}[t]{0.45\textwidth}
		\includegraphics[width=\textwidth]{fig/rw-3d-2}
		\caption{Anomalie: Zunahme an Rauschen}
	\end{minipage}
\end{figure}

\begin{figure}[!htbp]
	\centering
	\begin{minipage}[t]{0.45\textwidth}
		\includegraphics[width=\textwidth]{fig/rw-3d-3}
		\caption{Anomalie: Signal Drift}
	\end{minipage}
	\begin{minipage}[t]{0.45\textwidth}
		\includegraphics[width=\textwidth]{fig/rw-3d-4}
		\caption{Anomalie: Zunahme der Amplitude}
	\end{minipage}
\end{figure}

\begin{figure}[!htbp]
	\centering
	\begin{minipage}[t]{0.45\textwidth}
		\includegraphics[width=\textwidth]{fig/rw-3d-5}
		\caption{Anomalie: Zyklus mit h�herer Amplitude}
	\end{minipage}
	\begin{minipage}[t]{0.45\textwidth}
		\includegraphics[width=\textwidth]{fig/rw-3d-6}
		\caption{Anomalie: Zyklus mit geringerer Amplitude}
	\end{minipage}
\end{figure}

\begin{figure}[!htbp]
	\centering
	\begin{minipage}[t]{0.45\textwidth}
		\includegraphics[width=\textwidth]{fig/rw-3d-7}
		\caption{Anomalie: Zyklus-Aussetzer}
	\end{minipage}
	\begin{minipage}[t]{0.45\textwidth}
		\includegraphics[width=\textwidth]{fig/rw-3d-8}
		\caption{Anomalie: Signal-Aussetzer}
	\end{minipage}
\end{figure}

\begin{figure}
	\centering
	\begin{minipage}[t]{0.45\textwidth}
		\includegraphics[width=\textwidth]{fig/rw-3d-9}
		\caption{Anomalie: Frequenz�nderung der Zyklen}
	\end{minipage}
\end{figure}
\newpage

\section{F�nfdimensionales Signal}
\label{chap:a-rw-5D}

\begin{figure}[!htbp]
	\centering
	\begin{minipage}[t]{0.45\textwidth}
		\includegraphics[width=\textwidth]{fig/rw-5d-1}
		\caption{Anomalie: Einzelne Peaks}
	\end{minipage}
	\begin{minipage}[t]{0.45\textwidth}
		\includegraphics[width=\textwidth]{fig/rw-5d-2}
		\caption{Anomalie: Zunahme an Rauschen}
	\end{minipage}
\end{figure}

\begin{figure}[!htbp]
	\centering
	\begin{minipage}[t]{0.45\textwidth}
		\includegraphics[width=\textwidth]{fig/rw-5d-3}
		\caption{Anomalie: Signal Drift}
	\end{minipage}
	\begin{minipage}[t]{0.45\textwidth}
		\includegraphics[width=\textwidth]{fig/rw-5d-4}
		\caption{Anomalie: Zunahme der Amplitude}
	\end{minipage}
\end{figure}

\begin{figure}[!htbp]
	\centering
	\begin{minipage}[t]{0.45\textwidth}
		\includegraphics[width=\textwidth]{fig/rw-5d-5}
		\caption{Anomalie: Zyklus mit h�herer Amplitude}
	\end{minipage}
	\begin{minipage}[t]{0.45\textwidth}
		\includegraphics[width=\textwidth]{fig/rw-5d-6}
		\caption{Anomalie: Zyklus mit geringerer Amplitude}
	\end{minipage}
\end{figure}

\begin{figure}[!htbp]
	\centering
	\begin{minipage}[t]{0.45\textwidth}
		\includegraphics[width=\textwidth]{fig/rw-5d-7}
		\caption{Anomalie: Zyklus-Aussetzer}
	\end{minipage}
	\begin{minipage}[t]{0.45\textwidth}
		\includegraphics[width=\textwidth]{fig/rw-5d-8}
		\caption{Anomalie: Signal-Aussetzer}
	\end{minipage}
\end{figure}

\begin{figure}[H]
		\centering
		\includegraphics[width=8cm]{fig/rw-5d-9}
		\caption{Anomalie: Frequenz�nderung der Zyklen}
\end{figure}