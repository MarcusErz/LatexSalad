\chapter*{Kurzfassung}

Die Ausrei�ererkennung ist eine Problematik, deren Wichtigkeit in den letzten Jahren rasant zugenommen hat. Gerade die Nutzung der Datenrepr�sentation, als Graphen oder Netzwerken, zur Ermittlung von Ausrei�ern ist rapide gestiegen. Grund hierf�r sind die Vorteile, die die Abbildung von Daten in Graphen, sowie die einfachere Erkennung von Korrelationen zwischen Datenobjekten hat. Wird der Faktor Zeit hinzugezogen, so hat man die M�glichkeit, zum einen die Korrelationen besser zu erfassen und zum anderen, die strukturelle �nderung der Graphen �ber die Zeit zu entdecken. In einem weiteren Schritt k�nnen Zeitreihendaten als Graphen repr�sentiert und deren Ausrei�er effektiv identifiziert werden. \citep[vgl.][S.~1]{dyanom}


Einerseits wird in dieser Fortsetzung des Forschungsprojekts ein dynamischer Algorithmus, NetSimile, dem einen struktur-basierten Ansatz zugrunde liegt, als Pendant zum statischen OddBall-Algorithmus, evaluiert und angewendet. Andererseits wird der dynamische MIDAS-Algorithmus, der Ausrei�er in Abh�ngigkeit von Clusterbildungen detektiert, als Pendant f�r den statischen SCAN-Algorithmus n�her betrachtet. Weiterhin wird der Percolation-basierte Algorithmus als weiterer kanten-basierter Ansatz zur Erkennung von Ausrei�ern, sowie der IsoMap-basierte Algorithmus als Teil des Forschungsprojekts angewendet um die Ergebnisse verschiedenster Ans�tze miteinander zu vergleichen. \citep{Uzun}

F�r jeden dieser Algorithmentypen werden dieselben Netzwerkdaten sowie Zeitreihendaten genutzt, damit ein stringenter Vergleich der Ergebnisse erm�glicht wird. Dar�ber hinaus wird f�r den NetSimile-Algorithmus, der ein graphen-basierter Algorithmus zur Ausrei�ererkennung in dynamischen Graphen ist, Optimierungsm�glichkeiten evaluiert und erste Optimierungen vorgenommen, damit dieser alle Anforderungen eines erfolgreichen Einsatzes im Gebiet der Ausrei�ererkennung in Zeitreihen mittels graphen-basierter Algorithmen erf�llt. Dieser ist als Teil des Forschungsergebnisses sehr gut in der Erkennung von s�mtlichen Ausrei�er-Arten in Zeitreihen und zudem �beraus performant.


\textbf{Schlagw�rter:} Anomalie-Erkennung, Ausrei�ererkennung, NetSimile, MIDAS, dynamische Graphen, Percolation-basierte Ausrei�ererkennung, IsoMap-basierte Ausrei�ererkennung, graphen-basierte Algorithmen, Zeitreihen, Zeitreihentransformation, graphen-basierte Datenrepr�sentation, evolvierende Graphen
