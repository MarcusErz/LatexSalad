\chapter{Einleitung}
\label{chap:einl}

Im Rahmen der \workTyp werden verschiedene Algorithmen zur Ausrei�er-Erkennung in Graphen erforscht und getestet. Nachfolgend soll die Motivation hinter dieser Thematik erl�utert werden.

\workTodo{Den dynamischen Aspekt nicht vergessen}


\section{Hintergrund}
\label{sec:einl-hg}

\workTodo{formulieren}

\newpage
\section{Problemstellung}
\label{sec:einl-ps}

\workTodo{Ziele definieren}
Das Ziel dieser \workTyp ist es verschiedene Algorithmen anzuwenden und erste Erkenntnisse aus ihnen zu gewinnen.
Dieses Hauptziel,  im Zuge des ersten Semesters des Forschungsprojekts, kann wie folgt in drei Teilziele unterteilt werden:
\begin{enumerate}
	\item Verschaffen eines �berblicks �ber die existierenden Algorithmen zur Erkennung von Ausrei�ern in Graphen
	\item Die Entwicklung eines Ausrei�er-Scores f�r die zugrundeliegenden Algorithmen
	\item Erste Anwendung der verwendeten Graphen-basierten Algorithmen auf Zeitreihen 
\end{enumerate}

\section{Verwandte Arbeiten}
\label{sec:einl-va}

\workTodo{related work einf�gen}

Die \textbf{Anomalieerkennung in Edge Streams} verwendet als Eingabe einen Fluss von Kanten �ber die Zeit. Sie werden nach der Art der erkannten Anomalie kategorisiert:
\begin{description}
	\item[Erkennung anomaler Knoten:] Mithilfe eines Edge Streams erkennt (Yu et al. 2013) Knoten, deren Egonetze sich pl�tzlich und signifikant �ndern.
	\item[Erkennung anomaler Subgraphen]: Mithilfe eines Edge Streams identifiziert (Shin et al. 2017) dichte Teilgraphen, die innerhalb einer kurzen Zeit entstehen.
	\item[Anomale Kantenerkennung:] (Ranshous et al. 2016) konzentriert sich auf sp�rlich verbundene Teile eines Graphen, w�hrend (Eswaran und Faloutsos 2018) Kantenanomalien basierend auf dem Auftreten von Kanten, bevorzugter Anhaftung und gegenseitigen Nachbarn identifiziert.
\end{description}

Die \textbf{Ausrei�er Erkennung in Zeitreihen und Sequenziellen Daten} wurde bereits in vielen Literaturquellen diskutiert. 
\begin{description}
	\item[Netzwerk basierter Ansatz zur Erkennung von Ausrei�ern in Sequenziellen Daten\citep{RAHMANI201489}:] Der genannte Algorithmus wandelt Sequenziellen Daten in ein Netzwerk um. Dabei wird die Euklidischen Distanz genutzt um die Kantengewichte zu berechnen. Anschlie�end werden die Knoten mithilfe des Minimum Spanning Tree Algorithmus geclustered. Um hierraus Ausrei�er zu abzuleiten wird ein Voting Scheme verwendet. Der vorgestellte Algorithmus wurde genutzt um Ausrei�er in Wetter Daten sowie Aktienkursen zu identifizieren.
	
	\item[Ein robuster graphbasierter Algorithmus zur Erkennung und Charakterisierung von Anomalien in verrauschten multivariaten Zeitreihen:] In diesem Paper \citep{4733955} wird ein Algorithmus vorgestellt, der dazu in der Lage ist Ausrei�er in Multivariaten Zeitreihen zu erkennen. Die multivariate Zeitreihe wird dabei �ber ein Distanzma� in ein Netzwerk umgewandelt. Auf dem Netzwerk wird anschlie�end ein Random Walk Algorithmus ausgef�hrt. Daraufhin werden Knoten die besonders selten besucht wurden als Ausrei�er markiert.
	
	\item[�berblicksartikel �ber die Ausrei�er Erkennung in Diskreten Sequenzen]: In \citep{5645624} werden verschiedene Methoden vorgestellt, wie Ausrei�er in Sequenzen erkannt werden k�nnen. Es wird dabei, auch auf die Ausrei�er Erkennung in Zeitreihen eingegangen. Die vorgestellten Algorithmen werden in drei Kategorien untergliedert. 1:Erkennung abnormaler Sequenzen in Bezug auf eine Datenbank normaler Sequenzen 2: Erkennung einer abnormalen Untersequenz innerhalb einer langen Sequenz. 3: Erkennung eines Musters in einer Sequenz deren Auftrittsh�ufigkeit anomal ist.
	
	\item[Neuronale Netze zur Ausrei�er Erkennung]: Die Verwendung von Neuronalen Netzen zur Erkennung von Ausrei�ern wird immer beliebter. Beispielsweise wurde in \citep{10.1007/3-540-46145-0_17} ein Replicator Neuronales Netz, einerseits genutzt um St�rungen in einem Netzwerk zu erkennen. Des weiteren wurde das Neuronale Netz verwendet um Ausrei�er in einem Brustkrebs Datensatz zu identifizieren. Neuronale Netze wurden ebenso dazu eingesetzt um Ausrei�er in Zeitreihen zu finden \citep{8581424} . Ein Vorteil dieses Ansatzes ist, das Ausrei�er online entdeckt werden k�nnen. Das Neuronale Netz wird hierbei dazu genutzt den n�chsten Wert einer Zeitreihe zu sch�tzen. Die Differenz zwischen der Vorhersage und dem tats�chlich auftretenden Wert wird als Ausrei�er Score verwendet. 
\end{description}
