\chapter{Einleitung}
\label{chap:einl}

Im Rahmen der \workTyp werden verschiedene Algorithmen zur Ausrei�er-Erkennung in Graphen erforscht und getestet. Nachfolgend soll die Motivation hinter dieser Thematik erl�utert werden.

\workTodo{Den dynamischen Aspekt nicht vergessen}


\section{Hintergrund}
\label{sec:einl-hg}

\workTodo{formulieren}

\newpage
\section{Problemstellung}
\label{sec:einl-ps}

\workTodo{Ziele definieren}
Das Ziel dieser \workTyp ist es verschiedene Algorithmen anzuwenden und erste Erkenntnisse aus ihnen zu gewinnen.
Dieses Hauptziel,  im Zuge des ersten Semesters des Forschungsprojekts, kann wie folgt in drei Teilziele unterteilt werden:
\begin{enumerate}
	\item Verschaffen eines �berblicks �ber die existierenden Algorithmen zur Erkennung von Ausrei�ern in Graphen
	\item Die Entwicklung eines Ausrei�er-Scores f�r die zugrundeliegenden Algorithmen
	\item Erste Anwendung der verwendeten Graphen-basierten Algorithmen auf Zeitreihen 
\end{enumerate}

\section{Verwandte Arbeiten}
\label{sec:einl-va}

\workTodo{related work einf�gen}

Die \textbf{Anomalieerkennung in Edge Streams} verwendet als Eingabe einen Fluss von Kanten �ber die Zeit. Sie werden nach der Art der erkannten Anomalie kategorisiert:
\begin{description}
	\item[Erkennung anomaler Knoten:] Mithilfe eines Edge Streams erkennt (Yu et al. 2013) Knoten, deren Egonetze sich pl�tzlich und signifikant �ndern.
	\item[Erkennung anomaler Subgraphen]: Mithilfe eines Edge Streams identifiziert (Shin et al. 2017) dichte Teilgraphen, die innerhalb einer kurzen Zeit entstehen.
	\item[Anomale Kantenerkennung:] (Ranshous et al. 2016) konzentriert sich auf sp�rlich verbundene Teile eines Graphen, w�hrend (Eswaran und Faloutsos 2018) Kantenanomalien basierend auf dem Auftreten von Kanten, bevorzugter Anhaftung und gegenseitigen Nachbarn identifiziert.
\end{description}
