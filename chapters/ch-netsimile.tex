\newpage
\chapter{Netsimile}
\label{chap:ns}
\workTodo{In diesem Kapitel werden grundlegende Themen behandelt, die im Rahmen des Forschungsprojekts zum Verst�ndnis der Ausrei�er-Erkennung in Graphen gedient haben.}

\section{Grundlagen}
\label{sec:ns-gl}
\workTodo{Einf�hrung in den Algorithmus}


\subsection{Canberra Distance}
\label{sec:ns-gl-cd}

\subsubsection{Einf�hrung}
\label{sec:ns-gl-cd-int}

\workTodo{Stichworte sammeln}

\section{Ergebnisse Zeitreihe}
\label{sec:resultsTS}
\begin{tabular}{ |p{4.5cm}||p{4.5cm}||p{3cm}||p{3cm}|}
	\hline
	\textbf{Ausrei�er Typ}& \textbf{Datei Name}&
	\textbf{1D}&\textbf{2D}\\
	\hline
	\hline
	Einzelne Peaks & anomaly-art-daily-peaks & *& *\\
	\hline
	Zunahme an Rauschen & anomaly-art-daily-increase-noise &****& *\\
	\hline
	Signal Drift & anomaly-art-daily-drift &***& *\\
	\hline
	Kontinuierliche Zunahme der Amplitude& art-daily-amp-rise & ***& *\\
	\hline
	Zyklus mit h�herer Amplitude & art-daily-jumpsup &****& *\\
	\hline
	Zyklus mit geringerer Amplitude & art-daily-jumpsdown & ****& *\\
	\hline
	Zyklus-Aussetzer & art-daily-flatmiddle &****& *\\
	\hline
	Signal-Aussetzer & art-daily-nojump & -& *\\
	\hline
	Frequenz�nderung & anomaly-art-daily-sequence-change &-& *\\
	\hline
\end{tabular}

