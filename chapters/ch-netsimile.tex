\newpage
\chapter{Netsimile}
\label{chap:ns}
\workTodo{In diesem Kapitel werden grundlegende Themen behandelt, die im Rahmen des Forschungsprojekts zum Verst�ndnis der Ausrei�er-Erkennung in Graphen gedient haben.}

\section{Grundlagen}
\label{sec:ns-gl}
\workTodo{Einf�hrung in den Algorithmus}


\subsection{Canberra Distance}
\label{sec:ns-gl-cd}

\subsubsection{Einf�hrung}
\label{sec:ns-gl-cd-int}

\workTodo{Stichworte sammeln}

\section{Ergebnisse Zeitreihe}
\label{sec:resultsTS}
\workTodo{Bin mir nicht sicher ob ich meine Implementierung vom Netsimile genauer erkl�ren soll}

\workTodo{Au�erdem funktioniert das mit den einzelnen Peaks irgenwie nicht. Das muss ich noch verbessern}

\workTodo{Noch Parameter des ganzen erl�utern}

Um die Laufzeit des Netsimile Algorithmus zu optimieren wurde der Algorithmus zus�tzlich noch ohne Verwendung eines Graphen Frameworks implementiert. Dadurch konnte die Laufzeit sehr stark auf wenige Sekunden reduziert werden. F�r die Tests wurden ein und zweidimensionale Zeitreihen der Numenta Gruppe verwendet.\workTodo{Vielleicht Quelle noch rein machen}. Diese Zeitreihen enthalten verschiedene Ausrei�er Typen, auf deren Erkennung der Algorithmus getestet wurde. Die Qualit�t der Ausrei�ererkennung wurde mithilfe eines Sternesystems bewertet. Das System geht von - Ausrei�er nicht erkannt bis **** Ausrei�er sehr gut erkannt.  

\begin{table}
\caption{Netsimile Time Series Perfomance}
\begin{tabular}{ |p{4.5cm}||p{4.5cm}||p{3cm}||p{3cm}|}
	\hline
	\textbf{Ausrei�er Typ}& \textbf{Datei Name}&
	\textbf{1D}&\textbf{2D}\\
	\hline
	\hline
	Einzelne Peaks & anomaly-art-daily-peaks & *& *\\
	\hline
	Zunahme an Rauschen & anomaly-art-daily-increase-noise &****& ***\\
	\hline
	Signal Drift & anomaly-art-daily-drift &***& -\\
	\hline
	Kontinuierliche Zunahme der Amplitude& art-daily-amp-rise & ***& ***\\
	\hline
	Zyklus mit h�herer Amplitude & art-daily-jumpsup &****& *\\
	\hline
	Zyklus mit geringerer Amplitude & art-daily-jumpsdown & ****& -\\
	\hline
	Zyklus-Aussetzer & art-daily-flatmiddle &****& ***\\
	\hline
	Signal-Aussetzer & art-daily-nojump & ****& ***\\
	\hline
	Frequenz�nderung & anomaly-art-daily-sequence-change &****& ***\\
	\hline
\end{tabular}
\end{table}

Die Qualit�t der Ausrei�ererkennung des Algorithmus im eindimensionalen Fall ist sehr gut. Lediglich bei den einzelnen Peaks wird der Au�rei�er nicht erkannt. Des weiteren wird beim Signal Drift und der kontinuirlichen Zunahme der Amplitude lediglich der Start des Ausrei�ers erkannt und nicht der komplette Ausrei�er. Aus diesem Grund wurden die Ausrei�er F�lle lediglich mit drei Sternen bewertet. Des weiteren wird h�ufig das sechste oder sibte Element der Zeitreihe als Ausrei�er markiert. Der Grund daf�r ist, das Standaradabeweichung bei Fenstergr��e von f�nf am Anfang null ist. Dieser Umstand wurde bei der bewertung nicht ber�cksichtigt und spielt bei mehreren Abschnitten auch keine Rolle mehr. Im zweidimensionalen Fall ist die Qualit�t der Ausrei�ererkennung etwas durchwachsener. Besonders auffallend ist, dass Zyklen mit h�herer und niedriger Amplitude nicht erkannt werden, da diese von vielen Algorithmen sehr gut erkannt werden.  Des weiteren wird der Signal drift nicht mehr erkannt. Die weiteren Algorithmen erkennen die Ausrei�er weiterhin, teilwei�e aber nicht komplett und nur den Anfang. 

