% % Preamble BEGINN %%%%%%%%%%%%%%%%%%%%%%%%%%%%%%%%%%%%%%%%%%%%%%%%%%%%%%%%%

%%% Preamble (Dokumentenklasse)
\input{preamble/pre-class}

%%% Alle Namen usw. im Titel und im hyperref-Paket
% ------------------------------------------------------------------------
% LaTeX - Preambel ******************************************************
% ------------------------------------------------------------------------
% pre-work
% ========================================================================
% % ToDo kennzeichnen
\newcommand{\workTodo}[1]{\textcolor{red}{todo: #1}}

% % F�r Datum und Zeit in Fusszeile
% % !!!Inhalt bei Fertigstellung der Arbeit l�schen
\newcommand{\workMarkDateTime}{\workTodo{\today{} - \thistime\ Uhr}}

% % Alle Namen werden im Titel und im hyperref-Paket eingetragen
% % !!! Ueberall f�r <Wert> das Entsprechende eintragen

% <Typ> Studienarbeit, Dipolmarbeit, Studienarbeit oder Bachlor-Abschlussarbeit
\newcommand{\workTyp}{Forschungsprojekt\xspace}

% <Titel> der Arbeit
\newcommand{\workTitel}{Ausrei�ererkennung in Zeitreihen \\mittels graphen-basierter Algorithmen}

% <Studiengang> z.B. Kommunikationstechnik
\newcommand{\workStudiengang}{Angewandte Informatik\xspace}

% <Semester> mit Jahr z.B. Sommersemester 2008  
\newcommand{\workSemester}{Wintersemester 2020/2021\xspace}

% <Name> des Studenten
\newcommand{\workNameStudent}{Bahar Uzun\xspace}
%\newcommand{\workStudentNo}{123456\xpsace}
\newcommand{\workNameStudentt}{Jeremy Kielman\xspace}
%\newcommand{\workStudentNoo}{123456\xpsace}
\newcommand{\workNameStudenttt}{Marcus Erz\xspace}
%\newcommand{\workStudentNooo}{123456\xpsace}


% <Pruefer> Name des pr�fenden (betreuenden) Professor an der Hochschule
\newcommand{\workPruefer}{Prof. Dr. rer. nat. Gabriele G�hring\xspace} 

\newcommand{\workDeadline}{18. M�rz 2021}

% %%% Nur bei Abschluss-Arbeiten

% <Datum> der Abgabe der Arbeit (Eidesstatliche Erkl�rung)
\newcommand{\workDatum}{\today\xspace}

% <Zweitpr�fer>
%\newcommand{\workZweitPruefer}{Prof. Dr. Rainer Marchthaler\xspace}

% <Zeitraum>
%\newcommand{\workZeitraum}{15. M�rz 2020 - 31. August 2020\xspace}







%%% Preamble (Pakete)
\input{preamble/pre-packages}

%%% Neue Befehle
\input{preamble/pre-newcommands}
\input{preamble/pre-tablecommands} % Fuer Tabellen
%%% Silbentrennung
\input{preamble/pre-hyphenation}

% % Nur diese Kapitel (Dateien) einbinden
%\includeonly{
%chapters/ch-aa-titel,
%chapters/ch-aa-vorspiel,
%chapters/ch-einleitung,
%chapters/ch-hauptteil,
%chapters/ch-schluss,
%chapters/ch-zz-anhang
%}
% % Preamble ENDE %%%%%%%%%%%%%%%%%%%%%%%%%%%%%%%%%%%%%%%%%%%%%%%%%%%%%%%%%%

% % Inhalt BEGINN %%%%%%%%%%%%%%%%%%%%%%%%%%%%%%%%%%%%%%%%%%%%%%%%%%%%%%%%%
\newcount\mycount
\begin{document}
% Tabellen-Einstellungen
\input{preamble/pre-tablesettings}
% % %%%%%% Vorspiel
\begin{spacing}{1} % Vorspiel immer mit Standard-Zeilenabstand setzen
	\frontmatter
	% % Titelblatt
	% % Neue Befehle
\newcommand{\HRule}[2]{\noindent\rule[#1]{\linewidth}{#2}} % Horiz. Linie
\newcommand{\vlinespace}[1]{\vspace*{#1\baselineskip}} % Abstand
\newcommand{\titleemph}[1]{\textbf{#1}} % Hervorheben

\begin{titlepage}
 \sffamily % Titelseite in seriefenloser Schrift
      % Logo Hochschule Esslingen
      \begin{minipage}{0.49\textwidth}
      	\includegraphics[width=8cm]{fig/aa-titel/HE_Logo_4c}
      \end{minipage} 
      \begin{minipage}{0.49\textwidth}
%      \hfill \workFirmenLogo
      \end{minipage}
      \HRule{13pt}{1pt} 
   \centering
      \Large
      \vlinespace{3}\\
      \workTyp\\
      \huge
      \workTitel\\
%
      \Large
      \vlinespace{2}
          im Studiengang \workStudiengang\\
          der Fakult�t Informationstechnik\\
%      
      \workSemester\\
%     
      \vlinespace{2}
      \workNameStudent \\
      764647	\\
      \workNameStudentt	\\
      764097	\\
      \workNameStudenttt \\
      762294 \\

   \vfill
   \raggedright
%   
   \large
   \titleemph{Abgabedatum:} \workDeadline \\ % Nur bei Abschluss-Arbeiten
%   \titleemph{Datum:} \workDatum \\ % Nur bei Studien-Arbeiten
   \titleemph{Pr�ferin:} \workPruefer \\
%   \titleemph{Zweitpr�fer:} \workZweitPruefer \\ % Nur bei Abschluss-Arbeiten

 % Folgenden Abschnitt nur bei Industrie-Arbeiten darstellen
   \vlinespace{1}
   \HRule{13pt}{1pt} \\
%   	\titleemph{Firma:} \workFirma \\
 %  	\titleemph{Betreuer:} \workBetreuer 
%
\end{titlepage}

%	% %%%%%%%%%%%%%%%%%%%%%%%%%%%%%%%%%%%%%%%%%%%%%%%%%%%%%%%%%%%%%%%%%%%%%%%%%%
\chapter*{Ehrenw�rtliche Erkl�rung}

Hiermit versichere ich, die vorliegende Arbeit selbstst�ndig und unter ausschlie�licher Verwendung der angegebenen Literatur und Hilfsmittel erstellt zu haben.\\\\
Die Arbeit wurde bisher in gleicher oder �hnlicher Form keiner anderen Pr�fungsbeh�rde vorgelegt und auch nicht ver�ffentlicht.\\
\begin{tabbing}
          Esslingen, den \workDatum ~~	\= \rule{5cm}{0.3mm}\\
                                                                                                    \> Unterschrift
\end{tabbing}
%
\newpage
% %%%%%%%%%%%%%%%%%%%%%%%%%%%%%%%%%%%%%%%%%%%%%%%%%%%%%%%%%%%%%%%%%%%%%%%%%%
%
\chapter*{Sperrvermerk} % Optional; Hinweis auf Vertraulichkeit dieser Arbeit

Die nachfolgende \workTyp enth�lt vertrauliche Daten der \workFirma.
Ver�ffentlichungen oder Vervielf�ltigungen dieser Arbeit -- auch nur auszugsweise -- sind ohne ausdr�ckliche Genehmigung der \workFirma nicht gestattet.
Diese Arbeit ist nur den Pr�fern sowie den Mitgliedern des Pr�fungsausschusses zug�nglich zu machen.
\newpage
% %%%%%%%%%%%%%%%%%%%%%%%%%%%%%%%%%%%%%%%%%%%%%%%%%%%%%%%%%%%%%%%%%%%%%%%%%%
%
\chapter*{Zitat} % Optional
\begin{center}
\begin{minipage}{12cm}
\begin{quotation}
\textit{\enquote{Gegen Leistungen kommt man nur mit Leistungen auf.}}
\end{quotation}
\hfill \textsf Robert Bosch
\end{minipage}
\end{center}
\newpage{}
% %%%%%%%%%%%%%%%%%%%%%%%%%%%%%%%%%%%%%%%%%%%%%%%%%%%%%%%%%%%%%%%%%%%%%%%%%%
\chapter*{Danksagung} % oder Danksagung; Optional

An dieser Stelle m�chte ich mich bei all jenen bedanken, die mich w�hrend meiner Bachelorarbeit unterst�tzt und motiviert haben.

Ganz besonders geb�hrt mein Dank Herrn Prof. Dr. Ing. Hermann Kull f�r die gro�artige Unterst�tzung und die konstruktive Kritik bei der Erstellung dieser Bachelor-Abschlussarbeit.

Au�erdem m�chte ich mich bei der Robert Bosch GmbH f�r die Erm�glichung dieser Bachelor-Abschlussarbeit in dieser besonders schweren Situation bedanken. Ebenfalls gilt mein Dank an dieser Stelle meinem Betreuer Thomas Weihrich, meinem Vorgesetzten Norbert Lang und meinen Kollegen der Gruppe CC-DA/ECR6.
\newpage
% %%%%%%%%%%%%%%%%%%%%%%%%%%%%%%%%%%%%%%%%%%%%%%%%%%%%%%%%%%%%%%%%%%%%%%%%%%

	\chapter*{Kurzfassung}
\workTodo{Kurzfassung erstellen}

\textbf{Schlagw�rter:} Anomalie-Erkennung, Ausrei�er-Erkennung, NetSimile, MIDAS, Perculation, Iso-Map Graphen-basierte Algorithmen, Zeitreihen
	% % Verzeichnisse
	\tableofcontents
%	\chapter*{Abk\"urzungsverzeichnis}
\addcontentsline{toc}{chapter}{Abk\"urzungsverzeichnis}
\begin{acronym}	
	\acro{STIL}{Self Test Interpreter and Loader}
\end{acronym} %Abk. Verzeichnis
	\listoffigures
	\addcontentsline{toc}{chapter}{Abbildungsverzeichnis}
	\listoftables
	\addcontentsline{toc}{chapter}{Tabellenverzeichnis}
	\lstlistoflistings % Verzeichnis f�r Code-Listing
	\addcontentsline{toc}{chapter}{Listings}
\end{spacing}{1}
% % %%%%%% Textteil (Eigentliche Arbeit)
\mainmatter
%
\chapter{Einleitung}
\label{chap:einl}

Im Rahmen des Forschungsprojektes wird erforscht, wie Zeitreihendaten in Graphen umgewandelt werden k�nnen und welche Algorithmen diese Graphen am besten auf Ausrei�er untersuchen k�nnen. Es erfolgt eine kurze Einf�hrung in die Problemstellung. Ebenso werden verwandte Arbeiten vorgestellt und beschrieben.
\section{Problemstellung}
\label{sec:einl-ps}
Die Transformation von Zeitreihendaten in Graphen erm�glicht die Generierung von Korrelationen zwischen zwei Zeitelementen. Durch diese Datenrepr�sentation k�nnen die Datenobjekte, die in Abh�ngigkeit zueinander stehen, untersucht werden. Es wird angenommen, dass in der Erforschung von graphen-basierten Algorithmen zur Ausrei�ererkennung, gerade in Netzwerken, Ausrei�er bzw. Anomalien effizienter erkannt werden.

In der Forschung gibt es bereits viele Algorithmen zur Ausrei�ererkennung, die auf statischen sowie dynamischen Graphen anwendbar sind. Die Erkennung der Ausrei�er erfolgt hierbei mit verschiedenen Ans�tzen. So gibt es Algorithmen, die die Struktur der Graphen n�her betrachten sowie Algorithmen, die sich auf dichte-basierte Merkmale eines Graphen fokussieren. Bei der Untersuchung von dynamische Graphen haben die strukturellen Ver�nderungen �ber die Zeit ebenso wie pl�tzlich massiv zunehmende Aktivit�ten zwischen Knoten- und Kantenpaaren eine gro�e Bedeutung. Dar�ber hinaus gibt es Ans�tze zur Ausrei�ererkennung in sequenziellen Daten.

Bisher sind in der Forschung wenige graphen-basierten Algorithmen auf traditionellen Zeitreihendaten, wie Sensordaten, die �ber die Zeit gesammelt werden, vorhanden. Vielmehr liegt der Fokus auf sich �ber die Zeit �ndernden Netzwerkstrukturen.

Im Rahmen des Forschungsprojekts werden folglich die graphen-basierten Algorithmen zur Erkennung von Ausrei�ern in Zeitreihendaten herangezogen um erste Erkenntnisse �ber die Aussagef�higkeit der Ergebnisse treffen zu k�nnen. Bei einem erfolgreichen Einsatz der Algorithmen k�nnen die Anwendungsf�lle auf die Bereiche Internet of Things, Autonomes Fahren sowie die Erkennung von Krankheiten wie Krebs, erweitert werden. Dies macht das Thema der Ausrei�ererkennung mittels graphen-basierter Algorithmen zu einem aktuellen und wichtigen Forschungsgebiet, in dem die Vorteile einer graphen-basierten Struktur auf die Zeitreihen �bertragen werden. 

Das Ziel der vorliegenden Arbeit setzt sich aus den folgenden Teilzielen zusammen:

\begin{enumerate}
	\item Die Ermittlung einer Methode zur Transformation einer Zeitreihe in einen Graphen.
	\item Die empirische Anwendung der graphen-basierten Algorithmen auf Zeitreihendaten und deren Analyse hinsichtlich der Erkennung von Ausrei�ern.
\end{enumerate}


\newpage
\section{Verwandte Arbeiten}
\label{sec:einl-va}

In diesem Abschnitt werden Ans�tze zur Erkennung von Ausrei�ern in statischen und dynamischen Graphen vorgestellt, die im Rahmen des Forschungsprojekts zur Erreichung der Forschungsziele herangezogen werden.  


Die \textbf{Ausrei�ererkennung in statischen Graphen} kann nach den unterschiedlichen Ausrei�erkategorien unterteilt werden. So ergibt sich die nachfolgende Taxonomie.
\begin{mldescription}
	\mlitem{Struktur-basierter Ansatz:} Mithilfe der Darstellung von Knoten und Kanten in einem Ego-Netzwerk werden in \citep{Oddball} die zugeh�rigen Eigenschaften, wie die Anzahl der Knoten und Kanten, extrahiert. Im Anschluss identifiziert dieser Algorithmus diejenigen Knoten und Kanten, die sich strukturell stark von den restlichen Ego-Netzwerken unterschieden. 
	\mlitem{Clustering-basierter Ansatz:} In \citep{SCAN} werden zwei Knoten und die �berschneidung ihrer Nachbarknoten gegen�bergestellt. So wird die Annahme getroffen, dass Knoten, die sehr wenige Nachbarn im Vergleich zum Rest der Knoten in Ihrer Umgebung haben, Ausrei�er sind.
	\mlitem{IsoMap-basierter Ansatz:} Durch die Dimensionsreduktion mithilfe des IsoMap-basierten Algorithmus in \citep{10.3389/fphy.2019.00194} gehen Informationen �ber Ausrei�er verloren. Bei dem Versuch der Rekonstruktion k�nnen diese Informationen nicht wiederhergestellt werden. Durch einen anschlie�enden Vergleich der extrahierten Informationen werden Ausrei�er sichtbar.
	\mlitem{Perculation-basierter Ansatz:} In \citep{10.3389/fphy.2019.00194} wird der Perculation-basierte Algorithmus beschrieben. Bei diesem werden aus einem Graphen schrittweise die Kanten mit den h�chsten Gewichten entfernt. Dadurch werden Ausrei�er vom Rest des Netzwerks separiert. Die Annahme ist, dass Ausrei�er-Knoten h�here Kantengewichte zu ihren Nachbarn haben.
	\mlitem{kanten-basierter Ansatz:}Der \citep{RandomWalk} iteriert der Algorithmus zuf�llig �ber das Netzwerk. Dabei wird festgehalten wie oft ein Knoten besucht wurde. Ausrei�er-Knoten werden dabei besonders selten besucht und k�nnen somit identifiziert werden. 
\end{mldescription}


Die \textbf{Ausrei�ererkennung in dynamischen Graphen} kann hinsichtlich ihres Inputs unterteilt werden. So ergibt sich die nachfolgende Taxonomie.

\begin{mldescription}
	\mlitem{Erkennung auf Momentaufnahmen des Graphen in zeitlichen Abst�nden:} Der Algorithmus in \citep{Netsimile} vergleicht verschiedene strukturelle Merkmale zweier Momentaufnahmen eines Graphen miteinander um die �hnlichkeit zu bewerten. Der Ausrei�er wird bei einer starken Ver�nderung des Graphen deklariert.
	\mlitem{Erkennung mithilfe eines Datenstroms:} Der Algorithmus in \citep{MIDAS} vergleicht jede ankommende Kante, zum aktuellen Zeitpunkt, mit der Anzahl am bisherigen Vorkommen dieser Kante. Hierbei werden Mikrocluster entdeckt, die anomal sind.
\end{mldescription}


Um die graphen-basierten Algorithmen zur Erkennung von Ausrei�ern in Zeitreihendaten nutzen zu k�nnen, m�ssen diese Daten in Graphen umgewandelt werden. Hierbei werden Distanzma�e verwendet, um die Qualit�t der Ausrei�ererkennung zu verbessern. Im Rahmen des Forschungsprojekt ist somit die \textbf{Nutzung von Distanzma�en} essenziell. 

\begin{mldescription}
	\mlitem{�berblick an existierenden Distanzma�en:} Ein Vergleich der verschiedenen Distanzma�e ist in \citep{DistanceMeasures} zu finden. Diese werden f�r die Anwendung auf Wahrscheinlichkeitsdichtefunktionen evaluiert und gruppiert. 
\end{mldescription}

%Die \textbf{Ausrei�er Erkennung in Zeitreihen und Sequenziellen Daten} wurde bereits in vielen Literaturquellen diskutiert. 
%\begin{mldescription}
%	
%	\mlitem{Netzwerk basierter Ansatz zur Erkennung von Ausrei�ern in Sequenziellen Daten:} Der in \citep{RAHMANI201489} genannte Algorithmus wandelt sequenziellen Daten in ein Netzwerk um. Dabei wird die euklidischen Distanz genutzt, um die Kantengewichte zu berechnen. Anschlie�end werden die Knoten mithilfe des Minimum Spanning Tree Algorithmus geclustered. Um daraus Ausrei�er abzuleiten, wird ein \textit{Voting Scheme} verwendet. Der vorgestellte Algorithmus wurde genutzt um Ausrei�er in Wetterdaten sowie Aktienkursen zu identifizieren.
%	
%	\mlitem{Ein robuster graphen-basierter Algorithmus zur Erkennung und Charakterisierung von Anomalien in verrauschten multivariaten Zeitreihen:} In \citep{RandomWalk} wird ein Algorithmus vorgestellt, der dazu in der Lage ist Ausrei�er in multivariaten Zeitreihen zu erkennen. Die multivariate Zeitreihe wird hierbei �ber ein Distanzma� in ein Netzwerk umgewandelt. Auf dem Netzwerk wird anschlie�end ein Random Walk Algorithmus ausgef�hrt. Daraufhin werden Knoten die besonders selten besucht wurden als Ausrei�er markiert.
%	
%	\mlitem{�berblicksartikel �ber die Ausrei�ererkennung in diskreten Sequenzen:} In \citep{5645624} werden verschiedene Methoden vorgestellt, wie Ausrei�er in Sequenzen erkannt werden k�nnen. Es wird dabei ebenso auf die Ausrei�er Erkennung in Zeitreihen eingegangen. Die vorgestellten Algorithmen werden in drei Kategorien untergliedert. 1: Erkennung abnormaler Sequenzen in Bezug auf eine Datenbank normaler Sequenzen 2: Erkennung einer abnormalen Untersequenz innerhalb einer langen Sequenz. 3: Erkennung eines Musters in einer Sequenz deren Auftrittsh�ufigkeit anomal ist.
%	
%	\mlitem{Neuronale Netze zur Ausrei�ererkennung:} Neuronale Netze werden immer h�ufiger zur Erkennung von Ausrei�ern verwendet. Beispielsweise wurde in \citep{10.1007/3-540-46145-0_17} ein Replicator Neuronales Netz, einerseits genutzt um St�rungen in einem Netzwerk zu erkennen. Des weiteren wurde das Neuronale Netz verwendet um Ausrei�er in einem Brustkrebs Datensatz zu identifizieren. Neuronale Netze wurden ebenso dazu eingesetzt um Ausrei�er in Zeitreihen zu finden \citep{8581424}. Ein Vorteil dieses Ansatzes ist, dass Ausrei�er online entdeckt werden k�nnen. Das Neuronale Netz wird hierbei dazu genutzt den n�chsten Wert einer Zeitreihe zu sch�tzen. Die Differenz zwischen der Vorhersage und dem tats�chlich auftretenden Wert wird als Ausrei�er Score verwendet. 
%	
%\end{mldescription}

%\include{chapters}
%\newpage
\chapter{Transformation Zeitreihe zu Netzwerk}
Ziel unseres Forschungsprojektes ist es unter anderem verschiedene Algorithmen, zur Ausrei�er Erkennung in Netzwerken, auf Zeitreihendaten anzuwenden. Als erstes m�ssen hierzu die Zeitreihen in ein Netzwerk umgewandelt werden, dieser Schritt wird in diesem Kapitel erl�utert. Je nach Ausrei�er-Erkennung Algorithmus, muss die Transformation leicht unterschiedlich durchgef�hrt werden. Aus diesem Grund wird in \autoref{chap:trsnsNeti}, \autoref{chap:trsnsMidas} und \autoref{chap:trsnsMidasR} erl�utert wie die Umwandlung f�r die jeweiligen Algorithmen funktioniert.
\section{Netsimile}
\label{chap:trsnsNeti}
Der erste Schritt der Transformation ist, die Zeitreihe in kleinere Intervalle aufzusplitten. Anschlie�end kann f�r jedes der Intervalle ein Netzwerk berechnet werden. Die L�nge des Intervalls kann als Hyperparameter an den Algorithmus �bergeben werden. Je nach Zeitreihe funktionieren unterschiedliche Intervallgr��en besser oder schlechter. Insofern die Zeitreihe eine Saisonalit�t aufweist, kann diese bestimmt und als Intervallgr��e genutzt werden.
\workTodo{�berpr�fen ob Saisonalit�t der richtige Begriff ist.}\\
Um die einzelne Zeitintervalle in ein Netzwerk umzuwandeln, wird zun�chst die Distanz zwischen den einzelnen Elementen des Zeitintervalls berechnet. Hierzu wird auf das in \citep[vgl.][S.~2-3]{10.3389/fphy.2019.00194} vorgestellte Distanzma� zur�ckgegriffen. Insofern f�r p = 2 eingesetzt wird, handelt es sich um die euklidische Distanz. Die Abst�nde bilden die Kantengewichte zwischen den jeweiligen Elementen im Netzwerk. Die Elemente der Zeitreihe bilden die Knoten des Netzwerks. Die Netzwerke werden intern als Adjazenzmatrizen gespeichert.
\begin{equation}
D_{ij}=\left(\sum_{k} \left|v_{k}^{i}-v_{k}^{j}\right|^{p}\right)^{1/p}
\end{equation}
 Im n�chsten Schritt m�ssen die Netzwerke in CSV-Dateien gespeichert werden, sodass der  Netsimile Algorithmus die Daten einlesen kann. Dazu wird f�r jede Kante des Netzwerks eine  Zeile im der Datei, mit folgendem Format generiert: Ursprungsknoten, Zielknoten, Gewichtung. F�r jedes Zeitintervall muss eine einzelne CSV-Datei angelegt werden. Der Netsimile Algorithmus vergleicht...

\begin{figure}[H]
	\centering
	\includegraphics[width=13cm]{fig/tsToNet/tsToCsv}
	\caption{Umwandlung einer Zeitreihe in Netzwerk}
	\label{img:tsToNet}
\end{figure}
\workTodo{Das Netzwerk aus der Grafik noch ab�ndern}



\section{MIDAS}
\label{chap:trsnsMidas}
Die Transformation der Zeitreihe in mehrere Netzwerke funktioniert f�r den MIDAS Algorithmus gleich wie in \autoref{chap:trsnsNeti}. Allerdings kann der Algorithmus teilweise bessere Ergebnisse erzielen, wenn f�r p eine Zahl gr��er als zwei eingesetzt wird. Dadurch werden gr��ere Abst�nde zwischen Elementen st�rker gewichtet.\\
Au�erdem erwartet der MIDAS Algorithmus f�r die Netzwerkdaten ein anderes �bergabeformat. Hierbei k�nnen alle Daten der jeweiligen Zeitabschnitte in einen CSV-File geschrieben werden. Die CSV-Datei muss dabei folgenderma�en strukturiert sein: Ursprungsknoten, Zielknoten, Zeitintervall. Es ist nicht m�glich die Kantengewichtung direkt an den Algorithmus zu �bergeben. Um die Kantengewichtung trotzdem �bergeben zu k�nnen, wird die gleiche Kante mehrmals in Abh�ngigkeit der Gewichtung an den Algorithmus �bergeben.
Wie funkt Midas ganz kurz..


\begin{figure}[H]
	\centering
	\includegraphics[width=2cm]{fig/tsToNet/midasData}
	\caption{Datensatz Midas}
	\label{img:tsToNetMiData}
\end{figure}


\section{MIDAS-R}
\label{chap:trsnsMidasR}
Der Midas-R Algorithmus speichert weitere Features �ber die Netzwerke. Dazu geh�rt die Gesamtzahl an Kanten, welche von einem Knoten ausgehen und die Aktuelle Anzahl an Kanten die von einem Knoten ausgehen. Aus diesem Grund sind die Berechnungen zur Ausrei�er-Erkennung deutlich umfangreicher. Das Bewirkt, dass die Laufzeit des Algorithmus mit den Daten aus \autoref{chap:trsnsMidas} zu lange ist. Deshalb musste ein L�sung gefunden werden um den Umfang der Daten zu reduzieren, w�hrend die Informationen dennoch erhalten bleiben. Dazu wurde eine Hauptkomponenten Zerlegung durchgef�hrt, welche die Gr��e der Adjazenzmatrix verringert. Der Nachteil hiervon ist das nicht mehr genau gesagt werden kann, welche Kante genau der Ausrei�er ist. 

\workTodo{Midas R liefert eigentlich mehrere Ausrei�er Scores es w�re vielleicht interessant diese einzeln zu betrachten und nicht zusammenaddiert.}
\workTodo{Hab hier das mit der Hauptkomponentenzerlegung gemacht. Wenn es Ergebnisse hierf�r gibt. Kann ich das hier noch erkl�ren}

\newpage
\chapter{Verwendete Daten}

\section{Numenta Zeitreihen Daten}
Bei diesem Datensatz handelt es sich um k�nstlich erzeugte Zeitreihen der Numenta Gruppe. Diese Zeitreihen enthalten unterschiedliche Arten von Ausrei�ern. Dadurch kann untersucht werden f�r welche Ausrei�er Typen die Algorithmen gut geeignet sind. F�r die Tests auf multivariaten Zeitreihen wurden neue Zeitreihen erzeugt. Dabei wurde f�r die erste Dimension eine Zeitreihe der Numenta Gruppe verwendet. F�r weitere Dimensionen wurde auf eine Zeitreihe ohne Ausrei�er zur�ck gegriffen.

\workTodo{vielleicht noch Bild von einer Zeitreihe einf�gen.}
\workTodo{Vielleicht Numenta verlinken}
\newpage
\chapter{Statische Algorithmen }
\workTodo{Labels f�r die einzelnen Texte umbenennen}
Es ist mit dieser Art von Algorithmen m�glich Ausrei�er in einer vollst�ndigen und abgeschlossenen Zeitreihe zu identifizieren. Es werden zwei Algorithmen vorgestellt, ein auf Percolation basierender Algorithmus und ein auf IsoMap basierender Algorithmus. Beide Algorithmen wurden dazu entwickelt Ausrei�er in unterschiedlichen Typen von Datens�tzen zu erkennen (z.B. Videos, Bilder, Netzwerke). Voraussetzung hierf�r ist lediglich, dass eine Distanz zwischen unterschiedlichen Elementen des Datensatzes berechnet werden kann  \citep[vgl.][S.~2]{10.3389/fphy.2019.00194}. Im Folgenden werden die Algorithmen, hinsichtlich ihrer F�higkeit Ausrei�er in Zeitreihen zu identifizieren evaluiert.

F�r beide Algorithmen gilt, dass die Zeitreihe zun�chst in ein Netzwerk umgewandelt werden muss. Hierzu wird ebenfalls Formel 1 verwendet. Beide Algorithmen liefern lediglich einen Ausrei�er Score zur�ck. Um zu bestimmen inwiefern ein Element konkret ein Ausrei�er ist, wird zun�chst den Mittelwert und die Standardabweichung des Outlier Scores berechnet. Falls ein Element in Abh�ngigkeit von der Standardabweichung sehr stark vom Mittelwert abweicht wird das Element als Ausrei�er klassifiziert.
\section{IsoMap Basierter Algorithmus}
\label{chap:rw}
Der Grundgedanke hinter diesem Ansatz ist, dass Informationen �ber Ausrei�er bei der Reduzierung der Dimensionalit�t mit dem IsoMap Algorithmus verloren gehen. Insofern versucht wird, die Informationen zu rekonstruieren und mit der urspr�nglichen Matrix vergleicht, k�nnen gro�e Abweichungen bei Ausrei�er Elementen festgestellt werden \citep[vgl.][S.~3]{10.3389/fphy.2019.00194}. 


\subsection{IsoMap}
\label{sec:rw-gl}
Beim IsoMap handelt es sich um einen Algorithmus zur nichtlinearen Dimensionsreduktion. Zun�chst werden beim IsoMap die Nachbarn eines jeden Punktes �ber K-Nearest Neighbor bestimmt. Anschlie�end wird jeder Punkt mit den gefundenen Nachbarn verkn�pft, wodurch ein neuer K�rper entsteht. Daraufhin wird eine neue Distanzmatrix auf dem entstandenen K�rper berechnet. Diese Matrix kann auch als geod�tische Distanzmatrix bezeichnet werden und wird im weiteren Verlauf des Algorithmus ben�tigt. Der Zweck des Ablaufs ist es das nichtlineare Zusammenh�nge in der anschlie�enden Dimensionsreduktion erhalten bleiben. Die Dimensionsreduktion erfolgt anschlie�end �ber Eigenvektor ? \citep[vgl.][S.~3]{Tenenbaum2319}.
\workTodo{Noch nach Seite f�r Quelle suchen}

\begin{figure}[H]
	\centering
	\includegraphics[width=5cm]{fig/isomap_circle.png}
	\caption{Funktionswei�e IsoMap}
	\label{img:isomap}
\end{figure}


\subsection{IsoMap Basierter Algorithmus}
\label{sec:rw-gl-cd}
Mithilfe des IsoMap Algorithmus wurden neue Features berechnet. Im n�chsten Schritt wird versucht aus diesen Features die urspr�ngliche Distanzmatrix zu rekonstruieren. Nun kann die Distanzmatrix aus \autoref{sec:rw-gl} mit dieser Distanzmatrix verglichen werden. Dazu wird die Pearson Korrelation zwischen den jeweiligen Vektoren der Matrizen berechnet. F�r Ausrei�er wird erwartet, dass die �hnlichkeit sehr niedrig ist, da die Informationen �ber sie bei der Reduktion verloren gehen \citep[vgl.][S.~3]{10.3389/fphy.2019.00194}.

\subsection{Implementierung}
\label{sec:im-implement}
Da f�r die Implementierung des Algorithmus viele Berechnungen mit Matrizen durchgef�hrt werden m�ssen, wurde hierzu auf Python/Numpy zur�ckgegriffen. F�r den IsoMap Algorithmus existierte eine sehr gute Implementierung in SciKitLearn, deshalb wurde auf diese zur�ckgegriffen. An den Algorithmus k�nnen verschieden Parameter �bergeben werden, es handelt sich hierbei um dieselben Parameter, welche auch an den IsoMap Algorithmus �bergeben werden k�nnen.

\subsection{Ergebnisse}
\label{sec:im-results}
\begin{table}
\caption{IsoMap Performance}
\begin{tabular}{ |p{4.5cm}||p{4.5cm}||p{3cm}|}
	\hline
	\textbf{Ausrei�er Typ}& \textbf{Datei Name}&
	\textbf{1D}\\
	\hline
	\hline
	Einzelne Peaks & anomaly-art-daily-peaks & *\\
	\hline
	Zunahme an Rauschen & anomaly-art-daily-increase-noise &**\\
	\hline
	Signal Drift & anomaly-art-daily-drift &**\\
	\hline
	Kontinuierliche Zunahme der Amplitude& art-daily-amp-rise & **\\
	\hline
	Zyklus mit h�herer Amplitude & art-daily-jumpsup &*\\
	\hline
	Zyklus mit geringerer Amplitude & art-daily-jumpsdown & **\\
	\hline
	Zyklus-Aussetzer & art-daily-flatmiddle &*\\
	\hline
	Signal-Aussetzer & art-daily-nojump & -\\
	\hline
	Frequenz�nderung & anomaly-art-daily-sequence-change &-\\
	\hline
\end{tabular}
\end{table}

Der IsoMap Algorithmus liefert eher schwache Ergebnisse bei der Erkennung von Ausrei�ern in Zeitreihen. Hauptproblem hierbei ist, dass starke Anstiege in der Zeitreihe zu starken Ausschl�gen im Ausrei�er Score f�hrt. An den Stellen, an welchen sich tats�chlich Ausrei�er befinden, kommt es je nach Ausrei�er Typ ebenfalls zu Ausschl�gen im Ausrei�er Score. Jedoch sind diese Ausschl�ge etwa so gro� wie die der �berg�nge. Deshalb ist es nur schwer m�glich die Ausrei�er zu identifizieren. Das selbe Problem trat auch in \citep {Uzun} bei der Ausrei�er Erkennung mit dem Random Walk Algorithmus auf. Um das Problem zu l�sen wurde hierbei eine Gl�ttung der Zeitreihe durchgef�hrt. Dadurch sind die �berg�nge zwischen den Abschnitten nicht mehr so pl�tzlich und werden nicht mehr als Ausrei�er markiert \citep[vgl.][S.~31,36]{Uzun}. Um Zuk�nftig bessere Ergebnisse zu erzielen, w�re das ein m�glicher Ansatz.\\
Des Weiteren ist zu erkennen, dass der Algorithmus f�r einige Ausrei�er Typen nicht geeignet ist. Hierzu geh�ren die Ausrei�er Typen, welche sich nicht vom Wertebereich her �ndern.  

\workTodo{Wie genau bezieht man sich auf seine eigene Ausarbeitung} 

\begin{figure}[H]
	\centering
	\includegraphics[width=10cm]{fig/resultsIsoMap/problem_transitions.PNG}
	\caption{Problem �berg�nge}
	\label{img:perc_proc}
\end{figure}


\newpage
\section{Perculation}
\label{sec:per-intro}
Bei diesem Algorithmus werden schrittwei�e die Kanten mit den h�chsten Gewichten aus der Distanzmatrix entfernt. Ziel dieses Prozesses ist es Ausrei�er vom Hauptcluster zu trennen. Dabei kann davon ausgegangen werden, dass Ausrei�er h�here Kantengewichte zu ihren Nachbarn aufweisen und deshalb schneller separiert werden. Sobald ein Knoten komplett separiert ist, wird ihm ein Ausrei�er Score zugeordnet. Der Wert des Ausrei�er Scores wird �ber die zuletzt entfernte Kante des Knoten definiert \citep[vgl.][S.~3]{10.3389/fphy.2019.00194}. 

\begin{figure}[H]
	\centering
	\includegraphics[width=15cm]{fig/perco_procedure.png}
	\caption{Ablauf Perculation basierter Algorithmus}
	\label{img:perc_proc}
\end{figure}

\subsection{Implementierung}
\label{sec:per-imp}
Aus denselben Gr�nden wie in \autoref{sec:im-implement} erl�utert, wurde f�r die Implementierung auf Python/Numpy zur�ckgegriffen.  Da die Distanzmatrix sehr umfangreich werden kann wurden einige Ver�nderungen an dem Algorithmus vorgenommen, um ihn Performanter zu machen. Eine Modifikation, die vorgenommen wurde, ist das die Kanten nicht einzeln, sondern in Gruppen entfernt werden. Dadurch muss seltener �berpr�ft werden, ob ein Knoten mittlerweile komplett isoliert ist. Au�erdem wurde ein Abbruchkriterium implementiert, bei welchem der Algorithmus angehalten wird sobald eine bestimmte Prozentzahl an Kanten entfernt wurde. Dies hat keine Auswirkungen auf die Qualit�t der Ausrei�er Erkennung, da Ausreiser �blicherweise bereits zu beginn des Algorithmus isoliert werden. Der Algorithmus berechnet f�r jedes Element der Zeitreihe einen Ausrei�er Score. Allerdings k�nnen die Ausrei�er Scores sehr stark schwanken. Deshalb ist es schwierig Ausrei�er zu identifizieren, welche sich �ber mehrere Zeitschritte erstrecken, da kein kontinuierliches Ansteigen des Scores beobachtet werden kann. Eine M�glichkeit, um diese Art der Ausrei�er trotzdem zu identifizieren, ist es ein Sliding Window Verfahren einzusetzen. Dabei wird der Ausrei�er Score f�r jedes Element neu berechnet, indem ein Mittelwert �ber die Zeitpunkte vor einem und nach einem Element gebildet wird. Dadurch werden die Schwankungen im Ausrei�er Score abgemildert. Prinzipiell ist der Algorithmus parameterfrei, durch die Ver�nderungen kann jedoch die Gr��e des Sliding Window als Parameter �bergeben werden.

\begin{figure}[h]
	\centering
	\subfloat{
		\includegraphics[width=0.5\textwidth]{fig/resultsPercolation/art_daily_jumpsup}}
	\subfloat{
		\includegraphics[width=0.5\textwidth]{fig/resultsPercolation/art_daily_jumpsup_ohne}}
	\caption{Vergelich Perculation Algorithmus mit Sliding Window Verfahren und ohne Sliding Window Verfahren}
	\label{img:vardir1}
\end{figure}

\subsection{Ergebnisse}
\label{sec:per-result}

\begin{table}
\caption{Perculation Time Series Performance}
\begin{tabular}{ |p{6cm}||p{6cm}||p{3cm}|}
	\hline
	\textbf{Ausrei�er Typ}& \textbf{Datei Name}&
	\textbf{1D}\\
	\hline
	\hline
	Einzelne Peaks & anomaly-art-daily-peaks & *\\
	\hline
	Zunahme an Rauschen & anomaly-art-daily-increase-noise &****\\
	\hline
	Signal Drift & anomaly-art-daily-drift &***\\
	\hline
	Kontinuierliche Zunahme der Amplitude& art-daily-amp-rise & ***\\
	\hline
	Zyklus mit h�herer Amplitude & art-daily-jumpsup &****\\
	\hline
	Zyklus mit geringerer Amplitude & art-daily-jumpsdown & ****\\
	\hline
	Zyklus-Aussetzer & art-daily-flatmiddle &****\\
	\hline
	Signal-Aussetzer & art-daily-nojump & -\\
	\hline
	Frequenz�nderung & anomaly-art-daily-sequence-change &-\\
	\hline
\end{tabular}
\end{table}

\label{sec:per-results-text}
Die Qualit�t der Ausrei�er Erkennung mit dem Perculation Algorithmus kann gro�enteils als gut bis sehr gut bezeichnet werden. Lediglich die Ausrei�er Typen Einzelne Peaks, Signal Aussetzer und Frequenz�nderung k�nnen vom Algorithmus nicht erkannt werden. Bei den einzelnen Peaks liegt das an der Verwendung des Sliding Window Verfahren, dadurch werden die Ausschl�ge im Ausrei�er Score weggemittelt und k�nnen nur noch sehr schlecht identifiziert werden. Wird jedoch kein Sliding Window Verfahren angewandt k�nnen die Ausrei�er sehr gut identifiziert werden. Signal Aussetzer und Frequenz�nderungen k�nnen vom Perculation Algorithmus nicht identifiziert werden, weil die Werte der Zeitreihe hierbei nicht von den Werten der restlichen Zeitreihe abweichen.   



\workTodo{Die richtigen Ergebnisse rein machen und bisschen was dr�ber schreiben}

\workTodo{Die Bilder vielleicht noch �berarbeiten, sodass sie sch�ner aussehen. Vielleicht auch noch die Tabelle mit den Sternen rein machen. Vielleicht die beiden Graphiken zu einer Zusammenf�hren.}

\workTodo{Fragestellung: Inwieweit k�nnen vielleicht auch andere Datens�tze in Graphen umgewandelt werden, sodass z.B. der Netismile darauf angewendet werden kann. }
\newpage
\chapter{Dynamische Algorithmen zur Ausrei�er Erkennung}

In diesem Kapitel werden zwei Algorithmen zur dynamischen Erkennung von Ausrei�ern vorgestellt. Hierbei handelt es sich um den Netismile (vgl. \autoref{sec:ns}) und den MIDAS (vgl. \autoref{sec:midas}) Algorithmus. Dynamische Algorithmen k�nnen im Gegensatz zu statischen Algorithmen, Ausrei�er in Echtzeitdaten finden. Dies kann in der Praxis sehr wichtig sein, da Ausrei�er m�glichst schnell gefunden werden m�ssen um finanzielle Sch�den abzuwenden.  Die dynamischen Algorithmen wurden genauso wie die statischen Algorithmen, von uns so gestaltet, das sie mit unterschiedlichen Daten Typen umgehen k�nnen. In unseren Experimenten wurden die Algorithmen auf Netzwerk- und Zeitreihen Daten angewandt.

\section{Umwandlung der Daten in ein Netzwerk}
Damit die Algorithmen auf den Daten angewandt werden k�nnen m�ssen diese ebenfalls zun�chst in Netzwerkdaten umgewandelt werden. Die Umwandlung funktioniert hierbei �hnlich wie in \autoref{sec:trsnsNeti}. Jedoch werden nicht alle Daten auf einmal in ein gro�es Netzwerk umgewandelt. Es werden anstatt dessen immer kleine Abschnitte der Daten in kleine Netzwerke umgewandelt. In der Praxis kann somit immer kurz abgewartet werden bis einige Daten eintreffen. Diese Daten k�nnen dan in ein Netzwerk umgewandelt werden und dan nach Ausrei�ern darin suchen. Die L�nge des Intervalls kann als Hyperparameter an den Algorithmus �bergeben werden. Je nach Zeitreihe funktionieren unterschiedliche Intervallgr��en besser oder schlechter. Insofern die Zeitreihe eine Saisonalit�t aufweist, kann diese bestimmt und als Intervallgr��e genutzt werden. Hier musste das ganze noch in CSV Dateien geschrieben werden damit das alles funktioniert. In Fig ist graphisch dargestellt wie die Umwandlung funktioniert. 

\begin{figure}[H]
	\centering
	\includegraphics[width=13cm]{fig/tsToNet/tsToCsv}
	\caption{Umwandlung einer Zeitreihe in Netzwerk}
	\label{img:tsToNet}
\end{figure}

\label{ssec:trsnsCSV}
Bei Umwandlung in CSV kann bei Netismile die Gewichtung �bergeben werden bei MIDAS geht das nicht. Deshalb muss die selbe Kante mehrmals in Anh�ngigkeit der Gewichtung an die CSV Datei �bergeben werden Ursprungsknoten, Zielknoten, Zeitintervall.

\begin{figure}[H]
	\centering
	\includegraphics[width=2cm]{fig/tsToNet/midasData}
	\caption{Datensatz Midas}
	\label{img:tsToNetMiData}
\end{figure}



\label{ssec:trsnsMidasR}
Die Umwandlung der Daten f�r den MIDAS-R Algorithmus stellt ein spezial Fall dar. Der Midas-R Algorithmus speichert weitere Features �ber die Netzwerke. Dazu geh�rt die Gesamtzahl an Kanten, welche von einem Knoten ausgehen und die Aktuelle Anzahl an Kanten die von einem Knoten ausgehen. Aus diesem Grund sind die Berechnungen zur Ausrei�er-Erkennung deutlich umfangreicher. Das Bewirkt, dass die Laufzeit des Algorithmus mit den Daten aus \autoref{chap:trsnsMidas} zu lange ist. Deshalb musste ein L�sung gefunden werden um den Umfang der Daten zu reduzieren, w�hrend die Informationen dennoch erhalten bleiben. Dazu wurde eine Hauptkomponenten Zerlegung durchgef�hrt, welche die Gr��e der Adjazenzmatrix verringert. Der Nachteil hiervon ist das nicht mehr genau gesagt werden kann, welche Kante genau der Ausrei�er ist. 

\workTodo{Midas R liefert eigentlich mehrere Ausrei�er Scores es w�re vielleicht interessant diese einzeln zu betrachten und nicht zusammenaddiert.}

\newpage
\section{NetSimile}
\label{sec:ns}
In \autoref{ssec:ns-gl} werden die Grundlagen des NetSimile-Algorithmus n�her erl�utert und in den Abschnitten \autoref{sec:ns-ext} und \autoref{sec:ns-ts-1} die Testergebnisse vorgestellt.
\subsection{Grundlagen}
\label{ssec:ns-gl}

NetSimile ist ein skalierbarer Algorithmus zur Erkennung von �hnlichkeiten, sowie Anomalien, in Netzwerken unterschiedlicher Gr��en. Hierf�r wird der Datensatz in gleich gro�e Zeitintervalle unterteilt, um die daraus resultierenden Graphen auf unterschiedliche Merkmale zu untersuchen. Die Merkmale sind hierbei strukturelle Eigenschaften der einzelnen Knoten wie bspw. die Dichte eines Knotens oder die Anzahl an Nachbarn in einem Ego-Netzwerk. Die Signatur ergibt sich aus den einzelnen Aggregationen der Knoten wie bspw. dem Median aus der Dichte der jeweiligen Knoten. So entsteht bspw. aus sieben Merkmalen und f�nf Aggregationen ein Signaturvektor mit 35 verschiedenen Signaturen. So erm�glicht der Signaturvektor die Beschreibung sowie den Vergleich der einzelnen Graphen. F�r den Vergleich wird die Canberra-Distanz aus den beiden Signaturvektoren zweier zeitlich nebeneinander liegenden Graphen berechnet. \citep[vgl.][S.~1]{Netsimile} Als Input f�r diesen Algorithmus wird eine Menge von $k$-anonymisierten Netzwerken mit beliebig unterschiedlichen Gr��en, die keine �berlappenden Knoten oder Kanten besitzen, herangezogen. Das Resultat sind Werte f�r die strukturelle �hnlichkeit oder Abstands eines jeden Paares der gegebenen Netzwerke bzw. ein Merkmalsvektor f�r jedes Netzwerk. \citep[vgl.][S.~1]{Netsimile} NetSimile durchl�uft drei Schritte, die im Folgenden erl�utert werden.
\subsubsection{Extrahierung von Merkmalen}
F�r jeden Knoten $i$ werden, basierend auf ihren Ego-Netzwerken, die folgenden Merkmale generiert:
\begin{table}[H]
	\centering
	\begin{tabular}{p{0.13\linewidth}|p{0.8\linewidth}}
		\toprule
		$\overline{d}_i = |N(i)|$ & Die Anzahl der Nachbarn (\dah Grad) von Knoten $i$, wobei $N(i)$ die Nachbarn von Knoten $i$ beschreibt.\\
		\midrule
		$\overline{c}_i$ & Der Clustering-Koeffizient von Knoten $i$, der als die Anzahl von Dreiecken, die mit Knoten $i$ verbunden sind, �ber die Anzahl von verbundenen Dreiecken, die auf Knoten $i$ zentriert sind, definiert ist. \\
		\midrule
		$d_{N(i)}$ & Die durchschnittliche Anzahl der Nachbarn von Knoten $i$, die zwei Schritte entfernt sind. Dieser wird berechnet als \workTodo{Paper Seite 2 unten Formel einf�gen} \\
		\midrule
		$c_{N(i)}$ & Der durchschnittliche Clustering-Koeffizient von $N(i)$, der als \workTodo{Paper Seite 2 unten Formel einf�gen} berechnet wird. \\
		\midrule
		$|E_{ego(i)}|$ &  Die Anzahl der Kanten im Ego-Netzwerk vom Knoten $i$, wobei $ego(i)$ das Ego-Netzwerk von $i$ zur�ckgibt. \\
		\midrule
		$|E^{\circ}_{ego(i)}|$ & Die Anzahl der von $ego(i)$ ausgehenden Kanten. \\
		\midrule
		$|N(ego(i))|$ & Die Anzahl von Nachbarn von $ego(i)$. \\
		\bottomrule
	\end{tabular}
	\caption{Inhalte des Merkmalsvektors}
	\label{tab:netfeat}
\end{table}

%\begin{mldescription}
%	\mlitem{$\overline{d}_i = |N(i)|$} Die Anzahl der Nachbarn (d.h. Grad) von Knoten $i$, wobei $N(i)$ die Nachbarn von Knoten $i$ beschreibt.
%	\mlitem{$\overline{c}_i$} Der Clustering-Koeffizient von Knoten $i$, der als die Anzahl von Dreiecken, die mit Knoten $i$ verbunden sind, �ber die Anzahl von verbundenen Dreiecken, die auf Knoten $i$ zentriert sind, definiert ist.
%	\mlitem{$d_{N(i)}$} Die durchschnittliche Anzahl der Nachbarn von Knoten $i$, die zwei Schritte entfernt sind. Dieser wird berechnet als \workTodo{Paper Seite 2 unten Formel einf�gen}
%	\mlitem{$c_{N(i)}$} Der durchschnittliche Clustering-Koeffizient von $N(i)$, der als \workTodo{Paper Seite 2 unten Formel einf�gen} berechnet wird.
%	\mlitem{$|E_{ego(i)}|$} Die Anzahl der Kanten im Ego-Netzwerk vom Knoten $i$, wobei $ego(i)$ das Ego-Netzwerk von $i$ zur�ckgibt.
%	\mlitem{$|E^{\circ}_{ego(i)}|$} Die Anzahl der von $ego(i)$ ausgehenden Kanten. 
%	\mlitem{$|N(ego(i))|$}  Die Anzahl von Nachbarn von $ego(i)$. 
%\end{mldescription}

\subsubsection{Aggregierung von Merkmalen}
Im n�chsten Schritt wird f�r jeden Graphen \textit{$G_j$} eine $Knoten \times Merkmal$-Matrix $F_{G_j}$ zusammengefasst. Dieser besteht aus den Merkmalsvektoren aus Schritt 1.
Da der Vergleich von $k$-ten $F_{G_j}$ sehr aufw�ndig ist, wird f�r jede $F_{G_j}$ ein Signaturvektor $\vec{s}_{G_j}$ ausgegeben. Dieser aggregiert den Median, den Mittelwert, die Standardabweichung, die Schiefe, sowie die Kurtosis der Merkmale aus der Matrix.  

\subsubsection{Vergleich der Signaturvektoren}
\label{sec:ns-gl-cd}

F�r die Ausrei�ererkennung werden die letzten drei Graphen anhand der Canberra-Distanz-Funktion, die als �hnlichkeitsma� dient, herangezogen. Steigt die Canberra-Distanz zwischen zwei Graphen oberhalb des Schwellwerts so wird dies im Algorithmus festgehalten. Falls der darauf folgende Graph ebenfalls oberhalb des Schwellwerts liegt, so wird dieser als Ausrei�er definiert. Dadurch wird die Anzahl der Ausrei�er reduziert, damit nur diejenigen identifiziert werden, bei denen ein Trend hin zu einem abnormalen Verhalten erkennbar ist.

Der Algorithmus arbeitet dabei dynamisch, da die Signaturen der Graphen in einzelne Teilberechnungen aufgeteilt und zwischengespeichert werden k�nnen, ohne das eine Neuberechnung notwendig ist. Der Schwellwert wird aus dem Median und dem Mittelwert berechnet, die ebenfalls zwischengespeichert und nach Bedarf um weitere Graphen erg�nzt werden k�nnen.

\FloatBarrier

\subsection{Anwendung auf Netzwerkdaten}
\label{sec:ns-ext}
Beim ersten Versuch den Algorithmus auf Netzwerkdaten anzuwenden, wurde die nachfolgende Problematik festgestellt.

Der Algorithmus verwendet eine Bibliothek \textit{igraph}, welche Kanten zwischen zwei Knoten nur einmalig hinzuf�gen kann. Beim Eliminieren der Duplikate wird aber ein Drittel des Datensatzes nicht ber�cksichtigt, wodurch wertvolle Informationen bei der Ausrei�ererkennung verloren gehen. Aus diesem Grund wurden die Netzwerkdaten soweit angepasst, dass Mehrfachverbindungen zwischen zwei Kanten aufsummiert werden und als Gewichtung dieser Kante hinzugef�gt wird. 

\begin{lstlisting}[language=Python, caption=Gewichtung als neues Feature, label=lst:netsimile:code1]
for i in range(len(e_list)): 
	g.add_edge(e_list[i][0], e_list[i][1], weight=e_list[i][2])
\end{lstlisting}

Dadurch kann der Datensatz zum einen vollst�ndig analysiert werden und zum anderen kann dadurch ein weiteres Feature hinzugef�gt werden, dass durch die f�nf verschiedenen Aggregationen den Signaturvektor um diese f�nf Werte erweitert.

Dadurch dass der Datensatz zuerst eingelesen und in einen Graphen transformiert wird und anschlie�end aus dem Graphen die jeweiligen Features extrahiert werden, verliert der Algorithmus au�erordentlich an Performanz. Des Weiteren wird im ersten Schritt der maximale Knoten-Wert als Gr��e des Graphens �bergeben. Wird bspw. f�r jeden Mitarbeiter eine eigene ID �bergeben und diese ID inkrementell erh�ht, so kann es sein, dass aus einem Netzwerk mit 20 verschiedenen Knoten ein Graph erzeugt wird, der 1000 Knoten erzeugt, weil eine ID mit dem Wert 1000 vorhanden ist. Dadurch b��t die Performanz an Geschwindigkeit ein, da Iterationen nicht �ber die 20 Knoten durchgef�hrt werden, sondern �ber 1000. Hierbei muss entweder der Datensatz vorab angepasst werden, indem die IDs neu vergeben werden oder der Algorithmus muss grundlegend neu aufgebaut werden. Dies w�re grundlegend m�glich, da der Algorithmus Graphen als Ausrei�er zur�ck gibt und keine Knoten. 

Da der Fokus auf der Anwendung von Zeitreihen liegt, werden Optimierungen erst im Abschnitt \autoref{sec:ns-ts-1} in Betracht gezogen. 

\subsubsection{Anwendung auf ENRON-Datensatz}
\label{sec:ns-enron}

Da der Enron Datensatz ebenfalls von einem anderen Paper analysiert und ver�ffentlicht wurde (vgl. \autoref{chap:ap-data}), k�nnen die dort erkannten Ausrei�er zum Vergleich in Form eines gelabelten Datensatz herangezogen werden.
Betrachtet man in diesem Kontext den Ausrei�erscore, ist gut zu erkennen, dass der Ausrei�er Ende 2001 als alleiniger herausstechender Ausrei�er ebenso im NetSimile wiederzufinden ist. Grundlegend ist ebenso zu erkennen, dass die Ausrei�er sich nur sehr wenig voneinander unterscheiden, wodurch sich eine Klassifizierung innerhalb des Ausrei�erscores als schwierig erweist. Die Extrahierung weiterer Features k�nnte dieses Problem l�sen, wobei dies nicht im Rahmen dieses Forschungsprojektes behandelt werden soll, da der Fokus auf Zeitreihen liegt. Der Datensatz innerhalb von zwei Minuten analysiert werden, womit der NetSimile-Algorithmus performant zu sein scheint.

\begin{figure}[H]
	\centering
	\includegraphics[height=0.6\linewidth,width=\linewidth]{fig/netsimile/anomalie_3}
	\caption{Ausrei�er-Score im Enron-Datensatz mit dem NetSimile-Algorithmus}
	\label{img:netsimile:anomalie_3}
\end{figure}


Betrachtet man die Differenz aus dem Durchschnitt der Signaturvektoren und den der Ausrei�ergraphen in einer \textit{Headmap} kann man erkennen, dass die Ausrei�er vorwiegend den besonders gro�en Ego-Netzwerken und einer hohen Anzahl an E-Mails verschuldet sind. Vergleicht man die Zeitleiste (\citep{EnronTimeline}) mit den Ausrei�erdaten, erkennt man einen starken Anstieg des Email-Verkehrs im Oktober 2001. Hier wurde von Enron ein Report ver�ffentlicht �ber einen Quartals-Verlust von 618 Millionen US-Dollar und einer Reduzierung des Eigenkapitals um 1.2 Milliarde. Im Dezember 2001 erkennt man einen verh�ltnism��ig geringen Email-Verkehr. Betrachtet man hier die Zeitleiste, so wurden 4000 Mitarbeiter in dieser Zeitspanne entlassen. \workTodo{Quelle die Webseite}



\begin{figure}[H]
	\centering
	\includegraphics[height=0.8\linewidth,width=\linewidth]{fig/netsimile/heatmap_3}
	\caption{Darstellung der Ausrei�er in Heatmaps}
	\label{img:netsimile:heatmap_3}
\end{figure}

\subsubsection{Anwendung auf Darpa-Datensatz}
\label{sec:ns-darpa}

Beim Darpa-Datensatz k�nnen die Aurei�er besser klassifiziert werden. Die Gr�nde hierf�r liegen auf der Gr��e und Vielfalt des Datensatzes. Der Enron.Datensatz hat eine Gr��e von 1MB und rund 50.000 Kanten. Der Darpa-Datensatz hingegen hat eine Gr��e von 50 MB mit 4.5 Mio Kanten. Die Berechnung hat dabei eine L�nge von drei Stunden. Haben wir bei der Dateigr��e den Faktor 50 und bei der Kantenanzahl den Faktor 90, so ist bei der Berechnungszeit der Faktor 90 wiederzufinden. Betrachtet man die Laufzeit, so kann eine lineare Abh�ngigkeit zwischen Kantenanzahl und der ben�tigten Berechnungszeit festgestellt werden.

\begin{figure}[H]
	\centering
	\includegraphics[height=0.6\linewidth,width=\linewidth]{fig/netsimile/anomalie_4}
	\caption{Ausrei�er-Score im Darpa-Datensatz mit dem NetSimile-Algorithmus}
	\label{img:netsimile:anomalie_4}
\end{figure}

\subsection{Anwendung auf Zeitreihen}
\label{sec:ns-ts-1}

Wird der Algorithmus auf Zeitreihen anwendet, entsteht folgendes Problem. Bei der Transformation der Daten entstehen vollst�ndige Graphen, wodurch die strukturellen Eigenschaften sowie die daraus resultierenden Merkmale identisch werden, wie in \autoref{img:netsimile:graph_11} deutlich wird. 

\usetikzlibrary{graphs,graphs.standard}
\begin{figure}[H]
	\centering
	\begin{tikzpicture}
		\graph[nodes={draw, circle,fill=black!20,minimum size = 6mm}, clockwise, empty nodes, radius=4cm, n=11] { subgraph K_n };
	\end{tikzpicture}
	\caption{Vollst�ndiger Graph mit 11 Knoten}
	\label{img:netsimile:graph_11}
\end{figure}

So hat bspw. das Feature $|E^{\circ}_{ego(i)}|$ keine Aussagekraft in einem vollst�ndigen Graphen, da jeder Knoten die gleiche Anzahl an Kanten in seinem Ego-Netzwerk aufweist. Subtrahiert man also vom durchschnittlichen Signaturvektor aller Graphen die einzelnen Signaturvektoren, so erkennt man den Wert 0 in allen Headmaps.

Somit m�ssen hierbei f�r vollst�ndige Graphen andere Features extrahiert werden. Au�erdem ist die Laufzeit in gro�en Datens�tzen, wie bspw. dem Darpa-Datensatz mit 3 Stunden Berechnungszeit nicht performant.

Aus diesem Grund werden aus dem NetSimile lediglich die Ans�tze der Merkmals-Extrahierung, die Distanzbildung zweier Signaturvektoren, sowie der Schwellwert f�r die Ausrei�eridentifizierung �bernommen. Das hei�t die Netzwerke der Zeitreihe werden nicht in ein Graphenobjekt umgewandelt, sondern als Adjazenzmatrix gespeichert. Dadurch k�nnen die Features deutlich effizienter berechnet werden. Zudem werden lediglich Merkmale verwendet, die f�r vollst�ndige Graphen geeignet sind. Dabei werden die nachfolgenden Merkmale neu eingef�hrt.

\workTodo{n-te Wurzel bei Formel f�r Geometrischen Mittelwert}
\begin{description}
	\item[$\sum_{i=1}^{n} x_i $]\hfill\\ Summe der Kantengewichte eines Knoten.
	\item[$\frac{1}{n}\sum_{i=1}^{n} x_i $]\hfill\\ Arithmetisches Mittel der Kantengewichte eines Knoten.
	\item[$ \sqrt{\prod\limits_{i = 1}^{n} x_i} $]\hfill\\ Geometrisches Mittel der Kantengewichte eines Knoten.
	\item[$
	x(p) =
	\begin{cases}
		\frac{1}{2}x_(np) +       & \quad \text{if } n \text{ is even}\\
		-(n+1)/2  & \quad \text{if } n \text{ is odd}
	\end{cases}
	$]\hfill\\ Geometrisches Mittel der Kanten mit den 10\% h�chsten Kantengewichten. 
	\item[$\frac{1}{n}\sum_{i=1}^{n} x_i$]\hfill\\ Geometrisches Mittel der Kanten mit den 20\% h�chsten Kantengewichten. 
\end{description}

Auf diesen Merkmalen wurden anschlie�end die bisherigen Aggregation durchgef�hrt. Dadurch konnten erste Ausrei�er in der Zeitreihe gefunden werden (vgl. \autoref{img:netsimile:anomalie_2}).

\begin{figure}[H]
	\centering
	\includegraphics[height=0.6\linewidth,width=\linewidth]{fig/netsimile/anomalie_2}
	\caption{Ausrei�er-Score der vollst�ndigen Graphen mit gewichteten Kanten}
	\label{img:netsimile:anomalie_2}
\end{figure}

Des Weiteren wurde ein neuer Parameter eingef�hrt. �ber diesen kann gesteuert werden zu wie vielen vorherigen Abschnitten die Distanz berechnet werden soll. Dadurch kann gesteuert werden wie schnell ein Algorithmus \workTodo{bitte umformulieren: vergisst}. Eine Auflistung der Parameter des Algorithmus ist in \autoref{table:parmeterNeti} zu sehen.

Um zu untersuchen, wie gut der Algorithmus funktioniert, wurde er auf Zeitreihen getestet. Als Testdaten   wurden, ein- und zweidimensionale Zeitreihen der Numenta-Gruppe verwendet. Diese Zeitreihen enthalten verschiedene Ausrei�ertypen, auf denen die Erkennung der Algorithmus getestet wurde. Die Qualit�t der Ausrei�ererkennung wurde mithilfe eines Punktesystem bewertet. In diesem k�nnen maximal vier Sterne erreicht werden, die daf�r stehen, dass Ausrei�er sehr gut erkannt werden. Null Sterne hingegen bedeuten, dass Ausrei�er �berhaupt nicht erkannt wurden. Die Parameter, welche f�r die Tests gew�hlt werden mussten, werden in \autoref{table:parmeterNeti} beschrieben.\\

\begin{table}[H]
	\centering
	\begin{tabular}{p{0.42\linewidth}|p{0.37\linewidth}|P{0.05\linewidth}|P{0.05\linewidth}}
		\toprule
		\textbf{Ausrei�er Typ}& \textbf{Datei Name}&
		\textbf{1D}&\textbf{2D}\\
		\midrule
		Einzelne Peaks & anomaly-art-daily-peaks & **& -\\
		\midrule
		Zunahme an Rauschen & anomaly-art-daily-increase-noise &****& ***\\
		\midrule
		Signal Drift & anomaly-art-daily-drift &**& -\\
		\midrule
		Kontinuierliche Zunahme der Amplitude& art-daily-amp-rise & ****& ***\\
		\midrule
		Zyklus mit h�herer Amplitude & art-daily-jumpsup &****& *\\
		\midrule
		Zyklus mit geringerer Amplitude & art-daily-jumpsdown & ****& -\\
		\midrule
		Zyklus-Aussetzer & art-daily-flatmiddle &****& ***\\
		\midrule
		Signal-Aussetzer & art-daily-nojump & ****& ***\\
		\midrule
		Frequenz�nderung & anomaly-art-daily-sequence-change &****& ***\\
		\bottomrule
	\end{tabular}
	\caption{NetSimile-Performance auf Zeitreihen}
	\label{table:performanceNeti}
\end{table}

\autoref{table:performanceNeti} zeigt die Ergebnisse der Tests. Es ist zu erkennen, dass die Qualit�t der Ausrei�ererkennung im eindimensionalen Fall sehr gut ist. Lediglich einzelne Peaks k�nnen durch den Algorithmus nicht als Ausrei�er identifiziert werden. Au�erdem wird bei \textit{Signal Drifts} und der kontinuierlichen Zunahme der Amplitude lediglich der Anfang des Ausrei�ers detektiert. Aus diesem Grund wurde eine Bewertung mit drei Sternen vergeben. Bei der Betrachtung der Graphiken in \autoref{sec:appendix_net_one} \workTodo{richtige Referenz einf�gen} und \autoref{sec:appendix_net_two} ist zu erkennen, dass das sechste oder siebte Intervall der Zeitreihe h�ufig als Ausrei�er markiert wird. Der Grund hierf�r ist, das bei einer Fenstergr��e von f�nf f�r die ersten f�nf Abschnitte kein Ausrei�er-Score berechnet wird. Dadurch ist die Standardabweichung zu Beginn sehr niedrig wodurch Abschnitte schnell als Ausrei�er gekennzeichnet werden. Dieser Umstand wurde bei der Bewertung in \autoref{table:performanceNeti} nicht ber�cksichtigt. Im zweidimensionalen Fall ist die Qualit�t der Ausrei�ererkennung etwas durchwachsener. Auffallend ist, dass Zyklen mit h�herer und niedriger Amplitude nicht als Ausrei�er erkannt werden. Insbesondere ist dies auff�llig, da diese Ausrei�ertypen �blicherweise zuverl�ssig erkannt werden (vgl. \autoref{sec:rw-gl}).  Au�erdem ist der Algorithmus im zweidimensionalen Fall nicht mehr dazu in der Lage \textit{Signal Drifts } zu erkennen. Andere Ausrei�ertypen k�nnen durch den Algorithmus weiterhin erkannt werden, jedoch oftmals nicht mit der selben Qualit�t. 

\begin{table}[H]
	\centering
	\begin{tabular}{p{0.13\linewidth}|p{0.815\linewidth}}
		\toprule
		\textbf{Parameter}& \textbf{Beschreibung}\\
		\midrule
		Periodizit�t & Wie in \autoref{sec:trsnsNeti} \workTodo{Referenz sollte glaube ich autoref -> sec:trsnsNeti sein} erl�utert muss die Zeitreihe in kleinere Intervalle aufgegliedert werden. �ber diesen Parameter wird die Gr��e der Intervalle gesteuert. F�r die Tests wurde der Parameter auf 288 gesetzt, da es sich hierbei um die Saisonalit�t der Zeitreihen handelt.\\
		\midrule
		Fenstergr��e & Wie in \autoref{sec:optiNeti} erkl�rt, bestimmt dieser Parameter die Anzahl der vorangegangenen Abschnitte zu welchen die Canberra-Distanz berechnet wird. Dieser Parameter wurde f�r die Tests auf 5 gesetzt.\\
		\midrule
		Abweichung & Legt fest ab wann es sich bei einem Abschnitt um einen Ausrei�er handelt. Der Parameter wurde f�r die Tests auf 3 gesetzt. Bedeutet wenn der Ausrei�er Score um das dreifache der Standardabweichung vom Durchschnitt abweicht, wird der Abschnitt als Ausrei�er gekennzeichnet.\\
		\bottomrule
	\end{tabular}
	\caption{Parameter des NetSimile f�r die Anwendung auf Zeitreihen}
	\label{table:parmeterNeti}
\end{table}


%\begin{table}[H]
%	\centering
%	\begin{tabular}{p{0.18\linewidth}|p{0.19\linewidth}|P{0.05\linewidth}|p{0.35\linewidth}|p{0.09\linewidth}}
%		\bottomrule
%		\textbf{Ausrei�er Typ}& \textbf{Datei Name}&
%		\textbf{1D}&\textbf{Beschreibung}&\textbf{Laufzeit}\\
%		\midrule
%		Einzelne Peaks & anomaly-art-daily-peaks & **& Zwei gemeinsame Peaks werden erkannt, Einzelne eher schlecht
%		&61min\\
%		\midrule
%		Zunahme an Rauschen & anomaly-art-daily-increase-noise &****& Ausrei�er wird erkannt&51\\
%		\midrule
%		Signal Drift & anomaly-art-daily-drift &**& Nur zwei von vier Ausrei�er werden erkannt
%		Die letzten zwei werden also normal definiert
%		&57min\\
%		\midrule
%		Kontinuierliche Zunahme der Amplitude& art-daily-amp-rise & ****& Ausrei�er werden erkannt
%		&54min\\
%		\midrule
%		Zyklus mit h�herer Amplitude & art-daily-jumpsup &****& Ausrei�er werden erkannt
%		&50\\
%		\midrule
%		Zyklus mit geringerer Amplitude & art-daily-jumpsdown & ****& Ausrei�er werden erkannt
%		&51min\\
%		\midrule
%		Zyklus-Aussetzer & art-daily-flatmiddle &****& Ausrei�er werden erkannt
%		&62min\\
%		\midrule
%		Signal-Aussetzer & art-daily-nojump & ****& Ausrei�er werden erkannt
%		&65min\\
%		\midrule
%		Frequenz�nderung & anomaly-art-daily-sequence-change &****& Ausrei�er werden erkannt
%		&49min\\
%		\bottomrule
%	\end{tabular}
%	\caption{Urspr�nglicher Netsimile Performance}
%	\label{table:performanceNeti}
%\end{table}







\newpage
\section{MIDAS}
\label{sec:midas}
Im Folgenden wird der MIDAS-Algorithmus vorgestellt, sowie die Anwendung auf verschiedenen Datentypen n�her erl�utert.

\subsection{Grundlagen}
\label{sec:mc-gl}

MIDAS, Eng. \textit{Microcluster-Based Detector of Anomalies in Edge Streams}, steht f�r einen Algorithmus, der pl�tzlich auftretende Ausbr�che von Aktivit�ten in einem Netzwerk bzw. Graphen erkennt. Dieses vermehrte Auftreten von Aktivit�ten zeigt sich durch viele sich wiederholende Knoten- und Kantenpaare in einem sich zeitlich entwickelnden Graphen, die Mikrocluster bezeichnet werden. Mikrocluster bestehen demnach aus einem vermehrten Vorkommen eines einzigen Quell- und Zielpaares bzw. einer Kante $(u,v)$  \workTodo{Folgender Absatz kann vor der Beschreibung des Algorithmus eingef�gt werden, wie im Paper auch} Dies geschieht in Echtzeit, wobei jede Kante in konstanter Zeit und Speicher verarbeitet wird. In der Theorie garantiert er eine \textit{false-positive}-Wahrscheinlichkeit und ist durch einen 162 bis 644 mal schnelleren Ansatz, sowie einer 42\% bis 48\% h�here Genauigkeit, im Hinblick auf die AUC, \textit{\dq Area under the Curve\dq\space} sehr effektiv. \citep[vgl.][S.~1]{MIDAS}

Urspr�ngliche Anwendungsf�lle f�r MIDAS sind die Erkennung von Anomalien in Computer-Netzwerken, wie SPAM oder DoS-Angriffe, sowie Anomalien in Kreditkartentransaktionen.


\subsubsection{Count-Min-Sketch}
\label{sec:mc-gl-cms}

Damit die relevanten Informationen f�r den Algorithmus mit einem konstanten Speicher verarbeitet werden, wird Count-Min-Sketch genutzt, dass eine Streaming-Datenstruktur mithilfe der Nutzung von Hash-Funktionen entspricht. Count-Min-Sketch z�hlt somit die Frequenz einer Aktivit�t bei Streaming-Daten. Diese Datenstruktur hat ebenfalls den Vorteil, dass man zu Beginn keine Kenntnis �ber die Anzahl an Quell- und Zielpaaren haben muss. \citep{CMS04}

MIDAS verwendet zwei Arten von CMS. Die erste Variante $s_{uv}$ wird als die Anzahl an Kanten von $u$ zu $v$ bis zum aktuellen Zeitpunkt $t$ definiert. Durch die CMS-Datenstruktur werden alle Z�hlungen von $s_{uv}$ approximiert, sodass jederzeit eine ann�hernde Abfrage $\hat{s}_{uv}$ erhalten werden kann.
Die zweite Variante $a_{uv}$ wird als die Anzahl an Kanten von $u$ zu $v$ im aktuellen Zeitpunkt $t$ definiert. Dieser CMS ist identisch zu $s_{uv}$, wobei bei jedem �bergang zum n�chsten Zeitpunkt die Datenstruktur zur�ckgesetzt wird. Dadurch resultiert aus dem CMS f�r den aktuellen Zeitpunkt die ann�hernde Abfrage $\hat{a}_{uv}$. \citep[vgl.][S.~3]{MIDAS}


\subsubsection{Erkennung von Mikrocluster}
\label{sec:mc-gl-dm}

Mithilfe der N�herungswerte $\hat{s}_{uv}$ und $\hat{a}_{uv}$ ist das Detektieren von Mikroclustern m�glich. Hierzu wird der Mean (\dah die durchschnittliche Rate mit der Kanten erscheinen) betrachtet. Es wird hierbei angenommen, dass dieser f�r den aktuellen Zeitpunkt (\zB $t = 10$) �quivalent ist zu dem vor dem aktuellen Zeitpunkt ($t < 10$). Dadurch wird die Annahmen vermieden, dass die Daten auf einer bestimmten zugrundeliegenden Verteilung basieren oder Stationarit�t �ber die Zeit aufweisen.
\\
\\
Durch die genannte Annahme lassen sich vergangene Kanten in zwei Klassen einteilen. Eine f�r den aktuellen Zeitpunkt $t = 10$ und eine f�r alle vergangenen Zeitpunkte $t < 10$. Hierbei betr�gt die Anzahl der Ereignisse zum Zeitpunkt $t = 10$ $a_{uv}$ und die Anzahl der Kanten in vergangenen Zeitpunkten $t < 10$ ist $s_{uv} - a_{uv}$.
\\
\\
Die Auswertung der Daten kann mithilfe des chi-squared goodnes-of-fit test erfolgen. Hierbei wird die Summe der Klassen $t = 10$ und $t < 10$ f�r $\frac{(\text{beobachtet} - \text{erwartet})^2}{\text{erwartet}}$ bestimmt. Bei einer Gesamtanzahl von $s_{uv}$ Kanten ergibt sich, auf Basis eines Mean, f�r $t = 10$ eine erwartete Anzahl von $\frac{s_{uv}}{t}$ Kanten. Analog hierzu ergibt sich f�r $t < 10$ eine erwartete Anzahl an $\frac{t - 1}{t}s_{uv}$ vergangenen Kanten. Daraus ergibt sich f�r die chi-squared Statistik \cite[vgl.][S.~3]{MIDAS}:

\begin{align}
	\chi^2 &= \frac{\left(\text{beobachtet}_{(t = 10)} - \text{erwartet}_{(t = 10)}\right)^2}{\text{erwartet}_{(t = 10)}} \nonumber \\
	&+ \frac{\left(\text{beobachtet}_{(t < 10)} - \text{erwartet}_{(t < 10)}\right)^2}{\text{erwartet}_{(t < 10)}} \nonumber \\
	&= \frac{\left(a_{uv} - \frac{s_{uv}}{t}\right)^2}{\frac{s_{uv}}{t}} + \frac{\left(\left(s_{uv} - a_{uv}\right) - \frac{t - 1}{t}s_{uv}\right)^2}{\frac{t - 1}{t}s_{uv}} \nonumber \\
	&= \frac{\left(a_{uv} - \frac{s_{uv}}{t}\right)^2}{\frac{s_{uv}}{t}} + \frac{\left(a_{uv} - \frac{s_{uv}}{t}\right)^2}{\frac{t - 1}{t}s_{uv}} \nonumber \\
	&= \left(a_{uv} - \frac{s_{uv}}{t}\right)^2\frac{t^2}{s_{uv}(t - 1)}
	\label{eqn:midas:chi2}
\end{align}

Die Gr��en $a_{uv}$ und $s_{uv}$ k�nnen, mithilfe der CMS-Datenstruktur, approximiert werden. Daraus ergibt sich, unter Verwendung der approximierten Gr��en $\hat{a}_{uv}$ und $\hat{s}_{uv}$, der folgende Ausrei�er-Score \cite[vgl.][S.~4]{MIDAS}:

\begin{equation}
	score((u,v,t)) = \left(\hat{a}_{uv} - \frac{\hat{s}_{uv}}{t}\right)^2\frac{t^2}{\hat{s}_{uv}(t - 1)}
	\label{eqn:midas:score}
\end{equation}

Mithilfe des in \autoref{eqn:midas:score} angegeben Ausrei�er-Score l�sst sich eine neue Kante $(u,v)$ zum Zeitpunkt $t$ bewerten. Dieser wird in einem bin�ren Entscheidungsverfahren verwendet, um zu bestimmen, ob es sich bei einer neuen Kante um Anomalie handelt oder nicht. Die Wahrscheinlichkeit von false positive Ergebnissen soll hierbei nicht einen benutzerdefinierten Schwellenwert $\epsilon$ �bersteigen. CMS-Datenstrukturen mit einer angemessenen Gr��e besitzen die Eigenschaft, dass die Approximationen $\hat{a}_{uv}$, f�r beliebige $\epsilon$ und $\nu$, folgende Vorschrift mit einer Wahrscheinlichkeit von mindestens $1 - \frac{\epsilon}{2}$ erf�llen:

\begin{equation}
	\hat{a}_{uv} \leq a_{uv} + \nu \cdotp N_t
\end{equation}

$N_t$ beschreibt hierbei die Anzahl an Kanten zum Zeitpunkt $t$. Eine weitere Eigenschaft der CMS-Datenstrukturen ist, dass diese die tats�chlichen Anzahl an Kanten nur �berbewerten k�nnen:

\begin{equation}
	s_{uv} \leq \hat{s}_{uv}
\end{equation}

Der in \autoref{eqn:midas:score} gegebene Score kann wie folgt angepasst werden:

\begin{equation}
	\~{a}_{uv} = \hat{a}_{uv} - \nu N_t
\end{equation}

Daraus l�sst sich die in \autoref{eqn:midas:chi2} gegebene Statistik anpassen:

\begin{equation}
	\tilde{\chi}^2 = \left(\tilde{a}_{uv} - \frac{s_{uv}}{t}\right)^2\frac{t^2}{s_{uv}(t - 1)}
	\label{eqn:midas:chi2angepasst}
\end{equation}

Bei Verwendung der Teststatistik in \autoref{eqn:midas:chi2angepasst} und eines Schwellenwertes von $\chi^2_{1 - \frac{\epsilon}{2}}(1)$ ergibt sich eine Wahrscheinlichkeit f�r ein false positive Ergebnis von h�chstens $\epsilon$:

\begin{equation}
	P\left(\tilde{\chi}^2 > \chi^2_{1 - \frac{\epsilon}{2}}(1)\right) < \epsilon
	\label{eqn:midas:prob}
\end{equation}

Der Term $\chi^2_{1 - \frac{\epsilon}{2}}(1)$ beschreibt hierbei das $1 - \frac{\epsilon}{2}$-Quantil.

\subsubsection{Die Erweiterung zu MIDAS-R}
Bei dem MIDAS-R Algorithmus handelt es sich um eine Erweiterung des MIDAS Algorithmus. Das R steht hierbei f�r den Relationalen Ansatz des MIDAS-R Algorithmus. Dabei wird versucht die r�umliche oder zeitliche Verkn�pfung zwischen Kanten st�rker zu ber�cksichtigten. Es werden hierzu zwei neue Konzepte eingef�hrt \citep[vgl.][S.~4]{MIDAS}.

\textbf{Temporal Relations: }Durch diesen Ansatz soll der Algorithmus mehr zeitliche Flexibilit�t erhalten. Dabei sollen Kanten aus der j�ngsten Vergangenheit auch in einem neuen Zeitabschnitt ber�cksichtigt werden. Allerdings reduziert um eine bestimmte Gewichtung. Anstatt die CMS Datenstruktur nach jedem Zeitabschnitt zu reseten, werden die Gewichte hierbei um einen bestimmten Prozentsatz reduziert \citep[vgl.][S.~4]{MIDAS}.

\textbf{Spatial Relations: }Hierbei werden zwei neue Features eingef�hrt um verschiedene Ausrei�er-Typen identifizieren zu k�nnen. Die neuen Features werden hierbei in einer CMS Datenstrukturen gespeichert. Der Algorithmus speichert demzufolge diese drei Features:
\workTodo{Aufz�hlung vervollst�ndigen}

\begin{description}
	\item[$ $]\hfill\\ Anzahl an Kanten zwischen Knoten u un Knoten v. Dieses Feature wird auch vom MIDAS Algorithmus verwendet.
	\item[$ $]\hfill\\ Gesamtanzahl an Nachbarknoten eines Knoten u.
	\item[$  $]\hfill\\ Aktuelle Anzahl an Nachbarknoten eines Knoten u.
\end{description}

Aus diesen drei Features wird anschlie�end ein Ausrei�er Score abgeleitet.
 \citep[vgl.][S.~5]{MIDAS}


\subsection{Ausrei�ererkennung auf Netzwerkdaten}
\label{sec:m-ex}

Die Anwendung des MIDAS-Algorithmus auf Netzwerkdaten erfolgte problemlos. Es werden lediglich Daten ben�tigt, die in jeder Zeile aus einem Ausgangs-, einem Zielpunkt, sowie einem Zeitstempel bestehen. In welcher Form der Zeitstempel bereitgestellt wird, ist hierbei unwichtig. Nachfolgend werden die Ergebnisse mithilfe des MIDAS-Algorithmus auf Netzwerkdaten dargestellt.

\begin{figure}[H]
	\centering
	\includegraphics[height=0.6\linewidth,width=\linewidth]{fig/midas/Enron_Anomaly}
	\caption{Der Ausrei�er-Score �ber die Zeit beim ENRON-Datensatz}
	\label{img:midas:enron_anomaly}
\end{figure}


Vergleicht man die Ergebnisse aus \autoref{img:midas:enron_anomaly} mit den Ergebnissen aus \autoref{chap:ap-data} so kann erkannt werden, dass beide Algorithmen einen �hnlichen Verlauf vorweisen. Damit eine genauere Aussage getroffen werden kann, werden die MIDAS-Ergebnisse nachfolgend mit der Enron-Zeitleiste abgeglichen. So k�nnen m�gliche Auswirkungen f�r die identifizierten Ausrei�er deklariert werden. \workTodo{EnronTimelineQuelle}

Die \autoref{tab:enrontime} bietet eine �bersicht der historischen Ereignisse, die die Ausrei�er des MIDAS-Algorithmus best�tigen. Im Vergleich zu \citep{SedanSpot} werden mehr Ausrei�er erkannt.
  
\begin{table}[h!]
	\centering
    \begin{tabular}{p{0.05\linewidth}|p{0.89\linewidth}}
	\toprule
	1. & Aktie erreicht Allzeithoch. Federal Energy Regulatory Commission ordnet Untersuchung an.\\
	\midrule
	2. & \textbullet Viertelj�hrliche Telefonkonferenz zur Finanzsituation und erste Symptome eines Problems. \newline \textbullet \enquote{Geheimes} Treffen -- Schwarzenegger, Lay, Milken. Angebot zur Rettung der Deregulierung. \\
	\midrule
	3. & \textbullet Skilling (CEO) k�ndigt. Mitarbeiterin warnt Lay (Gr�nder) vor Pleite. Skilling verkauft seine Aktien. \newline \textbullet Enron ver�ffentlicht 618 Mio. \$ Verlust. Interessenskonflikt wird untersucht und Akten vernichtet. \\
	\midrule
	4. & \textbullet Beginn der Strafermittlung. Lay's R�cktritt \newline \textbullet Internen Ermittlung verteilt die Schuld auf F�hrungskr�fte und den Vorstand \\
	\bottomrule
\end{tabular}
	\caption{�bersicht �ber historische Ereignisse, die den Ausrei�ern zuzuordnen sind}
	\label{tab:enrontime}
\end{table}


Ein weiterer Test erfolgte mit dem DARPA-Datensatz. \citep{DARPA} In \autoref{img:midas:darpa_anomaly} werden die Ausrei�er-Scores dargestellt.

\begin{figure}[H]
	\centering
	\includegraphics[height=0.6\linewidth,width=\linewidth]{fig/midas/Darpa_Anomaly}
	\caption{Der Ausrei�er-Score �ber die Zeit beim DARPA-Datensatz}
	\label{img:midas:darpa_anomaly}
\end{figure}

Bei der Anwendung des MIDAS auf dem DARPA-Datensatz sieht man klare einzelne Ausrei�er, die entdeckt wurden. F�r diesen Datensatz gibt es einen speziell f�r MIDAS entwickelten \textit{ground truth}, der die \textit{labels} f�r diesen Datensatz zur Verf�gung stellt.

Bei der Berechnung der AUC f�r die ermittelten Ausrei�er-Scores wird ein Wert von $0.91727$ berechnet. Das bedeutet, dass der MIDAS-Algorithmus mit einer Wahrscheinlichkeit von ca. 91,73\% die Kanten des Datensatzes richtig klassifiziert.  

Somit kann festgehalten werden, dass MIDAS hinsichtlich der Ausrei�ererkennung in Graphen, ein sehr guter Algorithmus ist und eine sehr hohe Genauigkeit erreicht.

\workTodo{@Marcus ? Willst du dazu noch was sagen? -> Wenn man die Anomalyscores als gewichte nimmt, kommen Graphen in Networkx raus in denen man die anomalous nodes identifizieren kann dabei sollten es Edges sein}


\subsection{Ausrei�ererkennung in Zeitreihen}
\label{sec:resultsOTs}

Um den MIDAS-Algorithmus auf Zeitreihen anwenden zu k�nnen, muss die Zeitreihe, wie in \autoref{chap:trsnsMidas} beschrieben, zun�chst in verschiedene Netzwerke umgewandelt werden. Bei den Tests konnte festgestellt werden, dass der MIDAS Algorithmus nicht dazu in der Lage ist Ausrei�er in Zeitreihen zu erkennen. Die vollst�ndigen Ergebnisse der Tests k�nnen in \autoref{chap:appendix_midas_ts} eingesehen werden. Hierbei ist jedoch der Verlauf des Ausrei�er-Scores schwierig zu interpretieren. Es ist zu erkennen, das der Ausrei�er-Score zu Beginn eines jeden Abschnitts sehr hoch ist, am Ende des Abschnitts ist der Ausrei�er Score hingegen relativ niedrig. Grund hierf�r ist, das die Anzahl an Kanten zu Beginn eines Abschnittes im Verh�ltnis zu der Anzahl an Kanten aus den vorangegangenen Abschnitten deutlich niedriger ist. Im weiteren Verlauf werden weitere Kanten innerhalb des Abschnitts hinzugef�gt. Dadurch gleicht sich die Anzahl an Kanten innerhalb der Abschnitte an und der Ausrei�er Score sinkt.

\begin{figure}[H]
	\centering
	\includegraphics[width=0.5\textwidth]{fig/resultsMidasTS/art_daily_jumpsup_MIDAS_hdrei_labeled_result.png}
	\caption{MIDAS Algorithmus angewandt auf Zeitreihe mit einer erh�hten Amplitude.}
	\label{img:midasTSresultJumpsup}
\end{figure}

Der MIDAS-Algorithmus ist lediglich bei einer Zeitreihe mit erh�hter Amplitude (vgl. \autoref{img:midasTSresultJumpsup}) in der Lage den Ausrei�er zu identifizieren. Durch den Ausschlag nach oben in der Zeitreihe entsteht ein Netzwerk, mit sehr hohen Gewichten. Die hohen Gewichte f�hren zu einer erh�hten Anzahl an Kanten, was schlussendlich zu einem Ausschlag des Ausrei�er Scores f�hrt. Die erh�hte Anzahl an Kanten f�hrt ebenfalls dazu das der Abschnitt mit dem Ausrei�er in der Abbildung deutlich breiter ist als die anderen. Bei anderen Ausrei�er Typen sind die Differenzen zwischen den verschiedenen Elementen der Zeitreihe nicht so gro�. Dadurch ergeben sich keinerlei hohe Kantengewichte und der Ausrei�er kann nicht erkannt werden.


\begin{figure}[H]
	\centering
	\subfloat[Zeitreihe mit Zyklus Aussetzter]{
		\includegraphics[width=0.5\textwidth]{fig/resultsMidasTS/art_daily_flatmiddle_MIDAS_hdrei_labeled_result.png}}
	\subfloat[Zeitreihe mit Frequenz�nderung]{
		\includegraphics[width=0.5\textwidth]{fig/resultsMidasTS/anomaly_art_daily_sequence_change_MIDAS_hdrei_labeled_result.png}}
	\caption{Ausrei�er Erkennung in Zeitreihen MIDAS Algorithmus}
	\label{img:midasTSresultsFlatSeqChange}
\end{figure}


Teilweise f�hren die Ausrei�er ebenso zu besonders niedrigen Kantengewichte (vgl. \autoref{img:midasTSresultsFlatSeqChange}). Bei diesem Ausrei�er Typ sind alle Werte auf der selben Ebene. Dadurch gehen die Kantengewichte gegen Null. Dies f�hrt zu einem sehr kurzen Abschnitt in der Abbildung (Der Abschnitt wurde mit einem Pfeil markiert). \workTodo{Noch Pfeil in Graphik einf�gen} Des weiteren ergibt sich durch die Ausrei�er eine leicht ver�nderte Anzahl an Kanten in dem Abschnitt mit dem Ausrei�er (vgl. \autoref{img:midasTSresultsFlatSeqChange}). Die Abweichungen sind jedoch so gering, dass es nicht zu einem starken Anstieg des Ausrei�er Scores f�hrt.
\label{sec:resultTSwithoutMidas}

\begin{figure}[H]
	\centering
	\subfloat[Zeitreihe mit einer Frequenz�nderung]{
		\includegraphics[width=0.5\textwidth]{fig/resultsMidasTS/anomaly_art_daily_sequence_change_MIDAS_hdrei_labeled_110_result.png}}	
	\subfloat[Zeitreihe mit erh�hter Amplitude]{
		\includegraphics[width=0.5\textwidth]{fig/resultsMidasTS/art_daily_jumpsup_MIDAS_hdrei_labeled_110_result.png}}
	\caption{Ausrei�er Erkennung Zeitreihen MIDAS Algorithmus Fenstergr��e 110}
	\label{img:midasTSresults110}
\end{figure}

Es wurden au�erdem Tests durchgef�hrt um zu untersuchen, wie sich der Algorithmus bei ver�nderter Fenstergr��e verh�lt (vgl. \autoref{img:midasTSresults110}). Bei den Untersuchungen in \autoref{img:midasTSresultJumpsup} und \autoref{img:midasTSresultsFlatSeqChange} wurde einer Fenstergr��e von 288 genutzt, was der Saisonalit�t der Zeitreihe entspricht. F�r dieses Experiment wurde einer Fenstergr��e von 110 verwendet. Es konnte festgestellt werden, das diese Ver�nderung keinen zus�tzlichen Nutzen erbringt. Allerdings ist der Ausschlag nach oben im Ausrei�er Score f�r die Zeitreihe mit erh�hter Amplitude noch deutlicher zu erkennen. Die anderen Ausrei�er Typen werden weiterhin nicht erkannt.


\begin{figure}[H]
	\centering
	\subfloat[Zeitreihe mit geringerer Amplitude]{
		\includegraphics[width=0.5\textwidth]{fig/reultsMidasR/art_daily_jumpsdown_MIDAS_R_red_est_result}}
	\subfloat[Zeitreihe mit erh�hter Amplitude]{
		\includegraphics[width=0.5\textwidth]{fig/reultsMidasR/art_daily_jumpsup_MIDAS_R_red_est_result}}
	\caption{Ausrei�er Erkennung Zeitreihen MIDAS-R}
	\label{img:midasRTSresults}
\end{figure}


In einem n�chsten Schritt wurde untersucht inwiefern der MIDAS-R Algorithmus zu einer Verbesserung bei der Ausrei�er Erkennung beitragen kann (vgl. \autoref{img:midasRTSresults}). Der MIDAS-R Algorithmus ber�cksichtigt bei der Berechnung des Ausrei�er Scores f�r den aktuellen Abschnitt ebenso die Daten aus der j�ngsten Vergangenheit(vorangegangene Abschnitte). Aus diesem Grund erhofften wir uns durch den Einsatz des MIDAS-R Algorithmus, dass die Ausschl�ge zu beginn eines jeden Abschnitts aus bleiben, sodass Ausrei�er deutlicher hervortreten. Es konnte festgestellt werden, dass der Ausschlag des Ausrei�er Scores zu Beginn der Abschnitte deutlich kleiner ist. Jedoch steigt der Ausrei�er Score zum Ende eines jeden Abschnitts wieder an. Es konnte somit keine Signifikante Verbesserung bei der Erkennung von Ausrei�ern erreicht werden. Insbesondere da der MIDAS-R Algorithmus ebenfalls nur den Ausrei�er in der Zeitreihe mit erh�hter Amplitude anzeigt. Somit konnte festgestellt werden, dass die durch den MIDAS-R Algorithmus eingef�hrten Features ebenso zu keiner Verbesserung der Ergebnisse gef�hrt haben. 
\workTodo{Vielleicht k�nnte eine Verbesserung erreicht werden wenn andere Features eingef�hrt werden w�rden.}





%\newpage
\section{NetSimile}
\label{sec:ns}
In \autoref{ssec:ns-gl} werden die Grundlagen des NetSimile-Algorithmus n�her erl�utert und in den Abschnitten \autoref{sec:ns-ext} und \autoref{sec:ns-ts-1} die Testergebnisse vorgestellt.
\subsection{Grundlagen}
\label{ssec:ns-gl}

NetSimile ist ein skalierbarer Algorithmus zur Erkennung von �hnlichkeiten, sowie Anomalien, in Netzwerken unterschiedlicher Gr��en. Hierf�r wird der Datensatz in gleich gro�e Zeitintervalle unterteilt, um die daraus resultierenden Graphen auf unterschiedliche Merkmale zu untersuchen. Die Merkmale sind hierbei strukturelle Eigenschaften der einzelnen Knoten wie bspw. die Dichte eines Knotens oder die Anzahl an Nachbarn in einem Ego-Netzwerk. Die Signatur ergibt sich aus den einzelnen Aggregationen der Knoten wie bspw. dem Median aus der Dichte der jeweiligen Knoten. So entsteht bspw. aus sieben Merkmalen und f�nf Aggregationen ein Signaturvektor mit 35 verschiedenen Signaturen. So erm�glicht der Signaturvektor die Beschreibung sowie den Vergleich der einzelnen Graphen. F�r den Vergleich wird die Canberra-Distanz aus den beiden Signaturvektoren zweier zeitlich nebeneinander liegenden Graphen berechnet. \citep[vgl.][S.~1]{Netsimile} Als Input f�r diesen Algorithmus wird eine Menge von $k$-anonymisierten Netzwerken mit beliebig unterschiedlichen Gr��en, die keine �berlappenden Knoten oder Kanten besitzen, herangezogen. Das Resultat sind Werte f�r die strukturelle �hnlichkeit oder Abstands eines jeden Paares der gegebenen Netzwerke bzw. ein Merkmalsvektor f�r jedes Netzwerk. \citep[vgl.][S.~1]{Netsimile} NetSimile durchl�uft drei Schritte, die im Folgenden erl�utert werden.
\subsubsection{Extrahierung von Merkmalen}
F�r jeden Knoten $i$ werden, basierend auf ihren Ego-Netzwerken, die folgenden Merkmale generiert:
\begin{table}[H]
	\centering
	\begin{tabular}{p{0.13\linewidth}|p{0.8\linewidth}}
		\toprule
		$\overline{d}_i = |N(i)|$ & Die Anzahl der Nachbarn (\dah Grad) von Knoten $i$, wobei $N(i)$ die Nachbarn von Knoten $i$ beschreibt.\\
		\midrule
		$\overline{c}_i$ & Der Clustering-Koeffizient von Knoten $i$, der als die Anzahl von Dreiecken, die mit Knoten $i$ verbunden sind, �ber die Anzahl von verbundenen Dreiecken, die auf Knoten $i$ zentriert sind, definiert ist. \\
		\midrule
		$d_{N(i)}$ & Die durchschnittliche Anzahl der Nachbarn von Knoten $i$, die zwei Schritte entfernt sind. Dieser wird berechnet als \workTodo{Paper Seite 2 unten Formel einf�gen} \\
		\midrule
		$c_{N(i)}$ & Der durchschnittliche Clustering-Koeffizient von $N(i)$, der als \workTodo{Paper Seite 2 unten Formel einf�gen} berechnet wird. \\
		\midrule
		$|E_{ego(i)}|$ &  Die Anzahl der Kanten im Ego-Netzwerk vom Knoten $i$, wobei $ego(i)$ das Ego-Netzwerk von $i$ zur�ckgibt. \\
		\midrule
		$|E^{\circ}_{ego(i)}|$ & Die Anzahl der von $ego(i)$ ausgehenden Kanten. \\
		\midrule
		$|N(ego(i))|$ & Die Anzahl von Nachbarn von $ego(i)$. \\
		\bottomrule
	\end{tabular}
	\caption{Inhalte des Merkmalsvektors}
	\label{tab:netfeat}
\end{table}

%\begin{mldescription}
%	\mlitem{$\overline{d}_i = |N(i)|$} Die Anzahl der Nachbarn (d.h. Grad) von Knoten $i$, wobei $N(i)$ die Nachbarn von Knoten $i$ beschreibt.
%	\mlitem{$\overline{c}_i$} Der Clustering-Koeffizient von Knoten $i$, der als die Anzahl von Dreiecken, die mit Knoten $i$ verbunden sind, �ber die Anzahl von verbundenen Dreiecken, die auf Knoten $i$ zentriert sind, definiert ist.
%	\mlitem{$d_{N(i)}$} Die durchschnittliche Anzahl der Nachbarn von Knoten $i$, die zwei Schritte entfernt sind. Dieser wird berechnet als \workTodo{Paper Seite 2 unten Formel einf�gen}
%	\mlitem{$c_{N(i)}$} Der durchschnittliche Clustering-Koeffizient von $N(i)$, der als \workTodo{Paper Seite 2 unten Formel einf�gen} berechnet wird.
%	\mlitem{$|E_{ego(i)}|$} Die Anzahl der Kanten im Ego-Netzwerk vom Knoten $i$, wobei $ego(i)$ das Ego-Netzwerk von $i$ zur�ckgibt.
%	\mlitem{$|E^{\circ}_{ego(i)}|$} Die Anzahl der von $ego(i)$ ausgehenden Kanten. 
%	\mlitem{$|N(ego(i))|$}  Die Anzahl von Nachbarn von $ego(i)$. 
%\end{mldescription}

\subsubsection{Aggregierung von Merkmalen}
Im n�chsten Schritt wird f�r jeden Graphen \textit{$G_j$} eine $Knoten \times Merkmal$-Matrix $F_{G_j}$ zusammengefasst. Dieser besteht aus den Merkmalsvektoren aus Schritt 1.
Da der Vergleich von $k$-ten $F_{G_j}$ sehr aufw�ndig ist, wird f�r jede $F_{G_j}$ ein Signaturvektor $\vec{s}_{G_j}$ ausgegeben. Dieser aggregiert den Median, den Mittelwert, die Standardabweichung, die Schiefe, sowie die Kurtosis der Merkmale aus der Matrix.  

\subsubsection{Vergleich der Signaturvektoren}
\label{sec:ns-gl-cd}

F�r die Ausrei�ererkennung werden die letzten drei Graphen anhand der Canberra-Distanz-Funktion, die als �hnlichkeitsma� dient, herangezogen. Steigt die Canberra-Distanz zwischen zwei Graphen oberhalb des Schwellwerts so wird dies im Algorithmus festgehalten. Falls der darauf folgende Graph ebenfalls oberhalb des Schwellwerts liegt, so wird dieser als Ausrei�er definiert. Dadurch wird die Anzahl der Ausrei�er reduziert, damit nur diejenigen identifiziert werden, bei denen ein Trend hin zu einem abnormalen Verhalten erkennbar ist.

Der Algorithmus arbeitet dabei dynamisch, da die Signaturen der Graphen in einzelne Teilberechnungen aufgeteilt und zwischengespeichert werden k�nnen, ohne das eine Neuberechnung notwendig ist. Der Schwellwert wird aus dem Median und dem Mittelwert berechnet, die ebenfalls zwischengespeichert und nach Bedarf um weitere Graphen erg�nzt werden k�nnen.

\FloatBarrier

\subsection{Anwendung auf Netzwerkdaten}
\label{sec:ns-ext}
Beim ersten Versuch den Algorithmus auf Netzwerkdaten anzuwenden, wurde die nachfolgende Problematik festgestellt.

Der Algorithmus verwendet eine Bibliothek \textit{igraph}, welche Kanten zwischen zwei Knoten nur einmalig hinzuf�gen kann. Beim Eliminieren der Duplikate wird aber ein Drittel des Datensatzes nicht ber�cksichtigt, wodurch wertvolle Informationen bei der Ausrei�ererkennung verloren gehen. Aus diesem Grund wurden die Netzwerkdaten soweit angepasst, dass Mehrfachverbindungen zwischen zwei Kanten aufsummiert werden und als Gewichtung dieser Kante hinzugef�gt wird. 

\begin{lstlisting}[language=Python, caption=Gewichtung als neues Feature, label=lst:netsimile:code1]
for i in range(len(e_list)): 
	g.add_edge(e_list[i][0], e_list[i][1], weight=e_list[i][2])
\end{lstlisting}

Dadurch kann der Datensatz zum einen vollst�ndig analysiert werden und zum anderen kann dadurch ein weiteres Feature hinzugef�gt werden, dass durch die f�nf verschiedenen Aggregationen den Signaturvektor um diese f�nf Werte erweitert.

Dadurch dass der Datensatz zuerst eingelesen und in einen Graphen transformiert wird und anschlie�end aus dem Graphen die jeweiligen Features extrahiert werden, verliert der Algorithmus au�erordentlich an Performanz. Des Weiteren wird im ersten Schritt der maximale Knoten-Wert als Gr��e des Graphens �bergeben. Wird bspw. f�r jeden Mitarbeiter eine eigene ID �bergeben und diese ID inkrementell erh�ht, so kann es sein, dass aus einem Netzwerk mit 20 verschiedenen Knoten ein Graph erzeugt wird, der 1000 Knoten erzeugt, weil eine ID mit dem Wert 1000 vorhanden ist. Dadurch b��t die Performanz an Geschwindigkeit ein, da Iterationen nicht �ber die 20 Knoten durchgef�hrt werden, sondern �ber 1000. Hierbei muss entweder der Datensatz vorab angepasst werden, indem die IDs neu vergeben werden oder der Algorithmus muss grundlegend neu aufgebaut werden. Dies w�re grundlegend m�glich, da der Algorithmus Graphen als Ausrei�er zur�ck gibt und keine Knoten. 

Da der Fokus auf der Anwendung von Zeitreihen liegt, werden Optimierungen erst im Abschnitt \autoref{sec:ns-ts-1} in Betracht gezogen. 

\subsubsection{Anwendung auf ENRON-Datensatz}
\label{sec:ns-enron}

Da der Enron Datensatz ebenfalls von einem anderen Paper analysiert und ver�ffentlicht wurde (vgl. \autoref{chap:ap-data}), k�nnen die dort erkannten Ausrei�er zum Vergleich in Form eines gelabelten Datensatz herangezogen werden.
Betrachtet man in diesem Kontext den Ausrei�erscore, ist gut zu erkennen, dass der Ausrei�er Ende 2001 als alleiniger herausstechender Ausrei�er ebenso im NetSimile wiederzufinden ist. Grundlegend ist ebenso zu erkennen, dass die Ausrei�er sich nur sehr wenig voneinander unterscheiden, wodurch sich eine Klassifizierung innerhalb des Ausrei�erscores als schwierig erweist. Die Extrahierung weiterer Features k�nnte dieses Problem l�sen, wobei dies nicht im Rahmen dieses Forschungsprojektes behandelt werden soll, da der Fokus auf Zeitreihen liegt. Der Datensatz innerhalb von zwei Minuten analysiert werden, womit der NetSimile-Algorithmus performant zu sein scheint.

\begin{figure}[H]
	\centering
	\includegraphics[height=0.6\linewidth,width=\linewidth]{fig/netsimile/anomalie_3}
	\caption{Ausrei�er-Score im Enron-Datensatz mit dem NetSimile-Algorithmus}
	\label{img:netsimile:anomalie_3}
\end{figure}


Betrachtet man die Differenz aus dem Durchschnitt der Signaturvektoren und den der Ausrei�ergraphen in einer \textit{Headmap} kann man erkennen, dass die Ausrei�er vorwiegend den besonders gro�en Ego-Netzwerken und einer hohen Anzahl an E-Mails verschuldet sind. Vergleicht man die Zeitleiste (\citep{EnronTimeline}) mit den Ausrei�erdaten, erkennt man einen starken Anstieg des Email-Verkehrs im Oktober 2001. Hier wurde von Enron ein Report ver�ffentlicht �ber einen Quartals-Verlust von 618 Millionen US-Dollar und einer Reduzierung des Eigenkapitals um 1.2 Milliarde. Im Dezember 2001 erkennt man einen verh�ltnism��ig geringen Email-Verkehr. Betrachtet man hier die Zeitleiste, so wurden 4000 Mitarbeiter in dieser Zeitspanne entlassen. \workTodo{Quelle die Webseite}



\begin{figure}[H]
	\centering
	\includegraphics[height=0.8\linewidth,width=\linewidth]{fig/netsimile/heatmap_3}
	\caption{Darstellung der Ausrei�er in Heatmaps}
	\label{img:netsimile:heatmap_3}
\end{figure}

\subsubsection{Anwendung auf Darpa-Datensatz}
\label{sec:ns-darpa}

Beim Darpa-Datensatz k�nnen die Aurei�er besser klassifiziert werden. Die Gr�nde hierf�r liegen auf der Gr��e und Vielfalt des Datensatzes. Der Enron.Datensatz hat eine Gr��e von 1MB und rund 50.000 Kanten. Der Darpa-Datensatz hingegen hat eine Gr��e von 50 MB mit 4.5 Mio Kanten. Die Berechnung hat dabei eine L�nge von drei Stunden. Haben wir bei der Dateigr��e den Faktor 50 und bei der Kantenanzahl den Faktor 90, so ist bei der Berechnungszeit der Faktor 90 wiederzufinden. Betrachtet man die Laufzeit, so kann eine lineare Abh�ngigkeit zwischen Kantenanzahl und der ben�tigten Berechnungszeit festgestellt werden.

\begin{figure}[H]
	\centering
	\includegraphics[height=0.6\linewidth,width=\linewidth]{fig/netsimile/anomalie_4}
	\caption{Ausrei�er-Score im Darpa-Datensatz mit dem NetSimile-Algorithmus}
	\label{img:netsimile:anomalie_4}
\end{figure}

\subsection{Anwendung auf Zeitreihen}
\label{sec:ns-ts-1}

Wird der Algorithmus auf Zeitreihen anwendet, entsteht folgendes Problem. Bei der Transformation der Daten entstehen vollst�ndige Graphen, wodurch die strukturellen Eigenschaften sowie die daraus resultierenden Merkmale identisch werden, wie in \autoref{img:netsimile:graph_11} deutlich wird. 

\usetikzlibrary{graphs,graphs.standard}
\begin{figure}[H]
	\centering
	\begin{tikzpicture}
		\graph[nodes={draw, circle,fill=black!20,minimum size = 6mm}, clockwise, empty nodes, radius=4cm, n=11] { subgraph K_n };
	\end{tikzpicture}
	\caption{Vollst�ndiger Graph mit 11 Knoten}
	\label{img:netsimile:graph_11}
\end{figure}

So hat bspw. das Feature $|E^{\circ}_{ego(i)}|$ keine Aussagekraft in einem vollst�ndigen Graphen, da jeder Knoten die gleiche Anzahl an Kanten in seinem Ego-Netzwerk aufweist. Subtrahiert man also vom durchschnittlichen Signaturvektor aller Graphen die einzelnen Signaturvektoren, so erkennt man den Wert 0 in allen Headmaps.

Somit m�ssen hierbei f�r vollst�ndige Graphen andere Features extrahiert werden. Au�erdem ist die Laufzeit in gro�en Datens�tzen, wie bspw. dem Darpa-Datensatz mit 3 Stunden Berechnungszeit nicht performant.

Aus diesem Grund werden aus dem NetSimile lediglich die Ans�tze der Merkmals-Extrahierung, die Distanzbildung zweier Signaturvektoren, sowie der Schwellwert f�r die Ausrei�eridentifizierung �bernommen. Das hei�t die Netzwerke der Zeitreihe werden nicht in ein Graphenobjekt umgewandelt, sondern als Adjazenzmatrix gespeichert. Dadurch k�nnen die Features deutlich effizienter berechnet werden. Zudem werden lediglich Merkmale verwendet, die f�r vollst�ndige Graphen geeignet sind. Dabei werden die nachfolgenden Merkmale neu eingef�hrt.

\workTodo{n-te Wurzel bei Formel f�r Geometrischen Mittelwert}
\begin{description}
	\item[$\sum_{i=1}^{n} x_i $]\hfill\\ Summe der Kantengewichte eines Knoten.
	\item[$\frac{1}{n}\sum_{i=1}^{n} x_i $]\hfill\\ Arithmetisches Mittel der Kantengewichte eines Knoten.
	\item[$ \sqrt{\prod\limits_{i = 1}^{n} x_i} $]\hfill\\ Geometrisches Mittel der Kantengewichte eines Knoten.
	\item[$
	x(p) =
	\begin{cases}
		\frac{1}{2}x_(np) +       & \quad \text{if } n \text{ is even}\\
		-(n+1)/2  & \quad \text{if } n \text{ is odd}
	\end{cases}
	$]\hfill\\ Geometrisches Mittel der Kanten mit den 10\% h�chsten Kantengewichten. 
	\item[$\frac{1}{n}\sum_{i=1}^{n} x_i$]\hfill\\ Geometrisches Mittel der Kanten mit den 20\% h�chsten Kantengewichten. 
\end{description}

Auf diesen Merkmalen wurden anschlie�end die bisherigen Aggregation durchgef�hrt. Dadurch konnten erste Ausrei�er in der Zeitreihe gefunden werden (vgl. \autoref{img:netsimile:anomalie_2}).

\begin{figure}[H]
	\centering
	\includegraphics[height=0.6\linewidth,width=\linewidth]{fig/netsimile/anomalie_2}
	\caption{Ausrei�er-Score der vollst�ndigen Graphen mit gewichteten Kanten}
	\label{img:netsimile:anomalie_2}
\end{figure}

Des Weiteren wurde ein neuer Parameter eingef�hrt. �ber diesen kann gesteuert werden zu wie vielen vorherigen Abschnitten die Distanz berechnet werden soll. Dadurch kann gesteuert werden wie schnell ein Algorithmus \workTodo{bitte umformulieren: vergisst}. Eine Auflistung der Parameter des Algorithmus ist in \autoref{table:parmeterNeti} zu sehen.

Um zu untersuchen, wie gut der Algorithmus funktioniert, wurde er auf Zeitreihen getestet. Als Testdaten   wurden, ein- und zweidimensionale Zeitreihen der Numenta-Gruppe verwendet. Diese Zeitreihen enthalten verschiedene Ausrei�ertypen, auf denen die Erkennung der Algorithmus getestet wurde. Die Qualit�t der Ausrei�ererkennung wurde mithilfe eines Punktesystem bewertet. In diesem k�nnen maximal vier Sterne erreicht werden, die daf�r stehen, dass Ausrei�er sehr gut erkannt werden. Null Sterne hingegen bedeuten, dass Ausrei�er �berhaupt nicht erkannt wurden. Die Parameter, welche f�r die Tests gew�hlt werden mussten, werden in \autoref{table:parmeterNeti} beschrieben.\\

\begin{table}[H]
	\centering
	\begin{tabular}{p{0.42\linewidth}|p{0.37\linewidth}|P{0.05\linewidth}|P{0.05\linewidth}}
		\toprule
		\textbf{Ausrei�er Typ}& \textbf{Datei Name}&
		\textbf{1D}&\textbf{2D}\\
		\midrule
		Einzelne Peaks & anomaly-art-daily-peaks & **& -\\
		\midrule
		Zunahme an Rauschen & anomaly-art-daily-increase-noise &****& ***\\
		\midrule
		Signal Drift & anomaly-art-daily-drift &**& -\\
		\midrule
		Kontinuierliche Zunahme der Amplitude& art-daily-amp-rise & ****& ***\\
		\midrule
		Zyklus mit h�herer Amplitude & art-daily-jumpsup &****& *\\
		\midrule
		Zyklus mit geringerer Amplitude & art-daily-jumpsdown & ****& -\\
		\midrule
		Zyklus-Aussetzer & art-daily-flatmiddle &****& ***\\
		\midrule
		Signal-Aussetzer & art-daily-nojump & ****& ***\\
		\midrule
		Frequenz�nderung & anomaly-art-daily-sequence-change &****& ***\\
		\bottomrule
	\end{tabular}
	\caption{NetSimile-Performance auf Zeitreihen}
	\label{table:performanceNeti}
\end{table}

\autoref{table:performanceNeti} zeigt die Ergebnisse der Tests. Es ist zu erkennen, dass die Qualit�t der Ausrei�ererkennung im eindimensionalen Fall sehr gut ist. Lediglich einzelne Peaks k�nnen durch den Algorithmus nicht als Ausrei�er identifiziert werden. Au�erdem wird bei \textit{Signal Drifts} und der kontinuierlichen Zunahme der Amplitude lediglich der Anfang des Ausrei�ers detektiert. Aus diesem Grund wurde eine Bewertung mit drei Sternen vergeben. Bei der Betrachtung der Graphiken in \autoref{sec:appendix_net_one} \workTodo{richtige Referenz einf�gen} und \autoref{sec:appendix_net_two} ist zu erkennen, dass das sechste oder siebte Intervall der Zeitreihe h�ufig als Ausrei�er markiert wird. Der Grund hierf�r ist, das bei einer Fenstergr��e von f�nf f�r die ersten f�nf Abschnitte kein Ausrei�er-Score berechnet wird. Dadurch ist die Standardabweichung zu Beginn sehr niedrig wodurch Abschnitte schnell als Ausrei�er gekennzeichnet werden. Dieser Umstand wurde bei der Bewertung in \autoref{table:performanceNeti} nicht ber�cksichtigt. Im zweidimensionalen Fall ist die Qualit�t der Ausrei�ererkennung etwas durchwachsener. Auffallend ist, dass Zyklen mit h�herer und niedriger Amplitude nicht als Ausrei�er erkannt werden. Insbesondere ist dies auff�llig, da diese Ausrei�ertypen �blicherweise zuverl�ssig erkannt werden (vgl. \autoref{sec:rw-gl}).  Au�erdem ist der Algorithmus im zweidimensionalen Fall nicht mehr dazu in der Lage \textit{Signal Drifts } zu erkennen. Andere Ausrei�ertypen k�nnen durch den Algorithmus weiterhin erkannt werden, jedoch oftmals nicht mit der selben Qualit�t. 

\begin{table}[H]
	\centering
	\begin{tabular}{p{0.13\linewidth}|p{0.815\linewidth}}
		\toprule
		\textbf{Parameter}& \textbf{Beschreibung}\\
		\midrule
		Periodizit�t & Wie in \autoref{sec:trsnsNeti} \workTodo{Referenz sollte glaube ich autoref -> sec:trsnsNeti sein} erl�utert muss die Zeitreihe in kleinere Intervalle aufgegliedert werden. �ber diesen Parameter wird die Gr��e der Intervalle gesteuert. F�r die Tests wurde der Parameter auf 288 gesetzt, da es sich hierbei um die Saisonalit�t der Zeitreihen handelt.\\
		\midrule
		Fenstergr��e & Wie in \autoref{sec:optiNeti} erkl�rt, bestimmt dieser Parameter die Anzahl der vorangegangenen Abschnitte zu welchen die Canberra-Distanz berechnet wird. Dieser Parameter wurde f�r die Tests auf 5 gesetzt.\\
		\midrule
		Abweichung & Legt fest ab wann es sich bei einem Abschnitt um einen Ausrei�er handelt. Der Parameter wurde f�r die Tests auf 3 gesetzt. Bedeutet wenn der Ausrei�er Score um das dreifache der Standardabweichung vom Durchschnitt abweicht, wird der Abschnitt als Ausrei�er gekennzeichnet.\\
		\bottomrule
	\end{tabular}
	\caption{Parameter des NetSimile f�r die Anwendung auf Zeitreihen}
	\label{table:parmeterNeti}
\end{table}


%\begin{table}[H]
%	\centering
%	\begin{tabular}{p{0.18\linewidth}|p{0.19\linewidth}|P{0.05\linewidth}|p{0.35\linewidth}|p{0.09\linewidth}}
%		\bottomrule
%		\textbf{Ausrei�er Typ}& \textbf{Datei Name}&
%		\textbf{1D}&\textbf{Beschreibung}&\textbf{Laufzeit}\\
%		\midrule
%		Einzelne Peaks & anomaly-art-daily-peaks & **& Zwei gemeinsame Peaks werden erkannt, Einzelne eher schlecht
%		&61min\\
%		\midrule
%		Zunahme an Rauschen & anomaly-art-daily-increase-noise &****& Ausrei�er wird erkannt&51\\
%		\midrule
%		Signal Drift & anomaly-art-daily-drift &**& Nur zwei von vier Ausrei�er werden erkannt
%		Die letzten zwei werden also normal definiert
%		&57min\\
%		\midrule
%		Kontinuierliche Zunahme der Amplitude& art-daily-amp-rise & ****& Ausrei�er werden erkannt
%		&54min\\
%		\midrule
%		Zyklus mit h�herer Amplitude & art-daily-jumpsup &****& Ausrei�er werden erkannt
%		&50\\
%		\midrule
%		Zyklus mit geringerer Amplitude & art-daily-jumpsdown & ****& Ausrei�er werden erkannt
%		&51min\\
%		\midrule
%		Zyklus-Aussetzer & art-daily-flatmiddle &****& Ausrei�er werden erkannt
%		&62min\\
%		\midrule
%		Signal-Aussetzer & art-daily-nojump & ****& Ausrei�er werden erkannt
%		&65min\\
%		\midrule
%		Frequenz�nderung & anomaly-art-daily-sequence-change &****& Ausrei�er werden erkannt
%		&49min\\
%		\bottomrule
%	\end{tabular}
%	\caption{Urspr�nglicher Netsimile Performance}
%	\label{table:performanceNeti}
%\end{table}







%\newpage
\section{MIDAS}
\label{sec:midas}
Im Folgenden wird der MIDAS-Algorithmus vorgestellt, sowie die Anwendung auf verschiedenen Datentypen n�her erl�utert.

\subsection{Grundlagen}
\label{sec:mc-gl}

MIDAS, Eng. \textit{Microcluster-Based Detector of Anomalies in Edge Streams}, steht f�r einen Algorithmus, der pl�tzlich auftretende Ausbr�che von Aktivit�ten in einem Netzwerk bzw. Graphen erkennt. Dieses vermehrte Auftreten von Aktivit�ten zeigt sich durch viele sich wiederholende Knoten- und Kantenpaare in einem sich zeitlich entwickelnden Graphen, die Mikrocluster bezeichnet werden. Mikrocluster bestehen demnach aus einem vermehrten Vorkommen eines einzigen Quell- und Zielpaares bzw. einer Kante $(u,v)$  \workTodo{Folgender Absatz kann vor der Beschreibung des Algorithmus eingef�gt werden, wie im Paper auch} Dies geschieht in Echtzeit, wobei jede Kante in konstanter Zeit und Speicher verarbeitet wird. In der Theorie garantiert er eine \textit{false-positive}-Wahrscheinlichkeit und ist durch einen 162 bis 644 mal schnelleren Ansatz, sowie einer 42\% bis 48\% h�here Genauigkeit, im Hinblick auf die AUC, \textit{\dq Area under the Curve\dq\space} sehr effektiv. \citep[vgl.][S.~1]{MIDAS}

Urspr�ngliche Anwendungsf�lle f�r MIDAS sind die Erkennung von Anomalien in Computer-Netzwerken, wie SPAM oder DoS-Angriffe, sowie Anomalien in Kreditkartentransaktionen.


\subsubsection{Count-Min-Sketch}
\label{sec:mc-gl-cms}

Damit die relevanten Informationen f�r den Algorithmus mit einem konstanten Speicher verarbeitet werden, wird Count-Min-Sketch genutzt, dass eine Streaming-Datenstruktur mithilfe der Nutzung von Hash-Funktionen entspricht. Count-Min-Sketch z�hlt somit die Frequenz einer Aktivit�t bei Streaming-Daten. Diese Datenstruktur hat ebenfalls den Vorteil, dass man zu Beginn keine Kenntnis �ber die Anzahl an Quell- und Zielpaaren haben muss. \citep{CMS04}

MIDAS verwendet zwei Arten von CMS. Die erste Variante $s_{uv}$ wird als die Anzahl an Kanten von $u$ zu $v$ bis zum aktuellen Zeitpunkt $t$ definiert. Durch die CMS-Datenstruktur werden alle Z�hlungen von $s_{uv}$ approximiert, sodass jederzeit eine ann�hernde Abfrage $\hat{s}_{uv}$ erhalten werden kann.
Die zweite Variante $a_{uv}$ wird als die Anzahl an Kanten von $u$ zu $v$ im aktuellen Zeitpunkt $t$ definiert. Dieser CMS ist identisch zu $s_{uv}$, wobei bei jedem �bergang zum n�chsten Zeitpunkt die Datenstruktur zur�ckgesetzt wird. Dadurch resultiert aus dem CMS f�r den aktuellen Zeitpunkt die ann�hernde Abfrage $\hat{a}_{uv}$. \citep[vgl.][S.~3]{MIDAS}


\subsubsection{Erkennung von Mikrocluster}
\label{sec:mc-gl-dm}

Mithilfe der N�herungswerte $\hat{s}_{uv}$ und $\hat{a}_{uv}$ ist das Detektieren von Mikroclustern m�glich. Hierzu wird der Mean (\dah die durchschnittliche Rate mit der Kanten erscheinen) betrachtet. Es wird hierbei angenommen, dass dieser f�r den aktuellen Zeitpunkt (\zB $t = 10$) �quivalent ist zu dem vor dem aktuellen Zeitpunkt ($t < 10$). Dadurch wird die Annahmen vermieden, dass die Daten auf einer bestimmten zugrundeliegenden Verteilung basieren oder Stationarit�t �ber die Zeit aufweisen.
\\
\\
Durch die genannte Annahme lassen sich vergangene Kanten in zwei Klassen einteilen. Eine f�r den aktuellen Zeitpunkt $t = 10$ und eine f�r alle vergangenen Zeitpunkte $t < 10$. Hierbei betr�gt die Anzahl der Ereignisse zum Zeitpunkt $t = 10$ $a_{uv}$ und die Anzahl der Kanten in vergangenen Zeitpunkten $t < 10$ ist $s_{uv} - a_{uv}$.
\\
\\
Die Auswertung der Daten kann mithilfe des chi-squared goodnes-of-fit test erfolgen. Hierbei wird die Summe der Klassen $t = 10$ und $t < 10$ f�r $\frac{(\text{beobachtet} - \text{erwartet})^2}{\text{erwartet}}$ bestimmt. Bei einer Gesamtanzahl von $s_{uv}$ Kanten ergibt sich, auf Basis eines Mean, f�r $t = 10$ eine erwartete Anzahl von $\frac{s_{uv}}{t}$ Kanten. Analog hierzu ergibt sich f�r $t < 10$ eine erwartete Anzahl an $\frac{t - 1}{t}s_{uv}$ vergangenen Kanten. Daraus ergibt sich f�r die chi-squared Statistik \cite[vgl.][S.~3]{MIDAS}:

\begin{align}
	\chi^2 &= \frac{\left(\text{beobachtet}_{(t = 10)} - \text{erwartet}_{(t = 10)}\right)^2}{\text{erwartet}_{(t = 10)}} \nonumber \\
	&+ \frac{\left(\text{beobachtet}_{(t < 10)} - \text{erwartet}_{(t < 10)}\right)^2}{\text{erwartet}_{(t < 10)}} \nonumber \\
	&= \frac{\left(a_{uv} - \frac{s_{uv}}{t}\right)^2}{\frac{s_{uv}}{t}} + \frac{\left(\left(s_{uv} - a_{uv}\right) - \frac{t - 1}{t}s_{uv}\right)^2}{\frac{t - 1}{t}s_{uv}} \nonumber \\
	&= \frac{\left(a_{uv} - \frac{s_{uv}}{t}\right)^2}{\frac{s_{uv}}{t}} + \frac{\left(a_{uv} - \frac{s_{uv}}{t}\right)^2}{\frac{t - 1}{t}s_{uv}} \nonumber \\
	&= \left(a_{uv} - \frac{s_{uv}}{t}\right)^2\frac{t^2}{s_{uv}(t - 1)}
	\label{eqn:midas:chi2}
\end{align}

Die Gr��en $a_{uv}$ und $s_{uv}$ k�nnen, mithilfe der CMS-Datenstruktur, approximiert werden. Daraus ergibt sich, unter Verwendung der approximierten Gr��en $\hat{a}_{uv}$ und $\hat{s}_{uv}$, der folgende Ausrei�er-Score \cite[vgl.][S.~4]{MIDAS}:

\begin{equation}
	score((u,v,t)) = \left(\hat{a}_{uv} - \frac{\hat{s}_{uv}}{t}\right)^2\frac{t^2}{\hat{s}_{uv}(t - 1)}
	\label{eqn:midas:score}
\end{equation}

Mithilfe des in \autoref{eqn:midas:score} angegeben Ausrei�er-Score l�sst sich eine neue Kante $(u,v)$ zum Zeitpunkt $t$ bewerten. Dieser wird in einem bin�ren Entscheidungsverfahren verwendet, um zu bestimmen, ob es sich bei einer neuen Kante um Anomalie handelt oder nicht. Die Wahrscheinlichkeit von false positive Ergebnissen soll hierbei nicht einen benutzerdefinierten Schwellenwert $\epsilon$ �bersteigen. CMS-Datenstrukturen mit einer angemessenen Gr��e besitzen die Eigenschaft, dass die Approximationen $\hat{a}_{uv}$, f�r beliebige $\epsilon$ und $\nu$, folgende Vorschrift mit einer Wahrscheinlichkeit von mindestens $1 - \frac{\epsilon}{2}$ erf�llen:

\begin{equation}
	\hat{a}_{uv} \leq a_{uv} + \nu \cdotp N_t
\end{equation}

$N_t$ beschreibt hierbei die Anzahl an Kanten zum Zeitpunkt $t$. Eine weitere Eigenschaft der CMS-Datenstrukturen ist, dass diese die tats�chlichen Anzahl an Kanten nur �berbewerten k�nnen:

\begin{equation}
	s_{uv} \leq \hat{s}_{uv}
\end{equation}

Der in \autoref{eqn:midas:score} gegebene Score kann wie folgt angepasst werden:

\begin{equation}
	\~{a}_{uv} = \hat{a}_{uv} - \nu N_t
\end{equation}

Daraus l�sst sich die in \autoref{eqn:midas:chi2} gegebene Statistik anpassen:

\begin{equation}
	\tilde{\chi}^2 = \left(\tilde{a}_{uv} - \frac{s_{uv}}{t}\right)^2\frac{t^2}{s_{uv}(t - 1)}
	\label{eqn:midas:chi2angepasst}
\end{equation}

Bei Verwendung der Teststatistik in \autoref{eqn:midas:chi2angepasst} und eines Schwellenwertes von $\chi^2_{1 - \frac{\epsilon}{2}}(1)$ ergibt sich eine Wahrscheinlichkeit f�r ein false positive Ergebnis von h�chstens $\epsilon$:

\begin{equation}
	P\left(\tilde{\chi}^2 > \chi^2_{1 - \frac{\epsilon}{2}}(1)\right) < \epsilon
	\label{eqn:midas:prob}
\end{equation}

Der Term $\chi^2_{1 - \frac{\epsilon}{2}}(1)$ beschreibt hierbei das $1 - \frac{\epsilon}{2}$-Quantil.

\subsubsection{Die Erweiterung zu MIDAS-R}
Bei dem MIDAS-R Algorithmus handelt es sich um eine Erweiterung des MIDAS Algorithmus. Das R steht hierbei f�r den Relationalen Ansatz des MIDAS-R Algorithmus. Dabei wird versucht die r�umliche oder zeitliche Verkn�pfung zwischen Kanten st�rker zu ber�cksichtigten. Es werden hierzu zwei neue Konzepte eingef�hrt \citep[vgl.][S.~4]{MIDAS}.

\textbf{Temporal Relations: }Durch diesen Ansatz soll der Algorithmus mehr zeitliche Flexibilit�t erhalten. Dabei sollen Kanten aus der j�ngsten Vergangenheit auch in einem neuen Zeitabschnitt ber�cksichtigt werden. Allerdings reduziert um eine bestimmte Gewichtung. Anstatt die CMS Datenstruktur nach jedem Zeitabschnitt zu reseten, werden die Gewichte hierbei um einen bestimmten Prozentsatz reduziert \citep[vgl.][S.~4]{MIDAS}.

\textbf{Spatial Relations: }Hierbei werden zwei neue Features eingef�hrt um verschiedene Ausrei�er-Typen identifizieren zu k�nnen. Die neuen Features werden hierbei in einer CMS Datenstrukturen gespeichert. Der Algorithmus speichert demzufolge diese drei Features:
\workTodo{Aufz�hlung vervollst�ndigen}

\begin{description}
	\item[$ $]\hfill\\ Anzahl an Kanten zwischen Knoten u un Knoten v. Dieses Feature wird auch vom MIDAS Algorithmus verwendet.
	\item[$ $]\hfill\\ Gesamtanzahl an Nachbarknoten eines Knoten u.
	\item[$  $]\hfill\\ Aktuelle Anzahl an Nachbarknoten eines Knoten u.
\end{description}

Aus diesen drei Features wird anschlie�end ein Ausrei�er Score abgeleitet.
 \citep[vgl.][S.~5]{MIDAS}


\subsection{Ausrei�ererkennung auf Netzwerkdaten}
\label{sec:m-ex}

Die Anwendung des MIDAS-Algorithmus auf Netzwerkdaten erfolgte problemlos. Es werden lediglich Daten ben�tigt, die in jeder Zeile aus einem Ausgangs-, einem Zielpunkt, sowie einem Zeitstempel bestehen. In welcher Form der Zeitstempel bereitgestellt wird, ist hierbei unwichtig. Nachfolgend werden die Ergebnisse mithilfe des MIDAS-Algorithmus auf Netzwerkdaten dargestellt.

\begin{figure}[H]
	\centering
	\includegraphics[height=0.6\linewidth,width=\linewidth]{fig/midas/Enron_Anomaly}
	\caption{Der Ausrei�er-Score �ber die Zeit beim ENRON-Datensatz}
	\label{img:midas:enron_anomaly}
\end{figure}


Vergleicht man die Ergebnisse aus \autoref{img:midas:enron_anomaly} mit den Ergebnissen aus \autoref{chap:ap-data} so kann erkannt werden, dass beide Algorithmen einen �hnlichen Verlauf vorweisen. Damit eine genauere Aussage getroffen werden kann, werden die MIDAS-Ergebnisse nachfolgend mit der Enron-Zeitleiste abgeglichen. So k�nnen m�gliche Auswirkungen f�r die identifizierten Ausrei�er deklariert werden. \workTodo{EnronTimelineQuelle}

Die \autoref{tab:enrontime} bietet eine �bersicht der historischen Ereignisse, die die Ausrei�er des MIDAS-Algorithmus best�tigen. Im Vergleich zu \citep{SedanSpot} werden mehr Ausrei�er erkannt.
  
\begin{table}[h!]
	\centering
    \begin{tabular}{p{0.05\linewidth}|p{0.89\linewidth}}
	\toprule
	1. & Aktie erreicht Allzeithoch. Federal Energy Regulatory Commission ordnet Untersuchung an.\\
	\midrule
	2. & \textbullet Viertelj�hrliche Telefonkonferenz zur Finanzsituation und erste Symptome eines Problems. \newline \textbullet \enquote{Geheimes} Treffen -- Schwarzenegger, Lay, Milken. Angebot zur Rettung der Deregulierung. \\
	\midrule
	3. & \textbullet Skilling (CEO) k�ndigt. Mitarbeiterin warnt Lay (Gr�nder) vor Pleite. Skilling verkauft seine Aktien. \newline \textbullet Enron ver�ffentlicht 618 Mio. \$ Verlust. Interessenskonflikt wird untersucht und Akten vernichtet. \\
	\midrule
	4. & \textbullet Beginn der Strafermittlung. Lay's R�cktritt \newline \textbullet Internen Ermittlung verteilt die Schuld auf F�hrungskr�fte und den Vorstand \\
	\bottomrule
\end{tabular}
	\caption{�bersicht �ber historische Ereignisse, die den Ausrei�ern zuzuordnen sind}
	\label{tab:enrontime}
\end{table}


Ein weiterer Test erfolgte mit dem DARPA-Datensatz. \citep{DARPA} In \autoref{img:midas:darpa_anomaly} werden die Ausrei�er-Scores dargestellt.

\begin{figure}[H]
	\centering
	\includegraphics[height=0.6\linewidth,width=\linewidth]{fig/midas/Darpa_Anomaly}
	\caption{Der Ausrei�er-Score �ber die Zeit beim DARPA-Datensatz}
	\label{img:midas:darpa_anomaly}
\end{figure}

Bei der Anwendung des MIDAS auf dem DARPA-Datensatz sieht man klare einzelne Ausrei�er, die entdeckt wurden. F�r diesen Datensatz gibt es einen speziell f�r MIDAS entwickelten \textit{ground truth}, der die \textit{labels} f�r diesen Datensatz zur Verf�gung stellt.

Bei der Berechnung der AUC f�r die ermittelten Ausrei�er-Scores wird ein Wert von $0.91727$ berechnet. Das bedeutet, dass der MIDAS-Algorithmus mit einer Wahrscheinlichkeit von ca. 91,73\% die Kanten des Datensatzes richtig klassifiziert.  

Somit kann festgehalten werden, dass MIDAS hinsichtlich der Ausrei�ererkennung in Graphen, ein sehr guter Algorithmus ist und eine sehr hohe Genauigkeit erreicht.

\workTodo{@Marcus ? Willst du dazu noch was sagen? -> Wenn man die Anomalyscores als gewichte nimmt, kommen Graphen in Networkx raus in denen man die anomalous nodes identifizieren kann dabei sollten es Edges sein}


\subsection{Ausrei�ererkennung in Zeitreihen}
\label{sec:resultsOTs}

Um den MIDAS-Algorithmus auf Zeitreihen anwenden zu k�nnen, muss die Zeitreihe, wie in \autoref{chap:trsnsMidas} beschrieben, zun�chst in verschiedene Netzwerke umgewandelt werden. Bei den Tests konnte festgestellt werden, dass der MIDAS Algorithmus nicht dazu in der Lage ist Ausrei�er in Zeitreihen zu erkennen. Die vollst�ndigen Ergebnisse der Tests k�nnen in \autoref{chap:appendix_midas_ts} eingesehen werden. Hierbei ist jedoch der Verlauf des Ausrei�er-Scores schwierig zu interpretieren. Es ist zu erkennen, das der Ausrei�er-Score zu Beginn eines jeden Abschnitts sehr hoch ist, am Ende des Abschnitts ist der Ausrei�er Score hingegen relativ niedrig. Grund hierf�r ist, das die Anzahl an Kanten zu Beginn eines Abschnittes im Verh�ltnis zu der Anzahl an Kanten aus den vorangegangenen Abschnitten deutlich niedriger ist. Im weiteren Verlauf werden weitere Kanten innerhalb des Abschnitts hinzugef�gt. Dadurch gleicht sich die Anzahl an Kanten innerhalb der Abschnitte an und der Ausrei�er Score sinkt.

\begin{figure}[H]
	\centering
	\includegraphics[width=0.5\textwidth]{fig/resultsMidasTS/art_daily_jumpsup_MIDAS_hdrei_labeled_result.png}
	\caption{MIDAS Algorithmus angewandt auf Zeitreihe mit einer erh�hten Amplitude.}
	\label{img:midasTSresultJumpsup}
\end{figure}

Der MIDAS-Algorithmus ist lediglich bei einer Zeitreihe mit erh�hter Amplitude (vgl. \autoref{img:midasTSresultJumpsup}) in der Lage den Ausrei�er zu identifizieren. Durch den Ausschlag nach oben in der Zeitreihe entsteht ein Netzwerk, mit sehr hohen Gewichten. Die hohen Gewichte f�hren zu einer erh�hten Anzahl an Kanten, was schlussendlich zu einem Ausschlag des Ausrei�er Scores f�hrt. Die erh�hte Anzahl an Kanten f�hrt ebenfalls dazu das der Abschnitt mit dem Ausrei�er in der Abbildung deutlich breiter ist als die anderen. Bei anderen Ausrei�er Typen sind die Differenzen zwischen den verschiedenen Elementen der Zeitreihe nicht so gro�. Dadurch ergeben sich keinerlei hohe Kantengewichte und der Ausrei�er kann nicht erkannt werden.


\begin{figure}[H]
	\centering
	\subfloat[Zeitreihe mit Zyklus Aussetzter]{
		\includegraphics[width=0.5\textwidth]{fig/resultsMidasTS/art_daily_flatmiddle_MIDAS_hdrei_labeled_result.png}}
	\subfloat[Zeitreihe mit Frequenz�nderung]{
		\includegraphics[width=0.5\textwidth]{fig/resultsMidasTS/anomaly_art_daily_sequence_change_MIDAS_hdrei_labeled_result.png}}
	\caption{Ausrei�er Erkennung in Zeitreihen MIDAS Algorithmus}
	\label{img:midasTSresultsFlatSeqChange}
\end{figure}


Teilweise f�hren die Ausrei�er ebenso zu besonders niedrigen Kantengewichte (vgl. \autoref{img:midasTSresultsFlatSeqChange}). Bei diesem Ausrei�er Typ sind alle Werte auf der selben Ebene. Dadurch gehen die Kantengewichte gegen Null. Dies f�hrt zu einem sehr kurzen Abschnitt in der Abbildung (Der Abschnitt wurde mit einem Pfeil markiert). \workTodo{Noch Pfeil in Graphik einf�gen} Des weiteren ergibt sich durch die Ausrei�er eine leicht ver�nderte Anzahl an Kanten in dem Abschnitt mit dem Ausrei�er (vgl. \autoref{img:midasTSresultsFlatSeqChange}). Die Abweichungen sind jedoch so gering, dass es nicht zu einem starken Anstieg des Ausrei�er Scores f�hrt.
\label{sec:resultTSwithoutMidas}

\begin{figure}[H]
	\centering
	\subfloat[Zeitreihe mit einer Frequenz�nderung]{
		\includegraphics[width=0.5\textwidth]{fig/resultsMidasTS/anomaly_art_daily_sequence_change_MIDAS_hdrei_labeled_110_result.png}}	
	\subfloat[Zeitreihe mit erh�hter Amplitude]{
		\includegraphics[width=0.5\textwidth]{fig/resultsMidasTS/art_daily_jumpsup_MIDAS_hdrei_labeled_110_result.png}}
	\caption{Ausrei�er Erkennung Zeitreihen MIDAS Algorithmus Fenstergr��e 110}
	\label{img:midasTSresults110}
\end{figure}

Es wurden au�erdem Tests durchgef�hrt um zu untersuchen, wie sich der Algorithmus bei ver�nderter Fenstergr��e verh�lt (vgl. \autoref{img:midasTSresults110}). Bei den Untersuchungen in \autoref{img:midasTSresultJumpsup} und \autoref{img:midasTSresultsFlatSeqChange} wurde einer Fenstergr��e von 288 genutzt, was der Saisonalit�t der Zeitreihe entspricht. F�r dieses Experiment wurde einer Fenstergr��e von 110 verwendet. Es konnte festgestellt werden, das diese Ver�nderung keinen zus�tzlichen Nutzen erbringt. Allerdings ist der Ausschlag nach oben im Ausrei�er Score f�r die Zeitreihe mit erh�hter Amplitude noch deutlicher zu erkennen. Die anderen Ausrei�er Typen werden weiterhin nicht erkannt.


\begin{figure}[H]
	\centering
	\subfloat[Zeitreihe mit geringerer Amplitude]{
		\includegraphics[width=0.5\textwidth]{fig/reultsMidasR/art_daily_jumpsdown_MIDAS_R_red_est_result}}
	\subfloat[Zeitreihe mit erh�hter Amplitude]{
		\includegraphics[width=0.5\textwidth]{fig/reultsMidasR/art_daily_jumpsup_MIDAS_R_red_est_result}}
	\caption{Ausrei�er Erkennung Zeitreihen MIDAS-R}
	\label{img:midasRTSresults}
\end{figure}


In einem n�chsten Schritt wurde untersucht inwiefern der MIDAS-R Algorithmus zu einer Verbesserung bei der Ausrei�er Erkennung beitragen kann (vgl. \autoref{img:midasRTSresults}). Der MIDAS-R Algorithmus ber�cksichtigt bei der Berechnung des Ausrei�er Scores f�r den aktuellen Abschnitt ebenso die Daten aus der j�ngsten Vergangenheit(vorangegangene Abschnitte). Aus diesem Grund erhofften wir uns durch den Einsatz des MIDAS-R Algorithmus, dass die Ausschl�ge zu beginn eines jeden Abschnitts aus bleiben, sodass Ausrei�er deutlicher hervortreten. Es konnte festgestellt werden, dass der Ausschlag des Ausrei�er Scores zu Beginn der Abschnitte deutlich kleiner ist. Jedoch steigt der Ausrei�er Score zum Ende eines jeden Abschnitts wieder an. Es konnte somit keine Signifikante Verbesserung bei der Erkennung von Ausrei�ern erreicht werden. Insbesondere da der MIDAS-R Algorithmus ebenfalls nur den Ausrei�er in der Zeitreihe mit erh�hter Amplitude anzeigt. Somit konnte festgestellt werden, dass die durch den MIDAS-R Algorithmus eingef�hrten Features ebenso zu keiner Verbesserung der Ergebnisse gef�hrt haben. 
\workTodo{Vielleicht k�nnte eine Verbesserung erreicht werden wenn andere Features eingef�hrt werden w�rden.}




%\newpage
\section{NetSimile}
\label{sec:ns}
In \autoref{ssec:ns-gl} werden die Grundlagen des NetSimile-Algorithmus n�her erl�utert und in den Abschnitten \autoref{sec:ns-ext} und \autoref{sec:ns-ts-1} die Testergebnisse vorgestellt.
\subsection{Grundlagen}
\label{ssec:ns-gl}

NetSimile ist ein skalierbarer Algorithmus zur Erkennung von �hnlichkeiten, sowie Anomalien, in Netzwerken unterschiedlicher Gr��en. Hierf�r wird der Datensatz in gleich gro�e Zeitintervalle unterteilt, um die daraus resultierenden Graphen auf unterschiedliche Merkmale zu untersuchen. Die Merkmale sind hierbei strukturelle Eigenschaften der einzelnen Knoten wie bspw. die Dichte eines Knotens oder die Anzahl an Nachbarn in einem Ego-Netzwerk. Die Signatur ergibt sich aus den einzelnen Aggregationen der Knoten wie bspw. dem Median aus der Dichte der jeweiligen Knoten. So entsteht bspw. aus sieben Merkmalen und f�nf Aggregationen ein Signaturvektor mit 35 verschiedenen Signaturen. So erm�glicht der Signaturvektor die Beschreibung sowie den Vergleich der einzelnen Graphen. F�r den Vergleich wird die Canberra-Distanz aus den beiden Signaturvektoren zweier zeitlich nebeneinander liegenden Graphen berechnet. \citep[vgl.][S.~1]{Netsimile} Als Input f�r diesen Algorithmus wird eine Menge von $k$-anonymisierten Netzwerken mit beliebig unterschiedlichen Gr��en, die keine �berlappenden Knoten oder Kanten besitzen, herangezogen. Das Resultat sind Werte f�r die strukturelle �hnlichkeit oder Abstands eines jeden Paares der gegebenen Netzwerke bzw. ein Merkmalsvektor f�r jedes Netzwerk. \citep[vgl.][S.~1]{Netsimile} NetSimile durchl�uft drei Schritte, die im Folgenden erl�utert werden.
\subsubsection{Extrahierung von Merkmalen}
F�r jeden Knoten $i$ werden, basierend auf ihren Ego-Netzwerken, die folgenden Merkmale generiert:
\begin{table}[H]
	\centering
	\begin{tabular}{p{0.13\linewidth}|p{0.8\linewidth}}
		\toprule
		$\overline{d}_i = |N(i)|$ & Die Anzahl der Nachbarn (\dah Grad) von Knoten $i$, wobei $N(i)$ die Nachbarn von Knoten $i$ beschreibt.\\
		\midrule
		$\overline{c}_i$ & Der Clustering-Koeffizient von Knoten $i$, der als die Anzahl von Dreiecken, die mit Knoten $i$ verbunden sind, �ber die Anzahl von verbundenen Dreiecken, die auf Knoten $i$ zentriert sind, definiert ist. \\
		\midrule
		$d_{N(i)}$ & Die durchschnittliche Anzahl der Nachbarn von Knoten $i$, die zwei Schritte entfernt sind. Dieser wird berechnet als \workTodo{Paper Seite 2 unten Formel einf�gen} \\
		\midrule
		$c_{N(i)}$ & Der durchschnittliche Clustering-Koeffizient von $N(i)$, der als \workTodo{Paper Seite 2 unten Formel einf�gen} berechnet wird. \\
		\midrule
		$|E_{ego(i)}|$ &  Die Anzahl der Kanten im Ego-Netzwerk vom Knoten $i$, wobei $ego(i)$ das Ego-Netzwerk von $i$ zur�ckgibt. \\
		\midrule
		$|E^{\circ}_{ego(i)}|$ & Die Anzahl der von $ego(i)$ ausgehenden Kanten. \\
		\midrule
		$|N(ego(i))|$ & Die Anzahl von Nachbarn von $ego(i)$. \\
		\bottomrule
	\end{tabular}
	\caption{Inhalte des Merkmalsvektors}
	\label{tab:netfeat}
\end{table}

%\begin{mldescription}
%	\mlitem{$\overline{d}_i = |N(i)|$} Die Anzahl der Nachbarn (d.h. Grad) von Knoten $i$, wobei $N(i)$ die Nachbarn von Knoten $i$ beschreibt.
%	\mlitem{$\overline{c}_i$} Der Clustering-Koeffizient von Knoten $i$, der als die Anzahl von Dreiecken, die mit Knoten $i$ verbunden sind, �ber die Anzahl von verbundenen Dreiecken, die auf Knoten $i$ zentriert sind, definiert ist.
%	\mlitem{$d_{N(i)}$} Die durchschnittliche Anzahl der Nachbarn von Knoten $i$, die zwei Schritte entfernt sind. Dieser wird berechnet als \workTodo{Paper Seite 2 unten Formel einf�gen}
%	\mlitem{$c_{N(i)}$} Der durchschnittliche Clustering-Koeffizient von $N(i)$, der als \workTodo{Paper Seite 2 unten Formel einf�gen} berechnet wird.
%	\mlitem{$|E_{ego(i)}|$} Die Anzahl der Kanten im Ego-Netzwerk vom Knoten $i$, wobei $ego(i)$ das Ego-Netzwerk von $i$ zur�ckgibt.
%	\mlitem{$|E^{\circ}_{ego(i)}|$} Die Anzahl der von $ego(i)$ ausgehenden Kanten. 
%	\mlitem{$|N(ego(i))|$}  Die Anzahl von Nachbarn von $ego(i)$. 
%\end{mldescription}

\subsubsection{Aggregierung von Merkmalen}
Im n�chsten Schritt wird f�r jeden Graphen \textit{$G_j$} eine $Knoten \times Merkmal$-Matrix $F_{G_j}$ zusammengefasst. Dieser besteht aus den Merkmalsvektoren aus Schritt 1.
Da der Vergleich von $k$-ten $F_{G_j}$ sehr aufw�ndig ist, wird f�r jede $F_{G_j}$ ein Signaturvektor $\vec{s}_{G_j}$ ausgegeben. Dieser aggregiert den Median, den Mittelwert, die Standardabweichung, die Schiefe, sowie die Kurtosis der Merkmale aus der Matrix.  

\subsubsection{Vergleich der Signaturvektoren}
\label{sec:ns-gl-cd}

F�r die Ausrei�ererkennung werden die letzten drei Graphen anhand der Canberra-Distanz-Funktion, die als �hnlichkeitsma� dient, herangezogen. Steigt die Canberra-Distanz zwischen zwei Graphen oberhalb des Schwellwerts so wird dies im Algorithmus festgehalten. Falls der darauf folgende Graph ebenfalls oberhalb des Schwellwerts liegt, so wird dieser als Ausrei�er definiert. Dadurch wird die Anzahl der Ausrei�er reduziert, damit nur diejenigen identifiziert werden, bei denen ein Trend hin zu einem abnormalen Verhalten erkennbar ist.

Der Algorithmus arbeitet dabei dynamisch, da die Signaturen der Graphen in einzelne Teilberechnungen aufgeteilt und zwischengespeichert werden k�nnen, ohne das eine Neuberechnung notwendig ist. Der Schwellwert wird aus dem Median und dem Mittelwert berechnet, die ebenfalls zwischengespeichert und nach Bedarf um weitere Graphen erg�nzt werden k�nnen.

\FloatBarrier

\subsection{Anwendung auf Netzwerkdaten}
\label{sec:ns-ext}
Beim ersten Versuch den Algorithmus auf Netzwerkdaten anzuwenden, wurde die nachfolgende Problematik festgestellt.

Der Algorithmus verwendet eine Bibliothek \textit{igraph}, welche Kanten zwischen zwei Knoten nur einmalig hinzuf�gen kann. Beim Eliminieren der Duplikate wird aber ein Drittel des Datensatzes nicht ber�cksichtigt, wodurch wertvolle Informationen bei der Ausrei�ererkennung verloren gehen. Aus diesem Grund wurden die Netzwerkdaten soweit angepasst, dass Mehrfachverbindungen zwischen zwei Kanten aufsummiert werden und als Gewichtung dieser Kante hinzugef�gt wird. 

\begin{lstlisting}[language=Python, caption=Gewichtung als neues Feature, label=lst:netsimile:code1]
for i in range(len(e_list)): 
	g.add_edge(e_list[i][0], e_list[i][1], weight=e_list[i][2])
\end{lstlisting}

Dadurch kann der Datensatz zum einen vollst�ndig analysiert werden und zum anderen kann dadurch ein weiteres Feature hinzugef�gt werden, dass durch die f�nf verschiedenen Aggregationen den Signaturvektor um diese f�nf Werte erweitert.

Dadurch dass der Datensatz zuerst eingelesen und in einen Graphen transformiert wird und anschlie�end aus dem Graphen die jeweiligen Features extrahiert werden, verliert der Algorithmus au�erordentlich an Performanz. Des Weiteren wird im ersten Schritt der maximale Knoten-Wert als Gr��e des Graphens �bergeben. Wird bspw. f�r jeden Mitarbeiter eine eigene ID �bergeben und diese ID inkrementell erh�ht, so kann es sein, dass aus einem Netzwerk mit 20 verschiedenen Knoten ein Graph erzeugt wird, der 1000 Knoten erzeugt, weil eine ID mit dem Wert 1000 vorhanden ist. Dadurch b��t die Performanz an Geschwindigkeit ein, da Iterationen nicht �ber die 20 Knoten durchgef�hrt werden, sondern �ber 1000. Hierbei muss entweder der Datensatz vorab angepasst werden, indem die IDs neu vergeben werden oder der Algorithmus muss grundlegend neu aufgebaut werden. Dies w�re grundlegend m�glich, da der Algorithmus Graphen als Ausrei�er zur�ck gibt und keine Knoten. 

Da der Fokus auf der Anwendung von Zeitreihen liegt, werden Optimierungen erst im Abschnitt \autoref{sec:ns-ts-1} in Betracht gezogen. 

\subsubsection{Anwendung auf ENRON-Datensatz}
\label{sec:ns-enron}

Da der Enron Datensatz ebenfalls von einem anderen Paper analysiert und ver�ffentlicht wurde (vgl. \autoref{chap:ap-data}), k�nnen die dort erkannten Ausrei�er zum Vergleich in Form eines gelabelten Datensatz herangezogen werden.
Betrachtet man in diesem Kontext den Ausrei�erscore, ist gut zu erkennen, dass der Ausrei�er Ende 2001 als alleiniger herausstechender Ausrei�er ebenso im NetSimile wiederzufinden ist. Grundlegend ist ebenso zu erkennen, dass die Ausrei�er sich nur sehr wenig voneinander unterscheiden, wodurch sich eine Klassifizierung innerhalb des Ausrei�erscores als schwierig erweist. Die Extrahierung weiterer Features k�nnte dieses Problem l�sen, wobei dies nicht im Rahmen dieses Forschungsprojektes behandelt werden soll, da der Fokus auf Zeitreihen liegt. Der Datensatz innerhalb von zwei Minuten analysiert werden, womit der NetSimile-Algorithmus performant zu sein scheint.

\begin{figure}[H]
	\centering
	\includegraphics[height=0.6\linewidth,width=\linewidth]{fig/netsimile/anomalie_3}
	\caption{Ausrei�er-Score im Enron-Datensatz mit dem NetSimile-Algorithmus}
	\label{img:netsimile:anomalie_3}
\end{figure}


Betrachtet man die Differenz aus dem Durchschnitt der Signaturvektoren und den der Ausrei�ergraphen in einer \textit{Headmap} kann man erkennen, dass die Ausrei�er vorwiegend den besonders gro�en Ego-Netzwerken und einer hohen Anzahl an E-Mails verschuldet sind. Vergleicht man die Zeitleiste (\citep{EnronTimeline}) mit den Ausrei�erdaten, erkennt man einen starken Anstieg des Email-Verkehrs im Oktober 2001. Hier wurde von Enron ein Report ver�ffentlicht �ber einen Quartals-Verlust von 618 Millionen US-Dollar und einer Reduzierung des Eigenkapitals um 1.2 Milliarde. Im Dezember 2001 erkennt man einen verh�ltnism��ig geringen Email-Verkehr. Betrachtet man hier die Zeitleiste, so wurden 4000 Mitarbeiter in dieser Zeitspanne entlassen. \workTodo{Quelle die Webseite}



\begin{figure}[H]
	\centering
	\includegraphics[height=0.8\linewidth,width=\linewidth]{fig/netsimile/heatmap_3}
	\caption{Darstellung der Ausrei�er in Heatmaps}
	\label{img:netsimile:heatmap_3}
\end{figure}

\subsubsection{Anwendung auf Darpa-Datensatz}
\label{sec:ns-darpa}

Beim Darpa-Datensatz k�nnen die Aurei�er besser klassifiziert werden. Die Gr�nde hierf�r liegen auf der Gr��e und Vielfalt des Datensatzes. Der Enron.Datensatz hat eine Gr��e von 1MB und rund 50.000 Kanten. Der Darpa-Datensatz hingegen hat eine Gr��e von 50 MB mit 4.5 Mio Kanten. Die Berechnung hat dabei eine L�nge von drei Stunden. Haben wir bei der Dateigr��e den Faktor 50 und bei der Kantenanzahl den Faktor 90, so ist bei der Berechnungszeit der Faktor 90 wiederzufinden. Betrachtet man die Laufzeit, so kann eine lineare Abh�ngigkeit zwischen Kantenanzahl und der ben�tigten Berechnungszeit festgestellt werden.

\begin{figure}[H]
	\centering
	\includegraphics[height=0.6\linewidth,width=\linewidth]{fig/netsimile/anomalie_4}
	\caption{Ausrei�er-Score im Darpa-Datensatz mit dem NetSimile-Algorithmus}
	\label{img:netsimile:anomalie_4}
\end{figure}

\subsection{Anwendung auf Zeitreihen}
\label{sec:ns-ts-1}

Wird der Algorithmus auf Zeitreihen anwendet, entsteht folgendes Problem. Bei der Transformation der Daten entstehen vollst�ndige Graphen, wodurch die strukturellen Eigenschaften sowie die daraus resultierenden Merkmale identisch werden, wie in \autoref{img:netsimile:graph_11} deutlich wird. 

\usetikzlibrary{graphs,graphs.standard}
\begin{figure}[H]
	\centering
	\begin{tikzpicture}
		\graph[nodes={draw, circle,fill=black!20,minimum size = 6mm}, clockwise, empty nodes, radius=4cm, n=11] { subgraph K_n };
	\end{tikzpicture}
	\caption{Vollst�ndiger Graph mit 11 Knoten}
	\label{img:netsimile:graph_11}
\end{figure}

So hat bspw. das Feature $|E^{\circ}_{ego(i)}|$ keine Aussagekraft in einem vollst�ndigen Graphen, da jeder Knoten die gleiche Anzahl an Kanten in seinem Ego-Netzwerk aufweist. Subtrahiert man also vom durchschnittlichen Signaturvektor aller Graphen die einzelnen Signaturvektoren, so erkennt man den Wert 0 in allen Headmaps.

Somit m�ssen hierbei f�r vollst�ndige Graphen andere Features extrahiert werden. Au�erdem ist die Laufzeit in gro�en Datens�tzen, wie bspw. dem Darpa-Datensatz mit 3 Stunden Berechnungszeit nicht performant.

Aus diesem Grund werden aus dem NetSimile lediglich die Ans�tze der Merkmals-Extrahierung, die Distanzbildung zweier Signaturvektoren, sowie der Schwellwert f�r die Ausrei�eridentifizierung �bernommen. Das hei�t die Netzwerke der Zeitreihe werden nicht in ein Graphenobjekt umgewandelt, sondern als Adjazenzmatrix gespeichert. Dadurch k�nnen die Features deutlich effizienter berechnet werden. Zudem werden lediglich Merkmale verwendet, die f�r vollst�ndige Graphen geeignet sind. Dabei werden die nachfolgenden Merkmale neu eingef�hrt.

\workTodo{n-te Wurzel bei Formel f�r Geometrischen Mittelwert}
\begin{description}
	\item[$\sum_{i=1}^{n} x_i $]\hfill\\ Summe der Kantengewichte eines Knoten.
	\item[$\frac{1}{n}\sum_{i=1}^{n} x_i $]\hfill\\ Arithmetisches Mittel der Kantengewichte eines Knoten.
	\item[$ \sqrt{\prod\limits_{i = 1}^{n} x_i} $]\hfill\\ Geometrisches Mittel der Kantengewichte eines Knoten.
	\item[$
	x(p) =
	\begin{cases}
		\frac{1}{2}x_(np) +       & \quad \text{if } n \text{ is even}\\
		-(n+1)/2  & \quad \text{if } n \text{ is odd}
	\end{cases}
	$]\hfill\\ Geometrisches Mittel der Kanten mit den 10\% h�chsten Kantengewichten. 
	\item[$\frac{1}{n}\sum_{i=1}^{n} x_i$]\hfill\\ Geometrisches Mittel der Kanten mit den 20\% h�chsten Kantengewichten. 
\end{description}

Auf diesen Merkmalen wurden anschlie�end die bisherigen Aggregation durchgef�hrt. Dadurch konnten erste Ausrei�er in der Zeitreihe gefunden werden (vgl. \autoref{img:netsimile:anomalie_2}).

\begin{figure}[H]
	\centering
	\includegraphics[height=0.6\linewidth,width=\linewidth]{fig/netsimile/anomalie_2}
	\caption{Ausrei�er-Score der vollst�ndigen Graphen mit gewichteten Kanten}
	\label{img:netsimile:anomalie_2}
\end{figure}

Des Weiteren wurde ein neuer Parameter eingef�hrt. �ber diesen kann gesteuert werden zu wie vielen vorherigen Abschnitten die Distanz berechnet werden soll. Dadurch kann gesteuert werden wie schnell ein Algorithmus \workTodo{bitte umformulieren: vergisst}. Eine Auflistung der Parameter des Algorithmus ist in \autoref{table:parmeterNeti} zu sehen.

Um zu untersuchen, wie gut der Algorithmus funktioniert, wurde er auf Zeitreihen getestet. Als Testdaten   wurden, ein- und zweidimensionale Zeitreihen der Numenta-Gruppe verwendet. Diese Zeitreihen enthalten verschiedene Ausrei�ertypen, auf denen die Erkennung der Algorithmus getestet wurde. Die Qualit�t der Ausrei�ererkennung wurde mithilfe eines Punktesystem bewertet. In diesem k�nnen maximal vier Sterne erreicht werden, die daf�r stehen, dass Ausrei�er sehr gut erkannt werden. Null Sterne hingegen bedeuten, dass Ausrei�er �berhaupt nicht erkannt wurden. Die Parameter, welche f�r die Tests gew�hlt werden mussten, werden in \autoref{table:parmeterNeti} beschrieben.\\

\begin{table}[H]
	\centering
	\begin{tabular}{p{0.42\linewidth}|p{0.37\linewidth}|P{0.05\linewidth}|P{0.05\linewidth}}
		\toprule
		\textbf{Ausrei�er Typ}& \textbf{Datei Name}&
		\textbf{1D}&\textbf{2D}\\
		\midrule
		Einzelne Peaks & anomaly-art-daily-peaks & **& -\\
		\midrule
		Zunahme an Rauschen & anomaly-art-daily-increase-noise &****& ***\\
		\midrule
		Signal Drift & anomaly-art-daily-drift &**& -\\
		\midrule
		Kontinuierliche Zunahme der Amplitude& art-daily-amp-rise & ****& ***\\
		\midrule
		Zyklus mit h�herer Amplitude & art-daily-jumpsup &****& *\\
		\midrule
		Zyklus mit geringerer Amplitude & art-daily-jumpsdown & ****& -\\
		\midrule
		Zyklus-Aussetzer & art-daily-flatmiddle &****& ***\\
		\midrule
		Signal-Aussetzer & art-daily-nojump & ****& ***\\
		\midrule
		Frequenz�nderung & anomaly-art-daily-sequence-change &****& ***\\
		\bottomrule
	\end{tabular}
	\caption{NetSimile-Performance auf Zeitreihen}
	\label{table:performanceNeti}
\end{table}

\autoref{table:performanceNeti} zeigt die Ergebnisse der Tests. Es ist zu erkennen, dass die Qualit�t der Ausrei�ererkennung im eindimensionalen Fall sehr gut ist. Lediglich einzelne Peaks k�nnen durch den Algorithmus nicht als Ausrei�er identifiziert werden. Au�erdem wird bei \textit{Signal Drifts} und der kontinuierlichen Zunahme der Amplitude lediglich der Anfang des Ausrei�ers detektiert. Aus diesem Grund wurde eine Bewertung mit drei Sternen vergeben. Bei der Betrachtung der Graphiken in \autoref{sec:appendix_net_one} \workTodo{richtige Referenz einf�gen} und \autoref{sec:appendix_net_two} ist zu erkennen, dass das sechste oder siebte Intervall der Zeitreihe h�ufig als Ausrei�er markiert wird. Der Grund hierf�r ist, das bei einer Fenstergr��e von f�nf f�r die ersten f�nf Abschnitte kein Ausrei�er-Score berechnet wird. Dadurch ist die Standardabweichung zu Beginn sehr niedrig wodurch Abschnitte schnell als Ausrei�er gekennzeichnet werden. Dieser Umstand wurde bei der Bewertung in \autoref{table:performanceNeti} nicht ber�cksichtigt. Im zweidimensionalen Fall ist die Qualit�t der Ausrei�ererkennung etwas durchwachsener. Auffallend ist, dass Zyklen mit h�herer und niedriger Amplitude nicht als Ausrei�er erkannt werden. Insbesondere ist dies auff�llig, da diese Ausrei�ertypen �blicherweise zuverl�ssig erkannt werden (vgl. \autoref{sec:rw-gl}).  Au�erdem ist der Algorithmus im zweidimensionalen Fall nicht mehr dazu in der Lage \textit{Signal Drifts } zu erkennen. Andere Ausrei�ertypen k�nnen durch den Algorithmus weiterhin erkannt werden, jedoch oftmals nicht mit der selben Qualit�t. 

\begin{table}[H]
	\centering
	\begin{tabular}{p{0.13\linewidth}|p{0.815\linewidth}}
		\toprule
		\textbf{Parameter}& \textbf{Beschreibung}\\
		\midrule
		Periodizit�t & Wie in \autoref{sec:trsnsNeti} \workTodo{Referenz sollte glaube ich autoref -> sec:trsnsNeti sein} erl�utert muss die Zeitreihe in kleinere Intervalle aufgegliedert werden. �ber diesen Parameter wird die Gr��e der Intervalle gesteuert. F�r die Tests wurde der Parameter auf 288 gesetzt, da es sich hierbei um die Saisonalit�t der Zeitreihen handelt.\\
		\midrule
		Fenstergr��e & Wie in \autoref{sec:optiNeti} erkl�rt, bestimmt dieser Parameter die Anzahl der vorangegangenen Abschnitte zu welchen die Canberra-Distanz berechnet wird. Dieser Parameter wurde f�r die Tests auf 5 gesetzt.\\
		\midrule
		Abweichung & Legt fest ab wann es sich bei einem Abschnitt um einen Ausrei�er handelt. Der Parameter wurde f�r die Tests auf 3 gesetzt. Bedeutet wenn der Ausrei�er Score um das dreifache der Standardabweichung vom Durchschnitt abweicht, wird der Abschnitt als Ausrei�er gekennzeichnet.\\
		\bottomrule
	\end{tabular}
	\caption{Parameter des NetSimile f�r die Anwendung auf Zeitreihen}
	\label{table:parmeterNeti}
\end{table}


%\begin{table}[H]
%	\centering
%	\begin{tabular}{p{0.18\linewidth}|p{0.19\linewidth}|P{0.05\linewidth}|p{0.35\linewidth}|p{0.09\linewidth}}
%		\bottomrule
%		\textbf{Ausrei�er Typ}& \textbf{Datei Name}&
%		\textbf{1D}&\textbf{Beschreibung}&\textbf{Laufzeit}\\
%		\midrule
%		Einzelne Peaks & anomaly-art-daily-peaks & **& Zwei gemeinsame Peaks werden erkannt, Einzelne eher schlecht
%		&61min\\
%		\midrule
%		Zunahme an Rauschen & anomaly-art-daily-increase-noise &****& Ausrei�er wird erkannt&51\\
%		\midrule
%		Signal Drift & anomaly-art-daily-drift &**& Nur zwei von vier Ausrei�er werden erkannt
%		Die letzten zwei werden also normal definiert
%		&57min\\
%		\midrule
%		Kontinuierliche Zunahme der Amplitude& art-daily-amp-rise & ****& Ausrei�er werden erkannt
%		&54min\\
%		\midrule
%		Zyklus mit h�herer Amplitude & art-daily-jumpsup &****& Ausrei�er werden erkannt
%		&50\\
%		\midrule
%		Zyklus mit geringerer Amplitude & art-daily-jumpsdown & ****& Ausrei�er werden erkannt
%		&51min\\
%		\midrule
%		Zyklus-Aussetzer & art-daily-flatmiddle &****& Ausrei�er werden erkannt
%		&62min\\
%		\midrule
%		Signal-Aussetzer & art-daily-nojump & ****& Ausrei�er werden erkannt
%		&65min\\
%		\midrule
%		Frequenz�nderung & anomaly-art-daily-sequence-change &****& Ausrei�er werden erkannt
%		&49min\\
%		\bottomrule
%	\end{tabular}
%	\caption{Urspr�nglicher Netsimile Performance}
%	\label{table:performanceNeti}
%\end{table}







%\newpage
\section{MIDAS}
\label{sec:midas}
Im Folgenden wird der MIDAS-Algorithmus vorgestellt, sowie die Anwendung auf verschiedenen Datentypen n�her erl�utert.

\subsection{Grundlagen}
\label{sec:mc-gl}

MIDAS, Eng. \textit{Microcluster-Based Detector of Anomalies in Edge Streams}, steht f�r einen Algorithmus, der pl�tzlich auftretende Ausbr�che von Aktivit�ten in einem Netzwerk bzw. Graphen erkennt. Dieses vermehrte Auftreten von Aktivit�ten zeigt sich durch viele sich wiederholende Knoten- und Kantenpaare in einem sich zeitlich entwickelnden Graphen, die Mikrocluster bezeichnet werden. Mikrocluster bestehen demnach aus einem vermehrten Vorkommen eines einzigen Quell- und Zielpaares bzw. einer Kante $(u,v)$  \workTodo{Folgender Absatz kann vor der Beschreibung des Algorithmus eingef�gt werden, wie im Paper auch} Dies geschieht in Echtzeit, wobei jede Kante in konstanter Zeit und Speicher verarbeitet wird. In der Theorie garantiert er eine \textit{false-positive}-Wahrscheinlichkeit und ist durch einen 162 bis 644 mal schnelleren Ansatz, sowie einer 42\% bis 48\% h�here Genauigkeit, im Hinblick auf die AUC, \textit{\dq Area under the Curve\dq\space} sehr effektiv. \citep[vgl.][S.~1]{MIDAS}

Urspr�ngliche Anwendungsf�lle f�r MIDAS sind die Erkennung von Anomalien in Computer-Netzwerken, wie SPAM oder DoS-Angriffe, sowie Anomalien in Kreditkartentransaktionen.


\subsubsection{Count-Min-Sketch}
\label{sec:mc-gl-cms}

Damit die relevanten Informationen f�r den Algorithmus mit einem konstanten Speicher verarbeitet werden, wird Count-Min-Sketch genutzt, dass eine Streaming-Datenstruktur mithilfe der Nutzung von Hash-Funktionen entspricht. Count-Min-Sketch z�hlt somit die Frequenz einer Aktivit�t bei Streaming-Daten. Diese Datenstruktur hat ebenfalls den Vorteil, dass man zu Beginn keine Kenntnis �ber die Anzahl an Quell- und Zielpaaren haben muss. \citep{CMS04}

MIDAS verwendet zwei Arten von CMS. Die erste Variante $s_{uv}$ wird als die Anzahl an Kanten von $u$ zu $v$ bis zum aktuellen Zeitpunkt $t$ definiert. Durch die CMS-Datenstruktur werden alle Z�hlungen von $s_{uv}$ approximiert, sodass jederzeit eine ann�hernde Abfrage $\hat{s}_{uv}$ erhalten werden kann.
Die zweite Variante $a_{uv}$ wird als die Anzahl an Kanten von $u$ zu $v$ im aktuellen Zeitpunkt $t$ definiert. Dieser CMS ist identisch zu $s_{uv}$, wobei bei jedem �bergang zum n�chsten Zeitpunkt die Datenstruktur zur�ckgesetzt wird. Dadurch resultiert aus dem CMS f�r den aktuellen Zeitpunkt die ann�hernde Abfrage $\hat{a}_{uv}$. \citep[vgl.][S.~3]{MIDAS}


\subsubsection{Erkennung von Mikrocluster}
\label{sec:mc-gl-dm}

Mithilfe der N�herungswerte $\hat{s}_{uv}$ und $\hat{a}_{uv}$ ist das Detektieren von Mikroclustern m�glich. Hierzu wird der Mean (\dah die durchschnittliche Rate mit der Kanten erscheinen) betrachtet. Es wird hierbei angenommen, dass dieser f�r den aktuellen Zeitpunkt (\zB $t = 10$) �quivalent ist zu dem vor dem aktuellen Zeitpunkt ($t < 10$). Dadurch wird die Annahmen vermieden, dass die Daten auf einer bestimmten zugrundeliegenden Verteilung basieren oder Stationarit�t �ber die Zeit aufweisen.
\\
\\
Durch die genannte Annahme lassen sich vergangene Kanten in zwei Klassen einteilen. Eine f�r den aktuellen Zeitpunkt $t = 10$ und eine f�r alle vergangenen Zeitpunkte $t < 10$. Hierbei betr�gt die Anzahl der Ereignisse zum Zeitpunkt $t = 10$ $a_{uv}$ und die Anzahl der Kanten in vergangenen Zeitpunkten $t < 10$ ist $s_{uv} - a_{uv}$.
\\
\\
Die Auswertung der Daten kann mithilfe des chi-squared goodnes-of-fit test erfolgen. Hierbei wird die Summe der Klassen $t = 10$ und $t < 10$ f�r $\frac{(\text{beobachtet} - \text{erwartet})^2}{\text{erwartet}}$ bestimmt. Bei einer Gesamtanzahl von $s_{uv}$ Kanten ergibt sich, auf Basis eines Mean, f�r $t = 10$ eine erwartete Anzahl von $\frac{s_{uv}}{t}$ Kanten. Analog hierzu ergibt sich f�r $t < 10$ eine erwartete Anzahl an $\frac{t - 1}{t}s_{uv}$ vergangenen Kanten. Daraus ergibt sich f�r die chi-squared Statistik \cite[vgl.][S.~3]{MIDAS}:

\begin{align}
	\chi^2 &= \frac{\left(\text{beobachtet}_{(t = 10)} - \text{erwartet}_{(t = 10)}\right)^2}{\text{erwartet}_{(t = 10)}} \nonumber \\
	&+ \frac{\left(\text{beobachtet}_{(t < 10)} - \text{erwartet}_{(t < 10)}\right)^2}{\text{erwartet}_{(t < 10)}} \nonumber \\
	&= \frac{\left(a_{uv} - \frac{s_{uv}}{t}\right)^2}{\frac{s_{uv}}{t}} + \frac{\left(\left(s_{uv} - a_{uv}\right) - \frac{t - 1}{t}s_{uv}\right)^2}{\frac{t - 1}{t}s_{uv}} \nonumber \\
	&= \frac{\left(a_{uv} - \frac{s_{uv}}{t}\right)^2}{\frac{s_{uv}}{t}} + \frac{\left(a_{uv} - \frac{s_{uv}}{t}\right)^2}{\frac{t - 1}{t}s_{uv}} \nonumber \\
	&= \left(a_{uv} - \frac{s_{uv}}{t}\right)^2\frac{t^2}{s_{uv}(t - 1)}
	\label{eqn:midas:chi2}
\end{align}

Die Gr��en $a_{uv}$ und $s_{uv}$ k�nnen, mithilfe der CMS-Datenstruktur, approximiert werden. Daraus ergibt sich, unter Verwendung der approximierten Gr��en $\hat{a}_{uv}$ und $\hat{s}_{uv}$, der folgende Ausrei�er-Score \cite[vgl.][S.~4]{MIDAS}:

\begin{equation}
	score((u,v,t)) = \left(\hat{a}_{uv} - \frac{\hat{s}_{uv}}{t}\right)^2\frac{t^2}{\hat{s}_{uv}(t - 1)}
	\label{eqn:midas:score}
\end{equation}

Mithilfe des in \autoref{eqn:midas:score} angegeben Ausrei�er-Score l�sst sich eine neue Kante $(u,v)$ zum Zeitpunkt $t$ bewerten. Dieser wird in einem bin�ren Entscheidungsverfahren verwendet, um zu bestimmen, ob es sich bei einer neuen Kante um Anomalie handelt oder nicht. Die Wahrscheinlichkeit von false positive Ergebnissen soll hierbei nicht einen benutzerdefinierten Schwellenwert $\epsilon$ �bersteigen. CMS-Datenstrukturen mit einer angemessenen Gr��e besitzen die Eigenschaft, dass die Approximationen $\hat{a}_{uv}$, f�r beliebige $\epsilon$ und $\nu$, folgende Vorschrift mit einer Wahrscheinlichkeit von mindestens $1 - \frac{\epsilon}{2}$ erf�llen:

\begin{equation}
	\hat{a}_{uv} \leq a_{uv} + \nu \cdotp N_t
\end{equation}

$N_t$ beschreibt hierbei die Anzahl an Kanten zum Zeitpunkt $t$. Eine weitere Eigenschaft der CMS-Datenstrukturen ist, dass diese die tats�chlichen Anzahl an Kanten nur �berbewerten k�nnen:

\begin{equation}
	s_{uv} \leq \hat{s}_{uv}
\end{equation}

Der in \autoref{eqn:midas:score} gegebene Score kann wie folgt angepasst werden:

\begin{equation}
	\~{a}_{uv} = \hat{a}_{uv} - \nu N_t
\end{equation}

Daraus l�sst sich die in \autoref{eqn:midas:chi2} gegebene Statistik anpassen:

\begin{equation}
	\tilde{\chi}^2 = \left(\tilde{a}_{uv} - \frac{s_{uv}}{t}\right)^2\frac{t^2}{s_{uv}(t - 1)}
	\label{eqn:midas:chi2angepasst}
\end{equation}

Bei Verwendung der Teststatistik in \autoref{eqn:midas:chi2angepasst} und eines Schwellenwertes von $\chi^2_{1 - \frac{\epsilon}{2}}(1)$ ergibt sich eine Wahrscheinlichkeit f�r ein false positive Ergebnis von h�chstens $\epsilon$:

\begin{equation}
	P\left(\tilde{\chi}^2 > \chi^2_{1 - \frac{\epsilon}{2}}(1)\right) < \epsilon
	\label{eqn:midas:prob}
\end{equation}

Der Term $\chi^2_{1 - \frac{\epsilon}{2}}(1)$ beschreibt hierbei das $1 - \frac{\epsilon}{2}$-Quantil.

\subsubsection{Die Erweiterung zu MIDAS-R}
Bei dem MIDAS-R Algorithmus handelt es sich um eine Erweiterung des MIDAS Algorithmus. Das R steht hierbei f�r den Relationalen Ansatz des MIDAS-R Algorithmus. Dabei wird versucht die r�umliche oder zeitliche Verkn�pfung zwischen Kanten st�rker zu ber�cksichtigten. Es werden hierzu zwei neue Konzepte eingef�hrt \citep[vgl.][S.~4]{MIDAS}.

\textbf{Temporal Relations: }Durch diesen Ansatz soll der Algorithmus mehr zeitliche Flexibilit�t erhalten. Dabei sollen Kanten aus der j�ngsten Vergangenheit auch in einem neuen Zeitabschnitt ber�cksichtigt werden. Allerdings reduziert um eine bestimmte Gewichtung. Anstatt die CMS Datenstruktur nach jedem Zeitabschnitt zu reseten, werden die Gewichte hierbei um einen bestimmten Prozentsatz reduziert \citep[vgl.][S.~4]{MIDAS}.

\textbf{Spatial Relations: }Hierbei werden zwei neue Features eingef�hrt um verschiedene Ausrei�er-Typen identifizieren zu k�nnen. Die neuen Features werden hierbei in einer CMS Datenstrukturen gespeichert. Der Algorithmus speichert demzufolge diese drei Features:
\workTodo{Aufz�hlung vervollst�ndigen}

\begin{description}
	\item[$ $]\hfill\\ Anzahl an Kanten zwischen Knoten u un Knoten v. Dieses Feature wird auch vom MIDAS Algorithmus verwendet.
	\item[$ $]\hfill\\ Gesamtanzahl an Nachbarknoten eines Knoten u.
	\item[$  $]\hfill\\ Aktuelle Anzahl an Nachbarknoten eines Knoten u.
\end{description}

Aus diesen drei Features wird anschlie�end ein Ausrei�er Score abgeleitet.
 \citep[vgl.][S.~5]{MIDAS}


\subsection{Ausrei�ererkennung auf Netzwerkdaten}
\label{sec:m-ex}

Die Anwendung des MIDAS-Algorithmus auf Netzwerkdaten erfolgte problemlos. Es werden lediglich Daten ben�tigt, die in jeder Zeile aus einem Ausgangs-, einem Zielpunkt, sowie einem Zeitstempel bestehen. In welcher Form der Zeitstempel bereitgestellt wird, ist hierbei unwichtig. Nachfolgend werden die Ergebnisse mithilfe des MIDAS-Algorithmus auf Netzwerkdaten dargestellt.

\begin{figure}[H]
	\centering
	\includegraphics[height=0.6\linewidth,width=\linewidth]{fig/midas/Enron_Anomaly}
	\caption{Der Ausrei�er-Score �ber die Zeit beim ENRON-Datensatz}
	\label{img:midas:enron_anomaly}
\end{figure}


Vergleicht man die Ergebnisse aus \autoref{img:midas:enron_anomaly} mit den Ergebnissen aus \autoref{chap:ap-data} so kann erkannt werden, dass beide Algorithmen einen �hnlichen Verlauf vorweisen. Damit eine genauere Aussage getroffen werden kann, werden die MIDAS-Ergebnisse nachfolgend mit der Enron-Zeitleiste abgeglichen. So k�nnen m�gliche Auswirkungen f�r die identifizierten Ausrei�er deklariert werden. \workTodo{EnronTimelineQuelle}

Die \autoref{tab:enrontime} bietet eine �bersicht der historischen Ereignisse, die die Ausrei�er des MIDAS-Algorithmus best�tigen. Im Vergleich zu \citep{SedanSpot} werden mehr Ausrei�er erkannt.
  
\begin{table}[h!]
	\centering
    \begin{tabular}{p{0.05\linewidth}|p{0.89\linewidth}}
	\toprule
	1. & Aktie erreicht Allzeithoch. Federal Energy Regulatory Commission ordnet Untersuchung an.\\
	\midrule
	2. & \textbullet Viertelj�hrliche Telefonkonferenz zur Finanzsituation und erste Symptome eines Problems. \newline \textbullet \enquote{Geheimes} Treffen -- Schwarzenegger, Lay, Milken. Angebot zur Rettung der Deregulierung. \\
	\midrule
	3. & \textbullet Skilling (CEO) k�ndigt. Mitarbeiterin warnt Lay (Gr�nder) vor Pleite. Skilling verkauft seine Aktien. \newline \textbullet Enron ver�ffentlicht 618 Mio. \$ Verlust. Interessenskonflikt wird untersucht und Akten vernichtet. \\
	\midrule
	4. & \textbullet Beginn der Strafermittlung. Lay's R�cktritt \newline \textbullet Internen Ermittlung verteilt die Schuld auf F�hrungskr�fte und den Vorstand \\
	\bottomrule
\end{tabular}
	\caption{�bersicht �ber historische Ereignisse, die den Ausrei�ern zuzuordnen sind}
	\label{tab:enrontime}
\end{table}


Ein weiterer Test erfolgte mit dem DARPA-Datensatz. \citep{DARPA} In \autoref{img:midas:darpa_anomaly} werden die Ausrei�er-Scores dargestellt.

\begin{figure}[H]
	\centering
	\includegraphics[height=0.6\linewidth,width=\linewidth]{fig/midas/Darpa_Anomaly}
	\caption{Der Ausrei�er-Score �ber die Zeit beim DARPA-Datensatz}
	\label{img:midas:darpa_anomaly}
\end{figure}

Bei der Anwendung des MIDAS auf dem DARPA-Datensatz sieht man klare einzelne Ausrei�er, die entdeckt wurden. F�r diesen Datensatz gibt es einen speziell f�r MIDAS entwickelten \textit{ground truth}, der die \textit{labels} f�r diesen Datensatz zur Verf�gung stellt.

Bei der Berechnung der AUC f�r die ermittelten Ausrei�er-Scores wird ein Wert von $0.91727$ berechnet. Das bedeutet, dass der MIDAS-Algorithmus mit einer Wahrscheinlichkeit von ca. 91,73\% die Kanten des Datensatzes richtig klassifiziert.  

Somit kann festgehalten werden, dass MIDAS hinsichtlich der Ausrei�ererkennung in Graphen, ein sehr guter Algorithmus ist und eine sehr hohe Genauigkeit erreicht.

\workTodo{@Marcus ? Willst du dazu noch was sagen? -> Wenn man die Anomalyscores als gewichte nimmt, kommen Graphen in Networkx raus in denen man die anomalous nodes identifizieren kann dabei sollten es Edges sein}


\subsection{Ausrei�ererkennung in Zeitreihen}
\label{sec:resultsOTs}

Um den MIDAS-Algorithmus auf Zeitreihen anwenden zu k�nnen, muss die Zeitreihe, wie in \autoref{chap:trsnsMidas} beschrieben, zun�chst in verschiedene Netzwerke umgewandelt werden. Bei den Tests konnte festgestellt werden, dass der MIDAS Algorithmus nicht dazu in der Lage ist Ausrei�er in Zeitreihen zu erkennen. Die vollst�ndigen Ergebnisse der Tests k�nnen in \autoref{chap:appendix_midas_ts} eingesehen werden. Hierbei ist jedoch der Verlauf des Ausrei�er-Scores schwierig zu interpretieren. Es ist zu erkennen, das der Ausrei�er-Score zu Beginn eines jeden Abschnitts sehr hoch ist, am Ende des Abschnitts ist der Ausrei�er Score hingegen relativ niedrig. Grund hierf�r ist, das die Anzahl an Kanten zu Beginn eines Abschnittes im Verh�ltnis zu der Anzahl an Kanten aus den vorangegangenen Abschnitten deutlich niedriger ist. Im weiteren Verlauf werden weitere Kanten innerhalb des Abschnitts hinzugef�gt. Dadurch gleicht sich die Anzahl an Kanten innerhalb der Abschnitte an und der Ausrei�er Score sinkt.

\begin{figure}[H]
	\centering
	\includegraphics[width=0.5\textwidth]{fig/resultsMidasTS/art_daily_jumpsup_MIDAS_hdrei_labeled_result.png}
	\caption{MIDAS Algorithmus angewandt auf Zeitreihe mit einer erh�hten Amplitude.}
	\label{img:midasTSresultJumpsup}
\end{figure}

Der MIDAS-Algorithmus ist lediglich bei einer Zeitreihe mit erh�hter Amplitude (vgl. \autoref{img:midasTSresultJumpsup}) in der Lage den Ausrei�er zu identifizieren. Durch den Ausschlag nach oben in der Zeitreihe entsteht ein Netzwerk, mit sehr hohen Gewichten. Die hohen Gewichte f�hren zu einer erh�hten Anzahl an Kanten, was schlussendlich zu einem Ausschlag des Ausrei�er Scores f�hrt. Die erh�hte Anzahl an Kanten f�hrt ebenfalls dazu das der Abschnitt mit dem Ausrei�er in der Abbildung deutlich breiter ist als die anderen. Bei anderen Ausrei�er Typen sind die Differenzen zwischen den verschiedenen Elementen der Zeitreihe nicht so gro�. Dadurch ergeben sich keinerlei hohe Kantengewichte und der Ausrei�er kann nicht erkannt werden.


\begin{figure}[H]
	\centering
	\subfloat[Zeitreihe mit Zyklus Aussetzter]{
		\includegraphics[width=0.5\textwidth]{fig/resultsMidasTS/art_daily_flatmiddle_MIDAS_hdrei_labeled_result.png}}
	\subfloat[Zeitreihe mit Frequenz�nderung]{
		\includegraphics[width=0.5\textwidth]{fig/resultsMidasTS/anomaly_art_daily_sequence_change_MIDAS_hdrei_labeled_result.png}}
	\caption{Ausrei�er Erkennung in Zeitreihen MIDAS Algorithmus}
	\label{img:midasTSresultsFlatSeqChange}
\end{figure}


Teilweise f�hren die Ausrei�er ebenso zu besonders niedrigen Kantengewichte (vgl. \autoref{img:midasTSresultsFlatSeqChange}). Bei diesem Ausrei�er Typ sind alle Werte auf der selben Ebene. Dadurch gehen die Kantengewichte gegen Null. Dies f�hrt zu einem sehr kurzen Abschnitt in der Abbildung (Der Abschnitt wurde mit einem Pfeil markiert). \workTodo{Noch Pfeil in Graphik einf�gen} Des weiteren ergibt sich durch die Ausrei�er eine leicht ver�nderte Anzahl an Kanten in dem Abschnitt mit dem Ausrei�er (vgl. \autoref{img:midasTSresultsFlatSeqChange}). Die Abweichungen sind jedoch so gering, dass es nicht zu einem starken Anstieg des Ausrei�er Scores f�hrt.
\label{sec:resultTSwithoutMidas}

\begin{figure}[H]
	\centering
	\subfloat[Zeitreihe mit einer Frequenz�nderung]{
		\includegraphics[width=0.5\textwidth]{fig/resultsMidasTS/anomaly_art_daily_sequence_change_MIDAS_hdrei_labeled_110_result.png}}	
	\subfloat[Zeitreihe mit erh�hter Amplitude]{
		\includegraphics[width=0.5\textwidth]{fig/resultsMidasTS/art_daily_jumpsup_MIDAS_hdrei_labeled_110_result.png}}
	\caption{Ausrei�er Erkennung Zeitreihen MIDAS Algorithmus Fenstergr��e 110}
	\label{img:midasTSresults110}
\end{figure}

Es wurden au�erdem Tests durchgef�hrt um zu untersuchen, wie sich der Algorithmus bei ver�nderter Fenstergr��e verh�lt (vgl. \autoref{img:midasTSresults110}). Bei den Untersuchungen in \autoref{img:midasTSresultJumpsup} und \autoref{img:midasTSresultsFlatSeqChange} wurde einer Fenstergr��e von 288 genutzt, was der Saisonalit�t der Zeitreihe entspricht. F�r dieses Experiment wurde einer Fenstergr��e von 110 verwendet. Es konnte festgestellt werden, das diese Ver�nderung keinen zus�tzlichen Nutzen erbringt. Allerdings ist der Ausschlag nach oben im Ausrei�er Score f�r die Zeitreihe mit erh�hter Amplitude noch deutlicher zu erkennen. Die anderen Ausrei�er Typen werden weiterhin nicht erkannt.


\begin{figure}[H]
	\centering
	\subfloat[Zeitreihe mit geringerer Amplitude]{
		\includegraphics[width=0.5\textwidth]{fig/reultsMidasR/art_daily_jumpsdown_MIDAS_R_red_est_result}}
	\subfloat[Zeitreihe mit erh�hter Amplitude]{
		\includegraphics[width=0.5\textwidth]{fig/reultsMidasR/art_daily_jumpsup_MIDAS_R_red_est_result}}
	\caption{Ausrei�er Erkennung Zeitreihen MIDAS-R}
	\label{img:midasRTSresults}
\end{figure}


In einem n�chsten Schritt wurde untersucht inwiefern der MIDAS-R Algorithmus zu einer Verbesserung bei der Ausrei�er Erkennung beitragen kann (vgl. \autoref{img:midasRTSresults}). Der MIDAS-R Algorithmus ber�cksichtigt bei der Berechnung des Ausrei�er Scores f�r den aktuellen Abschnitt ebenso die Daten aus der j�ngsten Vergangenheit(vorangegangene Abschnitte). Aus diesem Grund erhofften wir uns durch den Einsatz des MIDAS-R Algorithmus, dass die Ausschl�ge zu beginn eines jeden Abschnitts aus bleiben, sodass Ausrei�er deutlicher hervortreten. Es konnte festgestellt werden, dass der Ausschlag des Ausrei�er Scores zu Beginn der Abschnitte deutlich kleiner ist. Jedoch steigt der Ausrei�er Score zum Ende eines jeden Abschnitts wieder an. Es konnte somit keine Signifikante Verbesserung bei der Erkennung von Ausrei�ern erreicht werden. Insbesondere da der MIDAS-R Algorithmus ebenfalls nur den Ausrei�er in der Zeitreihe mit erh�hter Amplitude anzeigt. Somit konnte festgestellt werden, dass die durch den MIDAS-R Algorithmus eingef�hrten Features ebenso zu keiner Verbesserung der Ergebnisse gef�hrt haben. 
\workTodo{Vielleicht k�nnte eine Verbesserung erreicht werden wenn andere Features eingef�hrt werden w�rden.}




%\chapter{Fazit und Ausblick}
\label{chap:fua}

\section{Fazit}
\label{sec:fua-f}

Das dritte Teilziel ist es die verschiedenen graphen-basierten Algorithmen, die wir im Rahmen des fortgef�hrten Forschungsprojekt analysiert und evaluiert haben miteinander zu vergleichen. (vgl. \autoref{sec:einl-ps}) Durch den Vergleich soll ermittelt werden, wie erfolgreich ein Algorithmus bei der Erkennung von Ausrei�ern in Daten ist. 


\begin{table}[H]
	\centering
	\begin{tabular}{p{0.2\linewidth}|P{0.1\linewidth}|P{0.13\linewidth}|P{0.13\linewidth}|P{0.13\linewidth}|P{0.14\linewidth}}
		\toprule
		& \textbf{Statisch} & \textbf{Dynamisch} & \textbf{Qualit�t Ausrei�er-Erkennung Netzwerke} & \textbf{Qualit�t Ausrei�er-Erkennung Zeitreihen} & \textbf{Performanz} \\
		\midrule
		IsoMap-based & \textbf{\textcolor{green}{+}} & \textbf{\textcolor{red}{-}} & \textbf{\textcolor{red}{-}} & \textbf{o}  & \textbf{\textcolor{green}{+}} \\
		\midrule
		Perculation-based & \textbf{\textcolor{green}{+}} & \textbf{\textcolor{red}{-}} & \textbf{\textcolor{red}{-}} & \textbf{\textcolor{green}{+}} & \textbf{\textcolor{green}{+}} \\
		\midrule
		Netsimile & \textbf{\textcolor{red}{-}} & \textbf{\textcolor{green}{+}} & \textbf{\textcolor{green}{+}} & \textbf{\textcolor{green}{++}} & \textbf{\textcolor{green}{+}} \\
		\midrule
		MIDAS & \textbf{\textcolor{red}{-}} & \textbf{\textcolor{green}{+}} & \textbf{\textcolor{green}{++}} & \textbf{\textcolor{red}{-}} & \textbf{o}  \\
		\bottomrule
	\end{tabular}
	\caption{Vergleich der Algorithmen}
	\label{tab:bewertung}
\end{table}


Aus der Tabelle \autoref{tab:bewertung} kann man die Kriterien entnehmen, die zum Vergleich der Algorithmen herangezogen wurden. So werden alle Algorithmen zun�chst einmal in statisch und dynamisch entsprechend der Taxonomie des Forschungsprojekts unterteilt. Im n�chsten Schritt werden die Ergebnisse aller Algorithmen hinsichtlich ihrer F�higkeit Ausrei�er in Graphen, sowie in Zeitreihen qualitativ zu erkennen gegen�bergestellt. Zuletzt liegt der Fokus auf ihrer Performance im Vergleich zu den anderen Algorithmen. Nachfolgend werden die Ergebnisse ausgef�hrt.
\\ \\
\workTodo{die Vergleiche autoref einf�gen.}
Der IsoMap-based und der Perculation-based Algorithmus sind statische Algorithmen, die jeweils nur auf einem Graphen angewendet werden k�nnen. Aus diesem Grund eignen sich diese nicht f�r die Ausrei�er-Erkennung in Netzwerken, da hier mehrere Graphen �bergeben werden. Beide k�nnen jedoch erfolgreich auf Zeitreihen angewendet werden, wobei beim IsoMap-based Algorithmus die Ausrei�er nur teilweise erkennbar sind, da sich die Ausrei�er nur geringf�gig von den anderen Werten unterscheiden. Der Perculation-based Algorithmus hingegen zeigt eindeutige Ausrei�er, weshalb dieser Algorithmus geeignet ist f�r eine qualitative Ausrei�er-Erkennung. Bei den Ausrei�er Typen \dq Signal-Aussetzer\dq\space und \dq Frequenz�nderung\dq\space k�nnen beide Algorithmen keine Ausrei�er identifizieren.  Die Performance beider Algorithmen ist gut, da sie die Berechnungen innerhalb weniger Sekunden durchf�hren.

Der Netsimile- und der MIDAS-Algorithmus k�nnen beide auf Netzwerkdatens�tzen angewendet werden, da sie dynamisch sind. Der Unterschied hierbei liegt bei dem Ausrei�er-Score. Beim Netsimile sind die erkannten Ausrei�er nahezu identisch, wodurch eine Klassifizierung unzureichend ist. Beim MIDAS hingegen sind die Ausschl�ge weitaus gr��er, wodurch eine Klassifizierung erzielt wird. Bei der Anwendung auf Zeitreihen liefert der Netsimile sehr gute Ergebnisse auf den Ausrei�er-Typen die wir verwendet haben (vgl. \autoref{sec:ns-ts}). Da sich bei der Transformation von Zeitreihen zu Graphen die Gewichtung nur geringf�gig unterscheidet und der MIDAS-Algorithmus Ausrei�er nur bei einer hohen Kantengewichtung erkennt, fallen die Ergebnisse hier schlechter aus (vgl. \autoref{sec:resultsOTs}). Dementsprechend ist MIDAS f�r die Erkennung von Ausrei�ern in Zeitreihen eher ungeeignet. Bei der Performanz hat der Netsimile-Algorithmus urspr�nglich bis zu einer Stunde ben�tigt, da dieser  rechenintensive Bibliotheken verwendete. Durch die Verwendung anderer Bibliotheken wurde eine Optimierung erreicht. Die Visualisierung des Graphens ist hierdurch zwar nicht mehr m�glich, jedoch wird die Rechenzeit auf wenige Sekunden reduziert. Der MIDAS ben�tigt hingegen mehrere Minuten, wodurch der Netsimile nach der Optimierung performanter geworden ist. (VGL: PERFORMANZ-TABELLE dann im Anhang)

Der Netsimile-Algorithmus ist demnach ein Algorithmus, der, nach Optimierung, den Anforderungen des Forschungsprojekts gerecht wird. Dieser Algorithmus ist dynamisch, multidimensional, performant und die Erkennung von Ausrei�ern in Zeitreihen ist sehr gut. Der Schwerpunkt hierbei liegt bei der Wahl der richtigen Features. Strukturelle Merkmale sind eher ungeeignet bei vollst�ndig verkn�pften Graphen und sollten ersetzt werden durch Features, wie bspw. die Gewichtung der Kanten. 

\workTodo{Tabelle in den Anhang}
\begin{table}[h!]
	\centering
	\begin{tabular}{P{0.3\linewidth}|P{0.31\linewidth}|P{0.31\linewidth}}
		\toprule
		& \textbf{Netsimile} & \textbf{MIDAS} \\
		\midrule
		AmpRise & 54 min 32 sec & 05 min 26 sec \\
		\midrule
		Drift & 57 min 29 sec & 05 min 22 sec \\
		\midrule
		Flatmiddle & 62 min 50 sec & 06 min 01 sec \\
		\midrule
		Increase Noise & 51 min 56 sec & 05 min 40 sec \\
		\midrule
		Jumps Down & 51 min 30 sec & 05 min 53 sec \\
		\midrule
		Jumps Up & 50 min 40 sec & 05 min 12 sec \\
		\midrule
		No Jump & 65 min 45 sec & 06 min 01 sec \\
		\midrule
		Peaks & 61 min 31 sec & 05 min 55 sec \\
		\midrule
		Sequence Change & 49 min 53 sec & 06 min 41 sec \\
		\midrule
		Darpa & 185 min 49 sec & 05 min 29 sec \\
		\midrule
		Enron & 01 min 47 sec & 05 min 14 sec \\
		\bottomrule
	\end{tabular}
	\caption{�berblick der Performanz in [min]}
	\label{tab:performance}
\end{table}



\section{Ausblick}
\label{sec:fua-a}

\workTodo{Zudem sind beide Algorithmen f�r die statische Analyse ausgerichtet und k�nnten deswegen bspw. nur als Feature f�r den Netsimile verwendet werden Er erkennt Ausrei�er in Zeitreihen, kann erweitert werden um neue Merkmale als auch statischen Algorithmen, die als Merkmal dienen k�nnen. Zudem ist der Algorithmus performant und dynamisch.}

Im Rahmen des Forschungsprojekts wurden drei Thematiken behandelt. Zun�chst einmal wurde eine M�glichkeit ermittelt Zeitreihendaten in einen Graphen umzuwandeln. Im Anschluss wurden graphen-basierte Algorithmen auf die transformierten Zeitreihendaten angewendet, um diese Algorithmen im Anschluss hinsichtlich ihrer Eignung zur Ausrei�er-Erkennung zu vergleichen.

\workTodo{eventuell andere Distanzma�e zur Transformierung hinzuziehen? Da wir so auf den ersten Punkt genauer eingehen w�rden}

Durch die erfolgreiche Transformation von Zeitreihendaten in Graphen kann der Vorgang ebenso f�r andere Datenkategorien herangezogen werden unter der Pr�misse, dass Distanzen zwischen den einzelnen Elementen des Datensatzes gebildet werden k�nnen. So ist es im n�chsten Schritt m�glich bspw. Ausrei�er in  Finanzdaten oder Bilddaten zu erkennen. 

Dar�ber hinaus nimmt der Netsimile \workTodo{bis hierhin habe ich es ausgef�hrt}
Kommen wir nun zum Ausblick:

Interessant w�ren weitere Datens�tze, auf denen die Algorithmen angewendet werden wie bspw. Finanzdaten oder auch Bilder. Die einzige Voraussetzung w�re hier die Bildung einer Distanz, die zwischen den einzelnen Elementen des Datensatzes gebildet werden m�ssten.

Zudem bringt der Netsimile unserer Meinung nach noch nicht alles mit. Zum einen gibt der Netsimile nur den Ausrei�er-Graphen zur�ck. Zum anderen wird momentan der Graph erst dann berechnet, wenn dieser alle Kanten eines Zeitintervalls beinhaltet. Hier k�nnten Optimierungen folgen, die bswp. den Ausrei�er-Graphen auf Knoten oder Kanten untersucht, die f�r den Ausrei�erscore am meisten beigetragen haben. Zudem k�nnten man einen Weg suchen, um den Graphen iterativ vergr��ern zu k�nnen, damit eventuell schon vor dem vollst�ndigen Erstellen des Graphens herausgefunden werden kann, ob es sich um einen Ausrei�er handelt.

\textit{Aufschrieb}
Der Netsimile Algorithmus wurde bisher nur auf Zeitreihendaten und Graphen angewendet. Interessant w�ren jedoch noch Bilder, Biologische Daten und andere Netzwerke. Die einzige Voraussetzung ist die Berechnung einer Distanz zwischen einzelnen Elementen in diesen Daten. Zudem gibt es viele statische Algorithmen, die als Feature dem Netsimile hinzugef�gt werden k�nnen. Erkennt der Perculation Algorithmus bspw. Ausrei�er-Typen wie Increase-noise, Jumpsdown, Jumpsup und Flatmiddle, so kann nach weiteren Algorithmen gesucht werden, die die anderen Ausrei�er Typen erkennt. Zudem wird beim Netsimile Algorithmus lediglich der Graph zur�ckgegeben der als Ausrei�er identifiziert wurde, nicht aber der Grund daf�r. Hierf�r k�nnten weitere Untersuchungen durchgef�hrt werden, wie bspw. die Analyse der Features, welches sich am meisten von den anderen unterschieden hat oder einer anschlie�enden Analyse durch den Oddball Algorithmus, welche die Anomalien innerhalb des Graphens untersucht. 

Des weiteren k�nnte eine allgemeine Forschung gestartet werden, welche untersucht, welche Features f�r spezifische Anwendungsgebiete geeignet sein k�nnten. 


\workTodo{Ausblick schreiben}

\appendix
% !TeX spellcheck = de_DE
\newpage
\chapter{NetSimile}

\subsubsection{Eindimensionales Signal}
\label{sec:appendix_net_original}
\begin{figure}[H]
	\centering
	\subfloat[Ausrei�ertyp Signal Drift\label{img:dailyDriftNetOne}]{
		\includegraphics[width=0.5\textwidth]{fig/resultsNetismile/original/daily_drift}}
	\subfloat[Ausrei�ertyp Zunahme am Rauschen\label{img:increaseNoiseNetOne}]{
		\includegraphics[width=0.5\textwidth]{fig/resultsNetismile/original/increase_noise}}
	\qquad
	\subfloat[Ausrei�ertyp Einzelne Peaks\label{img:dailyPeaksNetOne}]{
		\includegraphics[width=0.5\textwidth]{fig/resultsNetismile/original/daily_peaks}}
	\subfloat[Ausrei�ertyp Frequenz�nderung\label{img:sequenceChangeNetONe}]{		\includegraphics[width=0.5\textwidth]{fig/resultsNetismile/original/sequence_change}}
	\qquad
\end{figure}
\begin{figure}\ContinuedFloat
	\subfloat[Ausrei�ertyp Kontinuierliche Zunahme der Amplitude\label{img:ampRiseNetOne}]{
		\includegraphics[width=0.5\textwidth]{fig/resultsNetismile/original/amp_rise}}
	\subfloat[Ausrei�ertyp Zyklus Aussetzer\label{img:flatmiddleNetOne}]{
		\includegraphics[width=0.5\textwidth]{fig/resultsNetismile/original/flatmiddle}}
	\qquad
	\subfloat[Ausrei�ertyp Zyklus mit geringerer Amplitude\label{img:jumpsdownNetOne}]{
		\includegraphics[width=0.5\textwidth]{fig/resultsNetismile/original/jumpsdown}}
	\subfloat[Ausrei�ertyp Zyklus mit h�herer Amplitude\label{img:jumpsupNetOne}]{
		\includegraphics[width=0.5\textwidth]{fig/resultsNetismile/original/jumpsup}}
	\qquad
\end{figure}
\begin{figure}\ContinuedFloat
	\subfloat[Ausrei�ertyp Signal-Aussetzer\label{img:nojumpNetOne}]{
		\includegraphics[width=0.5\textwidth]{fig/resultsNetismile/original/nojump}}
	\qquad
\end{figure}

%\subsubsection{Eindimensionales Signal Optimierte Version}
%\label{sec:appendix_net_one}
%\begin{figure}[H]
%	\centering
%	\subfloat[Caption for sub-figure1]{
%		\includegraphics[width=0.5\textwidth]{fig/resultsNetismile/1D/anomaly_art_daily_drift}}
%	\subfloat[Caption for sub-figure1]{
%		\includegraphics[width=0.5\textwidth]{fig/resultsNetismile/1D/anomaly_art_daily_increase_noise}}
%	\qquad
%	\subfloat[Caption for sub-figure1]{
%		\includegraphics[width=0.5\textwidth]{fig/resultsNetismile/1D/anomaly_art_daily_peaks}}
%	\subfloat[Caption for sub-figure1]{
%		\includegraphics[width=0.5\textwidth]{fig/resultsNetismile/1D/anomaly_art_daily_sequece_change}}
%	\qquad
%	\label{img:isomappictures1}
%\end{figure}
%\begin{figure}\ContinuedFloat
%	\subfloat[Caption for sub-figure1]{
%		\includegraphics[width=0.5\textwidth]{fig/resultsNetismile/1D/art_daily_amp_rise}}
%	\subfloat[Caption for sub-figure1]{
%		\includegraphics[width=0.5\textwidth]{fig/resultsNetismile/1D/art_daily_flatmiddle}}
%	\qquad
%	\subfloat[Caption for sub-figure1]{
%		\includegraphics[width=0.5\textwidth]{fig/resultsNetismile/1D/art_daily_jumpsdown}}
%	\subfloat[Caption for sub-figure1]{
%		\includegraphics[width=0.5\textwidth]{fig/resultsNetismile/1D/art_daily_jumpsup}}
%	\qquad
%	\subfloat[Caption for sub-figure1]{
%		\includegraphics[width=0.5\textwidth]{fig/resultsNetismile/1D/art_daily_nojump}}
%	\label{img:isomappictures2}
%\end{figure}

%\workTodo{Wrong picture for daily peaks. Change that the sixed element is not always an outlier}

\newpage
\subsubsection{Zweidimensionales Signal}
\label{sec:appendix_net_two}
\begin{figure}[H]
	\centering
	\subfloat[Ausrei�er-Typ Signal Drift\label{img:dailyDriftNetTwo}]{
		\includegraphics[width=0.5\textwidth]{fig/resultsNetismile/2D/anomaly_art_daily_drift}}
	\subfloat[Ausrei�er-Typ Zunahme an Rauschen\label{img:increaseNoiseNetTwo}]{
		\includegraphics[width=0.5\textwidth]{fig/resultsNetismile/2D/anomaly_art_daily_increase_noise}}
	\qquad
	\subfloat[Ausrei�er-Typ Einzelne Peaks\label{img:dailyPeaksNetTwo}]{
		\includegraphics[width=0.5\textwidth]{fig/resultsNetismile/2D/anomaly_art_daily_peaks}}
	\subfloat[Ausrei�er-Typ Frequenz�nderung\label{img:sequenceChangeNetTwo}]{
		\includegraphics[width=0.5\textwidth]{fig/resultsNetismile/2D/anomaly_art_daily_sequence_change}}
	\qquad
	\label{img:isomappictures1}
\end{figure}
\begin{figure}\ContinuedFloat
	\subfloat[Ausrei�er-Typ Kontinuierliche Zunahme der Amplitude\label{img:ampRiseNetTwo}]{
		\includegraphics[width=0.5\textwidth]{fig/resultsNetismile/2D/art_daily_amp_rise}}
	\subfloat[Ausrei�er-Typ Zyklus-Aussetzer\label{img:flatmiddleNetTwo}]{
		\includegraphics[width=0.5\textwidth]{fig/resultsNetismile/2D/art_daily_flatmiddle}}
	\qquad
	\subfloat[Ausrei�er-Typ Zyklus mit geringerer Amplitude\label{img:jumpsdownNetTwo}]{
		\includegraphics[width=0.5\textwidth]{fig/resultsNetismile/2D/art_daily_jumpsdown}}
	\subfloat[Ausrei�er-Typ Zyklus mit h�herer Amplitude\label{img:jumpsupNetTwo}]{
		\includegraphics[width=0.5\textwidth]{fig/resultsNetismile/2D/art_daily_jumpsup}}
	\qquad
	\subfloat[Ausrei�er-Typ Signal-Aussetzer\label{img:nojumpNetTwo}]{
		\includegraphics[width=0.5\textwidth]{fig/resultsNetismile/2D/art_daily_nojump}}
	\label{img:isomappictures2}
\end{figure}

\newpage
\subsubsection{Ergebnisse vollst�ndiger Graphen}
\begin{figure}[H]
	\centering
	\includegraphics[height=0.8\linewidth,width=\linewidth]{fig/netsimile/heatmap_1}
	\caption{Signaturvektoren der Zeitreihe mit vollst�ndigen Graphen}
	\label{img:netsimile:heatmap_1}
\end{figure}

\begin{figure}[H]
	\centering
	\includegraphics[height=0.6\linewidth,width=\linewidth]{fig/netsimile/anomalie_1}
	\caption{Ausrei�er Score der vollst�ndigen Graphen}
	\label{img:netsimile:anomalie_1}
\end{figure}
\newpage
\chapter{Midas}
\subsubsection{Eindimensionales Signal}
\label{chap:appendix_midas_ts}
\begin{figure}[h]
	\centering
	\subfloat[Ausrei�er-Typ Signal Drift\label{img:dailyDriftIso}]{
		\includegraphics[width=0.5\textwidth]{fig/resultsMidasTS/anomaly_art_daily_drift_MIDAS_hdrei_labeled_result}}
	\subfloat[Ausrei�er-Typ Zunahme an Rauschen\label{img:increaseNoiseIso}]{
		\includegraphics[width=0.5\textwidth]{fig/resultsMidasTS/anomaly_art_daily_increase_noise_MIDAS_hdrei_labeled_result}}
	\qquad
	\subfloat[Ausrei�er-Typ Einzelne Peaks\label{img:dailyPeaksIso}]{
		\includegraphics[width=0.5\textwidth]{fig/resultsMidasTS/anomaly_art_daily_peaks_MIDAS_hdrei_labeled_result}}
	\subfloat[Ausrei�er-Typ Frequenz�nderung\label{img:sequenceChangeIso}]{
		\includegraphics[width=0.5\textwidth]{fig/resultsMidasTS/anomaly_art_daily_sequence_change_MIDAS_hdrei_labeled_result}}
	\qquad
	\label{img:midaspictures1}
\end{figure}
\begin{figure}\ContinuedFloat
	\subfloat[Ausrei�er-Typ Kontinuierliche Zunahme der Amplitude\label{img:ampRiseIso}]{
		\includegraphics[width=0.5\textwidth]{fig/resultsMidasTS/art_daily_amp_rise_MIDAS_hdrei_labeled_result}}
	\subfloat[Ausrei�er-Typ Zyklus-Aussetzer\label{img:flatmiddleIso}]{
		\includegraphics[width=0.5\textwidth]{fig/resultsMidasTS/art_daily_flatmiddle_MIDAS_hdrei_labeled_result}}
	\qquad
	\subfloat[Ausrei�er-Typ Zyklus mit geringerer Amplitude\label{img:jumpsdownIso}]{
		\includegraphics[width=0.5\textwidth]{fig/resultsMidasTS/art_daily_jumpsdown_MIDAS_hdrei_result}}
	\subfloat[Ausrei�er-Typ Zyklus mit h�herer Amplitude\label{img:jumpsupIso}]{
		\includegraphics[width=0.5\textwidth]{fig/resultsMidasTS/art_daily_jumpsup_MIDAS_hdrei_labeled_result}}
	\qquad
	\subfloat[Ausrei�er-Typ Signal-Aussetzer\label{img:nojumpIso}]{
		\includegraphics[width=0.5\textwidth]{fig/resultsMidasTS/art_daily_nojump_MIDAS_hdrei_result}}
	\label{img:midaspictures2}
\end{figure}
\newpage
\chapter{Isomap}
\subsubsection{Eindimensionales Signal}
\begin{figure}[h]
	\centering
	\subfloat[Ausrei�er-Typ Signal Drift\label{img:dailyDriftIso}]{
		\includegraphics[width=0.5\textwidth]{fig/resultsIsoMap/anomaly_art_daily_drift}}
	\subfloat[Ausrei�er-Typ Zunahme an Rauschen\label{img:increaseNoiseIso}]{
		\includegraphics[width=0.5\textwidth]{fig/resultsIsoMap/anomaly_art_daily_increase_noise}}
	\qquad
	\subfloat[Ausrei�er-Typ Einzelne Peaks\label{img:dailyPeaksIso}]{
		\includegraphics[width=0.5\textwidth]{fig/resultsIsoMap/anomaly_art_daily_peaks}}
	\subfloat[Ausrei�er-Typ Frequenz�nderung\label{img:sequenceChangeIso}]{
		\includegraphics[width=0.5\textwidth]{fig/resultsIsoMap/anomaly_art_daily_sequence_change}}
	\qquad
\end{figure}
\begin{figure}\ContinuedFloat
	\subfloat[Ausrei�er-Typ Kontinuierliche Zunahme der Amplitude\label{img:ampRiseIso}]{
		\includegraphics[width=0.5\textwidth]{fig/resultsIsoMap/art_daily_amp_rise}}
	\subfloat[Ausrei�er-Typ Zyklus-Aussetzer\label{img:flatmiddleIso}]{
		\includegraphics[width=0.5\textwidth]{fig/resultsIsoMap/art_daily_flatmiddle}}
	\qquad
	\subfloat[Ausrei�er-Typ Zyklus mit geringerer Amplitude\label{img:jumpsdownIso}]{
		\includegraphics[width=0.5\textwidth]{fig/resultsIsoMap/art_daily_jumpsdown}}
	\subfloat[Ausrei�er-Typ Zyklus mit h�herer Amplitude\label{img:jumpsupIso}]{
		\includegraphics[width=0.5\textwidth]{fig/resultsIsoMap/art_daily_jumpsup}}
	\qquad
	\subfloat[Ausrei�er-Typ Signal-Aussetzer\label{img:nojumpIso}]{
		\includegraphics[width=0.5\textwidth]{fig/resultsIsoMap/art_daily_nojump}}
\end{figure}






\newpage
\chapter{Percolation}
\subsubsection{Eindimensionales Signal}
\label{app-perc}
\begin{figure}[h]
	\centering
	\subfloat[Ausrei�er-Typ Signal Drift\label{img:dailyDriftPerc}]{
		\includegraphics[width=0.5\textwidth]{fig/resultsPercolation/anomaly_art_daily_drift}}
	\subfloat[Ausrei�er-Typ Zunahme an Rauschen\label{img:increaseNoisePerc}]{
		\includegraphics[width=0.5\textwidth]{fig/resultsPercolation/anomaly_art_daily_increase_noise}}
	\qquad
	\subfloat[Ausrei�er-Typ Einzelne Peaks\label{img:dailyPeaksPerc}]{
		\includegraphics[width=0.5\textwidth]{fig/resultsPercolation/anomaly_art_daily_peaks}}
	\subfloat[Ausrei�er-Typ Frequenz�nderung\label{img:sequenceChangePerc}]{
		\includegraphics[width=0.5\textwidth]{fig/resultsPercolation/anomaly_art_daily_sequence_change}}
	\qquad
	\label{img:isomappictures1}
\end{figure}
\begin{figure}\ContinuedFloat
	\subfloat[Ausrei�er-Typ Kontinuierliche Zunahme der Amplitude\label{img:ampRisePerc}]{
		\includegraphics[width=0.5\textwidth]{fig/resultsPercolation/art_daily_amp_rise}}
	\subfloat[Ausrei�er-Typ Zyklus-Aussetzer\label{img:flatmiddlePerc}]{
		\includegraphics[width=0.5\textwidth]{fig/resultsPercolation/art_daily_flatmiddle}}
	\qquad
	\subfloat[Ausrei�er-Typ Zyklus mit geringerer Amplitude\label{img:jumpsdownPerc}]{
		\includegraphics[width=0.5\textwidth]{fig/resultsPercolation/art_daily_jumpsdown}}
	\subfloat[Ausrei�er-Typ Zyklus mit h�herer Amplitude\label{img:jumpsupPerc}]{
		\includegraphics[width=0.5\textwidth]{fig/resultsPercolation/art_daily_jumpsup}}
	\qquad
	\subfloat[Ausrei�er-Typ Signal-Aussetzer\label{img:nojumpPerc}]{
		\includegraphics[width=0.5\textwidth]{fig/resultsPercolation/art_daily_nojump}}
	\label{img:isomappictures2}
\end{figure}


\newpage



% % %%%%%% Anhang
%\appendix
%\chapter{OddBall}
\label{chap:a-ob}
\section{Threshold-Ergebnisse}
\label{chap:a-ob-th}
\begin{table}[H]
	\centering
	\includegraphics[width=\textwidth, height=0.72\textheight]{fig/tabelle-threshold-1}
	\caption{Threshold-Ergebnisse}
\end{table}

\section{Local-Outlier-Factor-Ergebnisse}
\label{chap:a-ob-lof}

\begin{table}[H]
\centering
\includegraphics[width=\textwidth]{fig/tabelle-lof-1}
\end{table}

\begin{table}[H]
\centering
\includegraphics[width=\textwidth]{fig/tabelle-lof-2}
\end{table}

\begin{table}[H]
\centering
\includegraphics[width=\textwidth]{fig/tabelle-lof-3}
\end{table}

\begin{table}[H]
\centering
\includegraphics[width=\textwidth]{fig/tabelle-lof-4}
\caption{Local-Outlier-Factor-Ergebnisse}
\end{table}


\section{Local-Outlier-Probability-Ergebnisse}
\label{chap:a-ob-loop}
\begin{table}[H]
	\centering
	\includegraphics[width=\textwidth]{fig/tabelle-loop-1}
\end{table}

\begin{table}[H]
	\centering
	\includegraphics[width=\textwidth]{fig/tabelle-loop-2}
\end{table}

\begin{table}[H]
\centering
\includegraphics[width=\textwidth]{fig/tabelle-loop-3}
\end{table}

\begin{table}[H]
\centering
\includegraphics[width=\textwidth]{fig/tabelle-loop-4}
\end{table}

\begin{table}[H]
\centering
\includegraphics[width=\textwidth]{fig/tabelle-loop-5}
\end{table}

\begin{table}[H]
\centering
\includegraphics[width=\textwidth]{fig/tabelle-loop-6}
\caption{Local-Outlier-Probability-Ergebnisse}
\end{table}

\section{Vergleich LOF und LoOP}
\label{chap:a-ob-lof-loop}

\begin{table}[H]
	\centering
	\includegraphics[width=\textwidth]{fig/tabelle-lof-loop}
	\caption{Vergleich LOF und LoOP}
\end{table}

\section{DBSCAN-Ergebnisse}
\label{chap:a-ob-dbscan}

\begin{table}[H]
	\centering
	\includegraphics[width=\textwidth]{fig/tabelle-dbscan-1}
\end{table}

\begin{table}[H]
\centering
\includegraphics[width=\textwidth]{fig/tabelle-dbscan-2}
\end{table}

\begin{table}[H]
\centering
\includegraphics[width=\textwidth]{fig/tabelle-dbscan-3}
\caption{DBSCAN-Ergebnisse}
\end{table}

\section{DBSCAN+LoOP-Ergebnisse}
\label{chap:a-ob-dbscan-loop}

\begin{table}[H]
	\centering
	\includegraphics[width=\textwidth]{fig/tabelle-dbscan-loop-1}
\end{table}

\begin{table}[H]
\centering
\includegraphics[width=\textwidth]{fig/tabelle-dbscan-loop-4}
\caption{DBSCAN+LoOP-Ergebnisse}
\end{table}

\chapter{SCAN-Ergebnisse}
\label{chap:a-ob-scan}

\begin{table}[H]
	\centering
	\includegraphics[width=\textwidth]{fig/tabelle-scan-1}
\end{table}

\begin{table}[H]
\centering
\includegraphics[width=\textwidth]{fig/tabelle-scan-2}
\end{table}

\begin{table}[H]
\centering
\includegraphics[width=\textwidth]{fig/tabelle-scan-3}
\end{table}

\begin{table}[H]
\centering
\includegraphics[width=\textwidth]{fig/tabelle-scan-4}
\end{table}

\begin{table}[H]
\centering
\includegraphics[width=\textwidth]{fig/tabelle-scan-5}
\end{table}

\begin{table}[H]
\centering
\includegraphics[width=\textwidth]{fig/tabelle-scan-6}
\end{table}

\begin{table}[H]
\centering
\includegraphics[width=\textwidth]{fig/tabelle-scan-7}
\caption{SCAN-Ergebnisse}
\end{table}


\chapter{Random-Walk}
\label{chap:a-rw}
Die nachfolgend aufgelisteten Graphen zeigen die Ergebnisse der Ausrei�er Erkennung mithilfe des Random-Walk Algorithmus. Die Graphen bestehen dabei aus unterschiedlichen Zeitreihen, den identifizierten Ausrei�ern sowie dem Ausrei�er-Score. In den Kapiteln \autoref{chap:a-rw-3D} und \autoref{chap:a-rw-5D} wurden die Dimensionen 2-3 bzw. 2-5 nicht dargestellt, um die �bersichtlichkeit zu gew�hrleisten.

\section{Zweidimensionales Signal}
\label{chap:a-rw-2D}
\begin{figure}[!htbp]
	\centering
	\begin{minipage}[t]{0.45\textwidth}
		\includegraphics[width=\textwidth]{fig/rw-2d-1}
		\caption{Anomalie: Einzelne Peaks}
	\end{minipage}
	\begin{minipage}[t]{0.45\textwidth}
		\includegraphics[width=\textwidth]{fig/rw-2d-2}
		\caption{Anomalie: Zunahme an Rauschen}
	\end{minipage}
\end{figure}

\begin{figure}[!htbp]
	\centering
	\begin{minipage}[t]{0.45\textwidth}
		\includegraphics[width=\textwidth]{fig/rw-2d-3}
		\caption{Anomalie: Signal Drift}
	\end{minipage}
	\begin{minipage}[t]{0.45\textwidth}
		\includegraphics[width=\textwidth]{fig/rw-2d-4}
		\caption{Anomalie: Zunahme der Amplitude}
	\end{minipage}
\end{figure}

\begin{figure}[!htbp]
	\centering
	\begin{minipage}[t]{0.49\textwidth}
		\includegraphics[width=\textwidth]{fig/rw-2d-5}
		\caption{Anomalie: Zyklus mit h�herer Amplitude}
	\end{minipage}
	\begin{minipage}[t]{0.49\textwidth}
		\includegraphics[width=\textwidth]{fig/rw-2d-6}
		\caption{Anomalie: Zyklus mit geringerer Amplitude}
	\end{minipage}
\end{figure}

\begin{figure}[!htbp]
	\centering
	\begin{minipage}[t]{0.49\textwidth}
		\includegraphics[width=\textwidth]{fig/rw-2d-7}
		\caption{Anomalie: Zyklus Aussetzer}
	\end{minipage}
	\begin{minipage}[t]{0.49\textwidth}
		\includegraphics[width=\textwidth]{fig/rw-2d-8}
		\caption{Anomalie: Signal Aussetzer}
	\end{minipage}
\end{figure}

\begin{figure}
	\centering
	\begin{minipage}[t]{0.49\textwidth}
		\includegraphics[width=\textwidth]{fig/rw-2d-9}
		\caption{Anomalie: Frequenz�nderung der Zyklen}
	\end{minipage}
\end{figure}

\section{Dreidimensionales Signal}
\label{chap:a-rw-3D}

\begin{figure}[!htbp]
	\centering
	\begin{minipage}[t]{0.45\textwidth}
		\includegraphics[width=\textwidth]{fig/rw-3d-1}
		\caption{Anomalie: Einzelne Peaks}
	\end{minipage}
	\begin{minipage}[t]{0.45\textwidth}
		\includegraphics[width=\textwidth]{fig/rw-3d-2}
		\caption{Anomalie: Zunahme an Rauschen}
	\end{minipage}
\end{figure}

\begin{figure}[!htbp]
	\centering
	\begin{minipage}[t]{0.45\textwidth}
		\includegraphics[width=\textwidth]{fig/rw-3d-3}
		\caption{Anomalie: Signal Drift}
	\end{minipage}
	\begin{minipage}[t]{0.45\textwidth}
		\includegraphics[width=\textwidth]{fig/rw-3d-4}
		\caption{Anomalie: Zunahme der Amplitude}
	\end{minipage}
\end{figure}

\begin{figure}[!htbp]
	\centering
	\begin{minipage}[t]{0.45\textwidth}
		\includegraphics[width=\textwidth]{fig/rw-3d-5}
		\caption{Anomalie: Zyklus mit h�herer Amplitude}
	\end{minipage}
	\begin{minipage}[t]{0.45\textwidth}
		\includegraphics[width=\textwidth]{fig/rw-3d-6}
		\caption{Anomalie: Zyklus mit geringerer Amplitude}
	\end{minipage}
\end{figure}

\begin{figure}[!htbp]
	\centering
	\begin{minipage}[t]{0.45\textwidth}
		\includegraphics[width=\textwidth]{fig/rw-3d-7}
		\caption{Anomalie: Zyklus-Aussetzer}
	\end{minipage}
	\begin{minipage}[t]{0.45\textwidth}
		\includegraphics[width=\textwidth]{fig/rw-3d-8}
		\caption{Anomalie: Signal-Aussetzer}
	\end{minipage}
\end{figure}

\begin{figure}
	\centering
	\begin{minipage}[t]{0.45\textwidth}
		\includegraphics[width=\textwidth]{fig/rw-3d-9}
		\caption{Anomalie: Frequenz�nderung der Zyklen}
	\end{minipage}
\end{figure}
\newpage

\section{F�nfdimensionales Signal}
\label{chap:a-rw-5D}

\begin{figure}[!htbp]
	\centering
	\begin{minipage}[t]{0.45\textwidth}
		\includegraphics[width=\textwidth]{fig/rw-5d-1}
		\caption{Anomalie: Einzelne Peaks}
	\end{minipage}
	\begin{minipage}[t]{0.45\textwidth}
		\includegraphics[width=\textwidth]{fig/rw-5d-2}
		\caption{Anomalie: Zunahme an Rauschen}
	\end{minipage}
\end{figure}

\begin{figure}[!htbp]
	\centering
	\begin{minipage}[t]{0.45\textwidth}
		\includegraphics[width=\textwidth]{fig/rw-5d-3}
		\caption{Anomalie: Signal Drift}
	\end{minipage}
	\begin{minipage}[t]{0.45\textwidth}
		\includegraphics[width=\textwidth]{fig/rw-5d-4}
		\caption{Anomalie: Zunahme der Amplitude}
	\end{minipage}
\end{figure}

\begin{figure}[!htbp]
	\centering
	\begin{minipage}[t]{0.45\textwidth}
		\includegraphics[width=\textwidth]{fig/rw-5d-5}
		\caption{Anomalie: Zyklus mit h�herer Amplitude}
	\end{minipage}
	\begin{minipage}[t]{0.45\textwidth}
		\includegraphics[width=\textwidth]{fig/rw-5d-6}
		\caption{Anomalie: Zyklus mit geringerer Amplitude}
	\end{minipage}
\end{figure}

\begin{figure}[!htbp]
	\centering
	\begin{minipage}[t]{0.45\textwidth}
		\includegraphics[width=\textwidth]{fig/rw-5d-7}
		\caption{Anomalie: Zyklus-Aussetzer}
	\end{minipage}
	\begin{minipage}[t]{0.45\textwidth}
		\includegraphics[width=\textwidth]{fig/rw-5d-8}
		\caption{Anomalie: Signal-Aussetzer}
	\end{minipage}
\end{figure}

\begin{figure}[H]
		\centering
		\includegraphics[width=8cm]{fig/rw-5d-9}
		\caption{Anomalie: Frequenz�nderung der Zyklen}
\end{figure}

% % %%%%%% Literaturverzeichnis (darf im deutschen nicht in den Anhang!)
% Einfaches Literaturverzeichnis
%\input{chapters/ch-zz-bibEinfach}
% Literaturverzeichnis mit Bibtex

\bibliography{bib/bib}
\bibliographystyle{agsm}

% %  Inhalt ENDE %%%%%%%%%%%%%%%%%%%%%%%%%%%%%%%%%%%%%%%%%%%%%%%%%%%%%%%%%%
\end{document}
